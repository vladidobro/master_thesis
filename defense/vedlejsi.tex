\begin{frame}{Vedlejší přínosy -- Automatizace experimentu}
    \begin{itemize}
        \item Vytvoření softwarového balíčku v programovacím jazyku Python
            umožňující automatický běh experimentu
            \begin{itemize}
                \item Bohužel až po dokončení měření uvedených v diplomové práci
            \end{itemize}
    \end{itemize}
\end{frame}


\begin{frame}{Vedlejší přínosy -- Poruchová teorie optiky anizotropních multivrstev}
    \centering
    \includegraphics[width=0.7\textwidth]{appendix1.pdf}
\end{frame}

\begin{frame}{Vedlejší přínosy -- Poruchová teorie optiky anizotropních multivrstev}
    \begin{itemize}
        \item Magneto-optika se zajímá o změny v odraženém (prošlém) světle při malých změnách permitivity
        \item Výpočet v rámci Maxwellových rovnic
            \begin{itemize}
                \item Dosavadní výpočty počítaly vlastní módy v porušeném prostředí
                    \begin{itemize}
                        \item Analyticky řešitelné jen ve velmi speciálních případech
                        \item Matematické a numerické problémy
                        \item Výsledek je na konci stejně aproximován pro malou poruchu permitivity
                    \end{itemize}
                \item Nový přístup počítá přímo mocninný rozvoj v poruše permitivity
                    \begin{itemize}
                        \item Počítá pouze vlastní módy NEporušeného prostředí
                        \item Poskytuje lepší náhled na 
                        \item Je velmi obecnější, dovoluje dosud nevyřešené situace (vše lze ale samozřejmě počítat numericky)
                    \end{itemize}
            \end{itemize}
    \end{itemize}
\end{frame}


\begin{frame}{Vedlejší přínosy -- ``Stokesovy kovektory''}
    \centering
    \includegraphics[width=0.7\textwidth]{appendix3.pdf}
\end{frame}

\begin{frame}{Vedlejší přínosy -- ``Stokesovy kovektory''}
    \begin{itemize}
        \item Formalismus pro popis optických detekčních schémat lineárních ve vstupní intenzitě
            \begin{itemize}
                \item Např. zkřížené polarizátory, optický můstek a příbuzná schémata
            \end{itemize}
        \item Názorná geometrická interpretace
        \item Umožňuje snadné uvažování o vlivu nedokonalostí elementů detekční soustavy
        \item Detekční soustava je reprezentovaná (soustavou) rovnoběžek na Poincarého sféře, optické elementy je určitým způsobem transformují
    \end{itemize}
    \begin{center}
        \includegraphics{mustek.pdf}
        \includegraphics[width=4cm]{asy/kovek-1.pdf}
    \end{center}
    
\end{frame}

