\section{Detekce}
\label{chap:detekce}

V tomto oddílu se zaměříme na první z problémů, který je způsobený tím, že v uspořádání na obr. \ref{fig:zakladni-schema} neměříme přímo stočení polarizace vzorkem, tj. neplatí vzorec \eqref{eqn:mustek-delta-beta}: $\Udif/2\Usum\neq\Delta\beta$.
Příčiny jsou dvě: zrcadla mezi vzorkem a můstkem měnící polarizaci, a nedokonalost vlnové destičky a děliče v můstku.

V tomto oddílu naplno využijeme formalismus \emph{Stokesových kovektorů} -- lineárních forem na prostoru Stokesových vektorů, které popisují detektory a složitější detekční systémy složené z detektorů a optických prvků -- rozvinutý v dodatku \ref{app:kovektory}.
Pomocí něj vysvětlíme podstatu problémů a popíšeme několik způsobů, jakými vliv zrcadel a nedokonalostí prvků kompenzovat.
Nakonec v oddílu \ref{chap:elipticita} popíšeme způsob využití Berekova kompenzátoru v můstku, který umožňuje současné měření stočení i elipticity.

Pro další použití na obr. \ref{fig:kovektor-ideal-mustek} vykreslujeme kovektory můstku s idealizovanými prvky (jako v oddílu \ref{chap:mustek-kap2}).
$\Dtks'$ značíme kovektory vzhledem ke světlu před děličem, $\Dtks''$ vzhledem ke světlu před destičkou.

\begin{figure}[htbp]
    \centering
    \missingfigure{kovektory idealni}
    \caption{Stokesovy kovektory ideálního můstku. (a) Vzhledem ke světlu před děličem. (b) Vzhledem ke světlu před destičkou. (c) Ilustrace významu $\Dtks'$ a $\Dtks''$.}
    \label{fig:kovektor-ideal-mustek}
\end{figure}

\subsection{Kompenzace nedokonalostí a zrcadel}
Jedna z výhod formalismu Stokesových kovektorů snadné uvažování o tom, jak se detekční aparatura chová při malých změnách (nedokonalostech) optických prvků.
Zaměříme se na tři druhy druhy nedokonalostí, z nichž se nakonec jediná vyplatí kompenzovat -- nepřesné fázové zpoždění půlvlnné destičky.

Pro vyvažovací půlvlnnou destičku uvažujeme dva druhy nedokonalosti: rozdílnou propustnost obou módů a fázové zpoždění lišící se od přesné hodnoty $\pi/2$.
Kvůli symetrii však stále požadujeme, aby destička měla dvě navzájem kolmé optické osy -- dva vlastní módy navzájem ortogonálních lineárních polarizací.

Experimentálně bylo ověřeno, že polarizační dělič vysoce kvalitně dělí svazek na dvě ortogonální lineární polarizace.
V odraženém svazku je sice zastoupeny obě polarizace, ale kvůli mírně odlišnému úhlu lomu se prostorově oddělí.
Vložením polarizátoru před dělič a jeho vhodným otáčením bylo možné ho zkřížit vzhledem k oběma ramenům (zvlášť) s extinkčním poměrem $I_\textrm{min}/I_\textrm{max} \approx \num{1e-4}$.
Díky tomu se nijak neprojeví ani případná polarizační závislost detektorů.
Jediná nedokonalost zbytku můstku (nezahrnující půlvlnnou destičku) je tedy vyjádřena rozdílnou citlivostí obou ramen na příslušné lineární polarizace, která je způsobena jak rozdílnou propustností/odrazivostí děliče, tak nevyváženou citlivostí a zesílení obou detektorů.

Všechny tři zmíněné nedokonalosti uvažujeme zvlášť a zanedbáváme jejich vzájemné působení.
Nakonec se zaměříme na to, co se stane, když před můstek umístíme retardér -- zrcadlo nezbytné pro oddělení dopadajícího a odraženého svazku v reflexní geometrii, a pro vyvedení svazku ven z komory kryostatu v transmisní geometrii.


\subsubsection*{Nevyváženost ramen}
Rozdíl citlivostí ramen je vyjádřen $\eta$, takže vzhledem ke světlu před děličem
\begin{align}
    {\Dtks'}^\textrm{A}=\frac{1+\eta}{2}(1, 1, 0, 0) \,,\\
    {\Dtks'}^\textrm{B}=\frac{1-\eta}{2}(1, -1, 0, 0) \,,\\
    {\Dtks'}^\textrm{A-B}=(\eta, 1, 0, 0) \,,\\
    {\Dtks'}^\textrm{A+B}=(1, \eta, 0, 0) 
\end{align}
a vzhledem ke světlu před destičkou
\begin{align}
    {\Dtks''}^\textrm{A-B}=(\eta, \cos(\theta_{\lambda/2}), \cos(\theta_{\lambda/2}), 0) \,,\\
    {\Dtks''}^\textrm{A+B}=(1, \eta\cos(\theta_{\lambda/2}), \eta\cos(\theta_{\lambda/2}), 0) 
\end{align}

Prvním z projevů je polarizační závislost součtového signálu (${d'_1}^\textrm{A+B}=\eta\neq=0$).
To není velký problém, protože se ve vzorci $\Delta\beta=\Udif/2\Usum$ používá pouze pro normalizaci signálu, která lze určit i jiným způsobem (viz např. oddíl \ref{chap:elipticita}). 
Pro malé $\eta$ však není od věci tento vliv ignorovat.

Druhou známkou nevyvážených ramen je nenulové  ${d'_0}^\textrm{A-B}$, které se projeví posunutím nulové kružnice, viz obr. \ref{fig:mustek-nedokonale-ramena} (a), po otočení půlvlnnou destičkou pak \ref{fig:mustek-nedokonale-ramena} (b).
Důležitým rysem je, že nulová kružnice je vždy kolmá na rovník lineárních polarizací, což znamená, že $\partial\Udif/\partial\chi = 0$, a tedy v měřeném signálu se neprojeví změny elipticity\footnote{V prvním řádu, pokud do můstku vstupuje lineárně polarizované světlo.}.
I druhý projev tedy v důsledku pouze mění konstantu úměrnosti, pro $\chi=0$ platí
\begin{equation}
    \textrm{d}\Udif = 2I\cos(\eta) \textrm{d}\beta \,,
\end{equation}
což pro malé $\eta$ ignorujeme.

\begin{figure}[htbp]
    \centering
    \missingfigure{nedokonale ramena}
    \caption{Kovektory můstku s nevyváženými rameny pro světlo (a) před děličem, (b) před destičkou.}
    \label{fig:mustek-nedokonale-ramena}
\end{figure}

\subsubsection*{Rozdílná propustnost destičky}
Jonesova a Muellerova matice destičky s přesným fázovým zpožděním $\pi/2$, ale rozdílnou propustností obou lineárních polarizací je (pro $\theta_{\lambda/2}=0$)
\begin{equation}
    \mathcal{T}_{\lambda/2} = \begin{pmatrix} 1&0 \\ 0&-\eta \end{pmatrix} \,, \quad
    \M_{\lambda/2} = \begin{pmatrix} 1&0&0&0 \\ 0&0&0&0 \\ 0&0&0&0 \\ 0&0&0&-1 \end{pmatrix} \,,
\end{equation}
kde nedokonalost destičky je vyjádřena $\eta$ (ideální destička má $\eta=0$)
Akce takové Muellerovy matice je kombinace otočení o \SI{180}{\degree} a protáhnutí/posunutí kolem stejné osy\footnote{To lze nahlédnout z toho, že lze Jonesovu matici psát jako součin unitární a pozitivně semi-definitní hermitovské matice, které spolu komutují: $\mathcal{T}_{\lambda/2} = \begin{pmatrix} 1&0\\0&-1 \end{pmatrix} \begin{pmatrix} 1&0\\0&\eta \end{pmatrix}$.} -- procházející lineárními polarizacemi ve směru optické osy $\theta_{\lambda/2}$.

Výsledkem je stejně jako v případě nevyvážených ramen posunutí nulové kružnice (${d''_0}^\textrm{A-B}\neq0$), které v souladu s diskuzí v předešlém oddílu pro malé $\eta$ ignorujeme.

\subsubsection*{Nepřesné fázové zpoždění destičky}
Jonesova a Muellerova matice je (pro $\theta_{\lambda/2}=0$)
\begin{equation}
    \mathcal{T}_{\lambda/2} = \begin{pmatrix} 1&0 \\ 0&-e^{i\delta} \end{pmatrix} \,, \quad
    \M_{\lambda/2} = \begin{pmatrix} 1&0&0&0 \\ 0&0&0&0 \\ 0&0&0&0 \\ 0&0&0&-1 \end{pmatrix} \,,
\end{equation}
kde nedokonalost je vyjádřena dodatečným fázovým zpožděním $\delta$ (ideální destička má $\delta=0$).
Muellerova matice působí jako rotace o $\SI{180}{\degree}+\delta$ kolem osy vlastních módů.
Kvůli $\delta\neq0$ již nejsou obě osy destičky ekvivalentní.
Pozice destičky $\theta_{\lambda/2}$ a $\theta_{\lambda/2}+\SI{90}{\degree}$ tedy také nejsou ekvivalentní, odpovídají rotacím kolem té stejné osy o $\SI{180}{\degree}+\delta$, ale opačným směrem\footnote{Ekvivalentně s opačným znaménkem $\delta$.}.
Pozice $\theta_{\lambda/2}$ a $\theta_{\lambda/2}+\SI{180}{\degree}$ zůstávají nadále ekvivalentní.

Transformací ideálního kovektoru ${\Dtks'}^\textrm{A-B}=(0, 1, 0, 0)$ (vzhledem ke světlu před děličem) neideální Muellerovou maticí $\M_{\lambda/2}$ je zachováno ${d''_0}^\textrm{A-B}=0$.
Oproti předchozím případům však již není zachovaná kolmost nulové kružnice a rovníku lineárních polarizací, tj. ${d''_3}^\textrm{A-B}\neq0$.
Platí
\begin{equation}
    {\Dtks''}^\textrm{A-B} \equiv (0,\, \cos2\zeta_2\cos2\zeta_1,\, \cos2\zeta_2\sin2\zeta_1,\, \sin2\zeta_1) \,,
\end{equation}
kde jsme označili úhel průsečíku kružnice s rovníkem $\zeta_1(\theta_{\lambda/2})$ a jimi svíraný úhel $\zeta_2(\theta_{\lambda/2})$, viz obr. \ref{fig:mustek-nedokonale-desticka}.

\begin{figure}[htbp]
    \centering
    \missingfigure{mustek destika nepresna}
    \caption{Kovektor můstku s destičkou s nepřesným fázovým zpožděním. Je naznačen také diferenciál Stokesova vektoru.}
    \label{fig:mustek-nedokonale-desticka}
\end{figure}

Vyvážením můstku pro dané $\beta$ je nastaveno takové $\theta_{\lambda/2}$, aby platilo $\beta=\zeta_1(\theta_{\lambda/2})$.
Pak platí (viz \eqref{eqn:Stokes-diferencial} pro tvar $\textrm{d}\Stks$)
\begin{equation}
\label{eqn:mustek-s-elipticitou}
    \textrm{d}\Udif = {\Dtks''}^\textrm{A-B} \cdot \textrm{d}\Stks^\textrm{in} = 2I\left(\cos\zeta_2 \textrm{d}\beta + \sin\zeta_2\textrm{d}\chi\right) 
\end{equation}
s $\zeta_2$ dosazeným pro vyváženou polohu $\theta_{\lambda/2}$.

Zde však už narážíme na závažný problém: v měřeném signálu se nám projevují i změny elipticity neznámým faktorem $\sin\zeta_2$, které se navíc mění (i znaménko), v závislosti na $\beta$.
Řešení nabízí\todopn{zmínit TO?} skutečnost\footnote{Lze nahlédnout ze symetrie vůči obrácení točivosti světla či přímým výpočtem.}, že $\zeta_2$ je pro každé $\beta$ lichou funkcí nedokonalosti destičky $\delta$ -- mění znaménko při vyvážení polohami $\theta_{\lambda/2}$, resp. $\theta_{\lambda/2}+\SI{90}{\degree}$.
V praxi tedy měříme každé $\beta$ s oběma polohami destičky a bereme jejich aritmetický průměr
\begin{equation}
    \frac{1}{2}\left(\Udif^{\theta_{\lambda/2}}+\Udif^{\theta_{\lambda/2}+\SI{90}{\degree}}\right) = 2I\cos\zeta_2 \textrm{d}\beta \,.
\end{equation}
Změny elipticity se odečetly a jediný zbývající vliv je v dodatečném faktoru $\cos\zeta_2$, který pro malé $\delta$ (a tedy malé $\zeta_2$) zanedbáváme.
Pokud není explicitně uvedeno jinak (v oddílech \ref{chap:elipticita} a \ref{chap:ferh-field-cooling}), provádíme tento krok vždy.
Ilustrace dat, která ukazují na nutnost provádět popsanou kompenzaci, je na obr. \ref{fig:mustek-desticka-ilustrace}.
Potvrzením správnosti kroků jsou ovšem až výsledky dosažené v kap. \ref{chap:5}.

Poznamenejme, že uvedená procedura funguje pouze za předpokladu, že do destičky dopadá přibližně lineárně polarizované světlo, protože jenom pak jsou obě vyvažující polohy destičky posunuté přesně o \SI{90}{\degree}.
Při dopadu eliptického světla je situace výrazně složitější a dále se jí nezabýváme.

\begin{figure}[htbp]
    \centering
    \missingfigure{destikcy data}
    \caption{Ilustrace potřeby provádět kompenzaci nepřesného fázového zpoždění vyvažovací destičky. (a) Změřená data, která se liší pro dvě polohy destičky, které by pro ideální destičky byly ekvivalentní. (b) Schéma experimentu: CoFe, pokojová teplota, transmisní geometrie, žádná zrcadla mezi vzorkem a můstkem.}
    \label{fig:mustek-desticka-ilustrace}
\end{figure}

\subsubsection*{Zrcadla}

Ve schématu experimentu na obr. \ref{fig:zakladni-schema} se v transmisní i reflexní geometrii vyskytují v dráze svazku mezi vzorkem a můstkem zrcadla.
Důvody jsou čistě logistické, v reflexi je nutné odražený svazek prostorově oddělit od dopadajícího, v transmisi je nutné svazek vyvést z kryogenní komory.
V obou případech svazek dopadá pod úhlem \SI{45}{\degree}.

Experimentálně bylo zjištěno, že odraz od použitých zrcadel není izotropní.
Odražené světlo dosahuje ve studovaném rozsahu vlnových délek elipticity až \SI{10}{\degree}\todo{kolik}.
Zrcadlo modelujeme jako retardér s Jonesovou maticí
\begin{equation}
    \mathcal{T}_m = \begin{pmatrix} 1&0\\0&e^{i\delta} \end{pmatrix} \,.
\end{equation}

Vzhledem k nutnosti kompenzace nedokonalosti destičky, které vyžaduje, aby do ní vstupovala přibližně lineární polarizace,se všechno úsilí k řešení tohoto problému soustředilo na kompenzaci zrcadla -- vložení dalšího optického prvku, který působí inverzně.
Nejprve však ztratíme pár slov o tom, jaký by mělo zrcadlo vliv na ideální můstek.
Mohli bychom zavádět kovektor $\Dtks'''$ vzhledem ke světlu před zrcadlem, který by měl stejný tvar jako \eqref{eqn:mustek-s-elipticitou}, ale s rozdílnou závislostí $\zeta_1$ a $\zeta_2$ na $\theta_{\lambda/2}$.
Pro kvalitativní pochopení je v tomto případě ale názornější zůstat u $\Dtks''$ a namísto kovektoru zrcadlem zobrazit vstupní Stokesovy vektory.

Pro ideální můstek je nulová kružnice $\Dtks''$ tvořena poledníkem, který je během vyvažování otáčen kolem osy $s_3$ tak, aby procházel bodem odpovídajícím vstupní polarizaci.
Zrcadlo otočí rovník lineárních polarizací o úhel $\delta$ podél osy $s_1$.
Zároveň se Stokesovým vektorem však musíme zobrazit i jeho diferenciál (změna $\beta$ před zrcadlem se projeví i změnou $\chi$ za zrcadlem).
Vzájemnou orientací nulové kružnice a diferenciálu Stokesova vektoru lze graficky určit, jakými faktory se do měřeného signálu promítne $\textrm{d}\beta$ a $\textrm{d}\chi$.
Viz obr. \ref{fig:mustek-zrcadlo-data} (a).
Pro význačné $\beta$ lze odečíst přímo z obrázku
\begin{align}
    \textrm{d}\Udif(\beta=\SI{0}{\degree}) &= 2I\left(\cos(\delta)\textrm{d}\beta + \sin(\delta)\textrm{d}\chi\right) \,,\\
    \textrm{d}\Udif(\beta=\SI{90}{\degree}) &= 2I\left(\cos(\delta)\textrm{d}\beta - \sin(\delta)\textrm{d}\chi\right) \,,\\
    \textrm{d}\Udif(\beta=\SI{45}{\degree}) &= 2I\textrm{d}\beta \,, \label{eqn:must-zrc-1}\\
    \textrm{d}\Udif(\beta=\SI{135}{\degree}) &= 2I\textrm{d}\beta \,. \label{eqn:must-zrc-2}
\end{align}

Zde je jasně vidět původ prvního popsaného problému (, kterým trpí např. data na obr. \todo{odkaz na obr.}), totiž že změřená data pro $\beta$ a $\beta+\SI{90}{\degree}$ jsou zcela odlišná.
Problém se zrcadly dlouho unikal pozornosti, především z toho důvodu, že jsou to právě polarizace $\beta=\SI{0}{\degree}, \SI{90}{\degree}$, které jsou nejvíce ovlivněny, zatímco $\beta=\SI{45}{\degree}, \SI{135}{\degree}$ jsou minimálně.
S- a p-polarizace jsou přece odraženy beze změny\ldots

\begin{figure}[htbp]
    \centering
    \missingfigure{mustek zrcadlo}
    \caption{(a) }
    \label{fig:mustek-zrcadlo-data}
\end{figure}

Nejjednodušší způsob, jakým zanesenou elipticitu zrcadla kompenzovat, je přidat ještě jedno identické zrcadlo, ve kterém je role s- a p- polarizace prohozena (odraz nahoru a do strany jako na obr. \ref{fig:mustek-zrcadlo-data} (b).
Obě polarizace pak po průchodu oběma zrcadly nabydou stejného fázového faktoru.
Dvojici zrcadel pak nazýváme \emph{zkřížená zrcadla} (v diagramech je značíme jako na obr. \ref{fig:mustek-zrcadlo-data} (b)) a chovají se jako neutrální prvek.
Trik zkřížených zrcadel se ustálil a je použit ve všech experimentech v kap. \ref{chap:5}.

Nicméně otestována byla ještě jedna metoda kompenzace, kterou zde popíšeme -- pomocí Berekova kompenzátoru.
Berekův kompenzátor je laditelný retardér a je proto možné jeho správným nastavením zrcadlo kompenzovat.
Jedinou komplikací je právě správné nastavení.
To jsme provedli dvěma způsoby.

První způsob, který označujeme jako \emph{statický}, se snaží nastavit fixní polohu, která zrcadlo zcela kompenzuje pro všechna $\beta$.
Procedura je zdlouhavá a probíhá iteračně: nejdříve je správně zorientovaná optická osa Berekova kompenzátoru, a poté nastavené správné fázové zpoždění, zatímco je pomocí rotačního polarizátoru za kompenzátorem měřena elipticita pro vybrané hodnoty vstupního $\beta$ (především \SI{45}{\degree} a $\SI{135}{\degree}$.
Správného nastavení je dosaženo, když elipticita za kompenzátorem nulová pro všechna $\beta$ -- každá lineární polarizace před zrcadlem je lineární i za kompenzátorem.

Druhý způsob označujeme jako \emph{dynamický}, protože nemá ambice kompenzovat zrcadlo a je nastaven pro každé $\beta$ zvlášť.
Princip je založen na tom, že není striktně vzato nutné zrcadlo kompenzovat. Vždyť \eqref{eqn:must-zrc-1}, \eqref{eqn:must-zrc-2} jsou v pořádku i se zrcadlem -- můstek dokáže měřit čisté stočení $\textrm{d}\beta$, i když do něj vstupuje eliptické světlo.
Toto rozvolnění ubírá jeden stupeň volnosti Berekova kompenzátoru, který je nutno přesně nastavit, a tím výrazně usnadňuje proceduru.
Ve dvoudimenzionálním stavovém prostoru Berekova kompenzátoru (poloha osy a fázové zpoždění) existuje pouze jeden bod, který ho správně nastaví staticky, ale celá křivka, která ho nastaví dynamicky.

Cílem je nastavit pro každé $\beta$ Berekův kompenzátor tak, aby se vynuloval koeficient $\textrm{d}\chi$ v $\Udif$ a platilo prosté $\textrm{d}\Udif=2I\textrm{d}\beta$.
Za účelem zpětné vazby byl využit PEM umístěný před vzorek, viz obr. \ref{fig:dynamicky-berek} (a).
Amplituda fázového zpoždění PEM $\delta_\textrm{PEM}$ byla nastavována v rozsahu hodnot přibližně 0--\SI{10}{\degree}.
Pro malé zpoždění dochází totiž pouze k periodickému kmitání $\chi$ s frekvencí $\omega_\textrm{PEM}$, zatímco $\beta$ kmitá s amplitudou až v druhém řádu $\propto\delta_\textrm{PEM}^2$ a s frekvencí $2\omega_\textrm{PEM}$.
Rozdílové napětí na frekvenci $\omega_\textrm{PEM}$ by tedy mělo být úměrné koeficientu $\textrm{d}\chi$.

Nastavení tedy spočívalo v současném/střídavém točení vyvažovací destičky, osy a zpoždění kompenzátoru, zatímco na dvou lock-inech byly sledovány hodnoty $\Udif(\omega=0)$ a $\Udif(\omega=\omega_\textrm{PEM})$ s cílem obě současně vynulovat.
Tímto způsobem jsme se mimoděk vyhnuli problému s nedokonalými destičkami, protože nulovaný koeficient $\textrm{d}\chi$ není specifický pro zrcadla, ale zahrnuje v sobě celou detekční aparaturu.
S dynamickým Berekovým kompenzátorem není třeba měřit obě polohy destičky.


\begin{figure}[htbp]
    \centering
    \missingfigure{}
    \caption{(a) schéma. (b) PEM sfera}
    \label{fig:dynamicky-berek}
\end{figure}

Oběma způsoby je možné Berekův kompenzátor nastavit do dvou neekvivalentních poloh lišících se v tom, jestli se dvojice zrcadlo--kompenzátor dohromady chová jako neutrální prvek (``nedotáčivý mód'', $\delta_m+\delta_B=0$) či půlvlnná destička (``přetáčivý mód'', $\delta_m+\delta_B=\pi$).

Oba způsoby s oběma módy byly vyzkoušeny na vzorku CoFe v transmisní geometrii při pokojové teplotě (bez štítu kryostatu), viz obr. \ref{fig:g:mustek-kompenzace-berek-vysledky}.
Při měření byly mezi vzorkem a můstkem umístěny dvě zrcadla jako na obr. \ref{fig:zakladni-schema}.
Zrcadla byla umístěna paralelně, takže očekáváme, že se chovají jako retardér s dvojnásobným fázovým zpožděním.
Z neznámého důvodu byla data měřená v přetáčivém módu chybná (při porovnání s měřením bez zrcadel), přestože k tomu teoreticky není důvod.
Dále jsme se kompenzací zrcadel Berekovým kompenzátorem nevěnovali.

\begin{figure}[htbp]
    \centering
    \missingfigure{}
    \caption{}
    \label{fig:g:mustek-kompenzace-berek-vysledky}
\end{figure}

\subsection{Současné měření elipticity}
\label{chap:elipticita}

Měření dvou poloh vyvažovací destičky (pro kompenzaci její nedokonalosti) je časově náročné a zahazuje polovinu dat -- rozdíl signálů pro obě polohy, který v sobě nese informaci o elipticitě.
Učinili jsme proto pokus 
