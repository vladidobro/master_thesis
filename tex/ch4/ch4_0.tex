Předkládaná práce se zabývá jedním z magneto-optických experimentů probíhajících v Laboratoři OptoSpintroniky (LOS), společném pracovišti MFF UK a FZU AV ČR.

Nejvýstižněji lze vyvíjenou metodu popsat jako \emph{spektroskopii anizotropních kvadratických MO jevů} (MLD/Voigtova/Cotton-Moutonova jevu) a přidruženou \emph{in-plane magneto-metrie} tenkých feromagnetických filmů v téměř kolmém dopadu a rotujícím poli.
Schéma aparatury je na obr. \ref{fig:zakladni-schema} (a).
Používáme několik vektorových veličin v rovině $xy$ ($\beta$ -- vstupní lineární polarizace, $\phih$ -- vnější magnetické pole, $\phim$ -- magnetizace vzorku, $\gamma$ -- in-plane rotace vzorku, aj.), úhel odečítáme vždy od poloosy $+x$ v kladném směru k poloose $+y$, viz obr. \ref{fig:zakladni-schema} (b).

Následuje popis jednotlivých komponent tohoto experimentálního uspořádání.

\begin{figure}[htbp]
    \centering
    \includegraphics{./img/svg/schema-hlavni.drawio.pdf}
    \caption{(a) Schéma základní formy experimentu a definice experimentální souřadné soustavy.
    Laserové světlo vychází z filtračního modulu Varia a prochází přes řadu sklopných zrcadel, jejichž účelem je umožnit měření i v jiných geometriích.
Následuje přerušovač svazku (chopper, CH), interferenční filtr (IF) (pouze v části spektrálního rozsahu), nastavitelný šedý filtr (NDF) a hlavní fixní polarizátor P0 definující lineární polarizaci $\beta=\SI{0}{\degree}$.
Do tohoto místa je aparatura vždy identická a předchozí část tedy nekreslíme ve schématech jednotlivých experimentů (čočky také dále nebudeme kreslit).
Následuje rotační půlvlnná destička WP1 a polarizátor P1 nastavující požadované $\beta$.
Svazek je fokusován na vzorek v komoře kryostatu (se skleněnými okénky) mezi pólovými nástavci vektorového elektromagnetu.
Odražený/procházející (znázorněny oba) svazek je odkloněn zrcadly, kolimován a detekován schématem optického můstku.
Signál z obou detektorů je zpracován rozdílovým a součtovým předzesilovačem a zpracován lock-in zesilovači na frekvenci chopperu.
(b) Definice úhlů v rovině $xy$. $\alpha$ zde zastupuje libovolný z $\beta$, $\phih$, $\phim$, $\gamma$, aj.}
    \label{fig:zakladni-schema}
\end{figure}

\paragraph{Vektorový magnet}
V laboratoři sestavený elektro-magnet tvořený dvěma páry nezávislých cívek dokáže vytvořit libovolně orientované vnější magnetické pole $\vHext$ v rovině $xy$ o maximální velikosti \SI{210}{\milli\tesla}.
V praxi jsou kvůli hysterezi magnetu použity pouze definované charakterizované procedury (posloupnosti proudů).
Vývojem příslušných proudových tabulek se zabývají práce \cite{kimakCharakterizaciaDvojdimenzionalnehoElektromagnetu2017,kimakOptickaSpektroskopieAntiferomagnetu2019}.
V této práci používáme pouze dvě možnosti: rotaci pole o velikosti $\Hext=\SI{207}{\milli\tesla}$ a \SI{50}{\milli\tesla} s krokem v úhlu pole $\phih$ minimálně \SI{1}{\degree}.

\paragraph{Kryostat}
Kryostat (výrobce AR) s uzavřeným cyklem a topením dovoluje udržovat vzorek v rozmezí teplot cca 15--\SI{800}{\kelvin}.
Vzorek je nalepen na tzv. \emph{cold-finger} a v kryogenické komoře, která je vyčerpána turbomolekulární vývěvou, je umístěn mezi pólové nástavce magnetu.
Komora je opatřena zpředu a ze stran skleněnými okénky.
Magneto-optická aktivita okének (Faradayův jev) byla zkoumána v práci \cite{baduraMagnetooptickaMereniPro2019}.

\paragraph{Super-kontinuální laser}
Používaný laser \emph{SuperK EXTREME} (NKT Photonics) generuje široko-spektrální pulzy, které jsou dále filtrovány pro získání pulzů s šířkou pásma \SI{10}{\nano\meter}.
V rozmezí \num{460}--\SI{845}{\nano\meter} k filtraci používáme laditelný filtr \emph{SuperK VARIA}, v rozmezí \num{845}--\SI{1600}{\nano\meter} pak sadu pásmových interferenčních filtrů.
V úvodních měřeních jsme jednou využili také externí laser MatchBox (Integrated Optics) s vlnovou délkou \SI{405}{\nano\meter}.

\paragraph{Polarizační optika}
Pevně umístěné polarizatóry P0 a P2 jsou širokospektrální polarizátory typu Glan Laser (Thorlabs).
Polarizátor v motorizovaném rotačním držáku P1 je absorpční (\todo{doplnit výrobce}), daný rozsah vlnových délek je pokryt polarizátory s označením P-VIS (\num{400}--\SI{700}{\nano\meter})\todo{doplnit rozsahy} a P-IR (\num{600}--\SI{2000}{\nano\meter}).
Širokospektrální půlvlnné destičky WP1, WP2 (Newport) jsou také umístěny v motorizovaných otočných držácích (Thorlabs a Newport) a vyskytují se ve třech sadách: VIS1, VIS2 (\num{400}--\SI{700}{\nano\meter}); NIR1, NIR2 (\num{700}--\SI{1000}{\nano\meter}); IR1, IR2 (\num{1000}--\SI{1600}{\nano\meter}).
Posloupnost prvků P0, WP1, P1 má za úkol vytvořit vysoce kvalitní lineární polarizaci s definovaným natočením $\beta$, která dále dopadá na vzorek.
Destička WP2 slouží k vyvažování můstku.

\paragraph{PEM}
Foto-elastický modulátor (není na schématu) je zařízení, ve kterém vlivem zvukové vlny dochází k modulaci síly lineárního dvojlomu,
takže se chová jako retardér (fázová destička) s periodicky modulovaným fázovým zpožděním (viz \eqref{eqn:cisty-retarder}) $\delta_\textrm{PEM}(t) \propto \cos(\omega_\textrm{PEM}t)$. 
Charakterizaci a detailnímu popisu se věnuje práce \cite{minarModulacePolarizaceSvetelne2004}.

\paragraph{Berekův kompenzátor}
Berekův kompenzátor (není na schématu) je laditelný retardér tvořený dvojlomným krystalem, jehož naklápěním lze dosáhnout libovolného (v daném rozsahu) fázového zpoždění $\delta$ mezi kolmými lineárními polarizacemi.
Charakterizaci se věnuje práce \cite{schusserSkryteKouzloPolarizace2014}.

\paragraph{Detektory}
V rozsahu \num{460}--\SI{1100}{\nano\meter} používáme křemíkové diody.
V infračervené oblasti \num{1000}--\SI{1600}{\nano\meter} používáme detektory na bázi \ch{InGaAs}, které byly studovány v práci \cite{hovorakovaCharakterizaceInfracervenehoDetektoru}.

\paragraph{Optický můstek}
K měření je využito schéma optického můstku popsané v oddílu \ref{chap:mustek-kap2}.

\paragraph{Přerušovač svazku}
Ve dráze svazku je vždy umístěn intenzitní modulátor (\emph{chopper}), který dovoluje přesnější měření pomocí fázově citlivých zesilovačů (\emph{lock-inů}).
