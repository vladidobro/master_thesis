\section{Určení anizotropie MLD}
\label{chap:anizotropie-MLD}

Snaha o vystižení anizotropie MLD pomocí vzorce \eqref{eqn:PMLD-naivni} založeném na \eqref{eqn:PMLD} nebyla úspěšná.
Výskyt $\phim$ v argumentu sinu se zakládá na argumentech symetrie; pro plně rotačně symetrické prostředí v rovině rozhraní zanáší magnetizace jediný význačný směr a osa zanesené optické anizotropie s ní pak musí splývat.
Tento argument však selhává i pro čtyřčetnou rotační symetrii, pak totiž není důvod aby např. $\phim=\SI{10}{\degree}$ zanášelo optickou anizotropii ve stejném směru.

Správné rozšíření vzorečku by mělo tvar
\begin{equation}
\label{eqn:PMLD-ansatz}
    \Delta\beta = P(\phim) \sin \left( 2\varphi_O(\phim) - 2\beta   \right)
\end{equation}
se směrem optické osy $\varphi_O$, které se už pro neúplnou rotační symetrii nemusí rovnat $\phim$.
V takovém tvaru postihuje MO jevy všech řádů (stejně jako původní izotropní vzorec) a trpí stejnou degenerací jako \eqref{eqn:PMLD-naivni}, takže lze $P$ a $\varphi_O$ určit pouze s nějakým přepokladem (např. kubické symetrie), který degeneraci sejme.

Se vzorečkem je ještě jeden problém: $\phim$ v něm vystupuje pouze skrze libovolné funkce $P$ a $\varphi_O$, a bez přidaných požadavků na jejich tvar nelze určit magnetickou anizotropii $\phim(\phih)$.
Shodná magnetická anizotropie je ale ve výsledku jediným spolehlivým kritériem, které nás dokáže přesvědčit o správnosti změřených dat.
Přídavný požadavek na $P$ a $\varphi_O$, který zde přijmeme, je takový, že měřené stočení je MO jevem maximálně kvadratickým v magnetizaci vzorku $\vec{M}$.
Uvidíme, že tento požadavek je k určení magnetické anizotropie dostačující (viz oddíl \ref{chap:urceni-magneticke-anizotropie}), tj. zajišťuje vzájemnou jednoznačnost $\Delta\beta$ a $\phim$.

Vzhledem k libovolnosti funkcí $P$ a $\varphi_O$ je jediná informace obsažená v ansatzu \eqref{eqn:PMLD-ansatz} vzájemná závislost signálu pro různá $\beta$.
Ta je založena na dvou předpokladech.
Zaprvé musí být průchod/odraz přibližně izotropní, tj. nultý člen rozvoje transmisní/reflexní Jonesovy matice (zde obě značíme $\mathcal{R}$) musí být přibližně úměrný jednotkové matici.
V praxi to znamená téměř kolmý dopad a přibližně izotropní tenzor permitivity vzorku.
Označíme odchylku od jednotkové matice $\mathcal{R}'$ a rozdělíme jí na ``nemagnetickou'' a ``magnetickou'' část (po vytknutní celkového faktoru $R_0$).
\begin{align}
\label{eqn:PMLD-Jones}
\mathcal{R}(\vec{M}) &= R_0 \left[ \begin{pmatrix} 1&0\\0&1 \end{pmatrix} + \mathcal{R}'(\vec{M}) \right] \\
                     &= R_0 \left[ \begin{pmatrix} 1&0\\0&1 \end{pmatrix} + \mathcal{R}'_0 + \mathcal{R}'_M(\vec{M}) \right]
\end{align}
Pro jednoduchost zápisu dále vypustíme celkový faktor $R_0$ (pokládáme $R_0=1$), který se ve výpočtu nijak neprojeví.
Pokud je $\mathcal{R}'$ malé, je rozumné uvažovat, že můstek je vyvážen pro stejnou lineární polarizaci, jako je ta dopadající.
Díky tomu je pak možné spočítat, na které prvky reflexní matice je měřený signál citlivý.
Pokud toto splněno není, pak je můstek pro každé $\beta$ citlivý na jiný mix prvků $\mathcal{R}'_M$, a tento mix je složitou funkcí nemagnetické $\mathcal{R}'_0$; vzorec pak nemá jednoduchou $\beta$-závislost \eqref{eqn:PMLD-ansatz}.
Po případném elipsometrickém změření $\mathcal{R}'_0$ je možné správnou $\beta$-závislost dopočítat a dosadit do nového ansatzu, my se tím však v této práci nezabýváme, protože uvedený předpoklad splňujeme.

Pomocí Stokesových kovektorů lze jednoduše vyjádřit stočení a elipticitu do prvního řádu v $\mathcal{R}'$.
Pro dané vstupní $\beta$ (Stokesův vektor $\Stks(\beta)$) zavedeme kovektory\footnote{Argument $\beta$ zde značí, pro jaké $\beta$ je daný kovektor platný. Horní index $\beta$ či $\chi$ značí, jestli kovektor měří stočení či elipticitu.} $\Dtks^\beta(\beta)=(0, \cos2\beta, \sin2\beta, 0)$, resp. $\Dtks^\chi(\beta)=(0, 0, 0, 1)$, které měří stočení, resp. elipticitu vyjádřenou změnou Stokesova vektoru $\textrm{d}\Stks$:
\begin{align}
    \Dtks^\beta(\beta) \cdot \Stks(\beta) = 0 \,, \quad &\textrm{d}\beta = \Dtks^\beta \cdot \textrm{d}\Stks \\
    \Dtks^\chi(\beta) \cdot \Stks(\beta) = 0 \,, \quad &\textrm{d}\chi = \Dtks^\chi \cdot \textrm{d}\Stks \,.
\end{align}
Rozvedeme Muellerovu matici \eqref{eqn:Mueller-rozklad} vzorku do prvního řádu\footnote{Striktně vzato po tomto kroku již nejde o mocninný rozvoj měřeného signálu v $\vec{M}$, protože zahazujeme některé kvadratické členy a ponecháváme jiné. Pro malá stočení jsou však zahozené členy silně potlačeny.} v $\mathcal{R}'$ (vyjádřeného diferenciálem $\textrm{d}\mathcal{R}$ v okolí jednotkové matice\footnote{Píšeme $\mathcal{R}= \begin{pmatrix} 1&0\\0&1 \end{pmatrix} + \textrm{d}\mathcal{R}$.}) 
\begin{align}
\label{eqn:dif-Mueller}
    \textrm{d}M_{00} = 0 \\
    \textrm{d}M_{i0} = \textrm{d}M_{0i} =   \\
    \textrm{d}M_{ij} = \delta_{ij} \Re\operatorname{Tr}\lbrace\textrm{d}\mathcal{R}\rbrace + \epsilon_{ijk} \Im\operatorname{Tr}\lbrace\sigma_k\textrm{d}\mathcal{R}\rbrace
\end{align}
pro $i,\,j = 1,\,2,\,3$.

Stočení a elipticita jsou
\begin{equation}
    \textrm{d}\beta = \begin{pmatrix} -\sin2\beta&\cos2\beta \end{pmatrix} \begin{pmatrix} \textrm{d}M_{10}&\textrm{d}M_{11}&\textrm{d}M_{12}\\\textrm{d}M_{20}&\textrm{d}M_{21}&\textrm{d}M_{22} \end{pmatrix} \begin{pmatrix} 1\\\cos2\beta\\\sin2\beta \end{pmatrix}
\end{equation}
a
\begin{equation}
    \textrm{d}\chi = \begin{pmatrix} \textrm{d}M_{30}&\textrm{d}M_{31}&\textrm{d}M_{32} \end{pmatrix} \begin{pmatrix} 1\\\cos2\beta\\\sin2\beta \end{pmatrix} \,,
\end{equation}
po dosazení \eqref{eqn:dif-Mueller} pak 
\begin{align}
    \textrm{d}\beta &= -\sin2\beta \ldots + \cos2\beta \ldots + \Im \operatorname{Tr}\lbrace \sigma_3 \textrm{d}\mathcal{R} \rbrace\,,\label{eqn:dbeta-stokes}\\
    \textrm{d}\chi &= \ldots
\end{align}
či kompaktněji
\begin{equation}
    \textrm{d}\beta + i \textrm{d}\chi = 
\end{equation}

Vidíme, že v okolí izotropního odrazu/průchodu jsou jediné povolené $\beta$-závislosti stočení a elipticity dané třemi členy: $\cos2\beta$, $\sin2\beta$ a konstantou\footnote{Stočení konstantní pro všechna $\beta$. Např. Faradayův jev je způsoben tímto členem.}.
Konstantní členy $\propto \operatorname{Tr}\lbrace \sigma_3 \textrm{d}\mathcal{R} \rbrace$ jsou navíc často vyloučeny symetrií, konkrétně ve všech případech studovaných v této práci: při kolmém dopadu na [001]-normálově orientovaný kubický vzorek s in-plane magnetizací.
Pak jsou povolené jen členy harmonické s frekvencí $2\beta$, které se složí do vzorečku \eqref{eqn:PMLD-ansatz}.

Poznamenejme, že nemagnetická část reflexní matice $\mathcal{R}'_0$ zanáší do měřeného signálu pouze konstantu, kterou můstek měřit nedokáže.

Dále se budeme věnovat situaci relevantní pro tuto práci, tzn. měřeno je pouze stočení a na základě předpokládané kubické symetrie, kolmého dopadu a in-plane magnetizace neuvažujeme konstantní člen v \eqref{eqn:dbeta-stokes}.

Předpokládáme, že vzorek je v jedno-doménovém stavu se saturovanou in-plane magnetizací (Stonerův-Wohlfarthův model z oddílu \ref{chap:magneticka-anizotropie}).
Díky tomu se každý člen řádu $k$ mocninného rozvoje v $\vec{M}$ redukuje na harmonickou funkci $k\phim$.
Do druhého řádu se tedy vyskytují členy $\cos\phim$, $\sin\phim$, $\cos2\phim$, $\sin2\phim$ a konstanta, kterou nedokážeme změřit kvůli vyvažování můstku ($\xi(\beta)$).
První dva členy jsou lineární MO jevy a většinou kvůli téměř kolmému dopadu ($< \SI{1}{\degree}$) poměrně malé.

Zajímáme se pouze o kvadratické jevy, proto je od lineárních odseparujeme stejným způsobem jako metoda rotujícího pole (rovnice \eqref{eqn:rotmoke-separace}).
Předpokládáme in-plane magnetickou anizotropii s dvoučetnou rotační symetrií, pak platí $\phim(\phih+\SI{180}{\degree})=\phim(\phih)+\SI{180}{\degree}$ a symetrizaci podle $\vec{M}$ můžeme provést pomocí symetrizace podle $\vHext$:
\begin{equation}
    \label{eqn:symetrizace-H}
    \Delta\beta^\textrm{Q}(\phih) = \frac{1}{2}\left(\Delta\beta(\phih) + \Delta\beta(\phih+\SI{180}{\degree})\right)
\end{equation}

V dalším předpokládáme, že symetrizace byla provedena a $\Delta\beta^\textrm{Q}$ značíme bez indexu jako $\Delta\beta$.
Poznamenáme však, že tato symetrizace teoreticky není nezbytná.
V konečném důsledku jsou fitovány koeficienty mocninného rozvoje a principiálně není problém fitovat o několik parametrů navíc.
V této práci jsme se o to nepokoušeli.

Pro každé změřené $\beta$ máme tedy dva neznámé koeficienty funkcí $\cos2\phim$ a $\sin2\phim$.
Za splnění předpokladů vedoucích na \eqref{eqn:dbeta-stokes} je pak $\beta$-závislost této dvojice koeficientů také popsána dvojicí koeficientů $\cos2\beta$ a $\sin2\beta$, celkem máme pro celou sadu měření 4 koeficienty.
Používaný model je tedy
\begin{alignat}{2}
    \label{eqn:PMLD-matice}
    \textrm{d}\beta &= && P_{11} \cos2\phim\cos2\beta + P_{12}\cos2\phim\sin2\beta \\
                    & && + P_{21}\sin2\phim\cos2\beta + P_{22}\sin2\phim\sin2\beta \\
                    &\equiv & & P_\textrm{ISO} \sin(2\phim - 2\beta - 2\alpha_\textrm{ISO}) \\
                    &  & & + P_\textrm{ANISO} \sin(2\phim + 2\beta - 2\alpha_\textrm{ANISO})
\end{alignat}
s vyjádřením parametrů dvěma ekvivalentními způsoby.
Pro účely fitu používáme vyjádření první rovností, protože je v parametrech lineární.
Pro prezentování výsledků používáme druhý způsob.

Pro [100]-normálově orientovaný kubický vzorek má reflexní i transmisní matice tvar (s položeným $R_0=1$)
\begin{equation}
    \mathcal{R} = \begin{pmatrix} 1&0\\0&1 \end{pmatrix} + 
\end{equation}
kde jsme přímo dosadili fenomenologické koeficienty $P$.
Pro souvislost se složkami MO tenzorů a materiálových parametrů nemagnetických vrstev viz dodatek \ref{app:berreman}.
Zde pouze uvedeme, že $P_\textrm{ISO}$, resp. $P_\textrm{ANISO}$ jsou úměrné $(G_s-K^2/n^2)$, resp. $\Delta G$ se stejnou konstantou úměrnosti.

Z tvaru také výplývá, že $\alpha_\textrm{ISO}=0$ a $\alpha_\textrm{ANISO}=\gamma$, což vysvětluje zavedené značení.
Pro izotropní vzorek platí $\Delta G = 0$ a tedy i $P_\textrm{ANISO}=0$, a člen úměrný $P_\textrm{ISO}$ je nezávislý na úhlu in-plane rotace $\gamma$.
Přestože $\alpha_\textrm{ISO}$ by teoreticky mělo být nulové, ve zpracování ho připouštíme, ale přikládáme mu nový význam.
Souřadné soustavy polarizace (, vůči které je měřené $\beta$) a magnetu (, vůči které je měřené $\phih$, $\phim$) byly stanoveny zcela odlišnými způsoby a pravděpodobně jsou spojeny vzájemnou rotací o malý úhel ($<\SI{5}{\degree}$).
Úhlu $\alpha_\textrm{ISO}$ tedy přikládáme tento význam a kontrolujeme, jestli není příliš vysoké.
S takovým nenulovým ofsetem je anizotropní část $\alpha_\textrm{ANISO}=\alpha_\textrm{ISO}+\gamma$.

V soustavě spojené se vzorkem používáme ještě jednu parametrizaci, která vyjadřuje sílu kvadratického jevu při magnetizaci ve význačných směrech [100] a [110] a lze vizuálně přibližně odečítat z měřených dat (pro případ $\alpha_\textrm{ANISO}=0$)
\begin{equation}
    \Delta\beta = P_{[100]} \cos 2 (\phim-\gamma) \sin 2(\beta-\gamma) + P_{[110]} \sin 2 (\phim-\gamma) \cos 2(\beta-\gamma)
\end{equation}
Při orientaci polarizace ve směru [100] ($\beta=\gamma$) je amplituda změřené křivky $\Delta\beta(\phim)$ přímo $P_{[110]}$, při polarizaci ve směru [1-10] ($\beta=\gamma-\SI{45}{\degree}$) je amplituda $P_{[100]}$, viz obr. \ref{fig:urceni-MLD-ilustrace}.
Platí
\begin{equation}
    P_{[100]} = P_\textrm{ISO} + P_\textrm{ANISO} \,,\quad P_{[110]} = P_\textrm{ISO} - P_\textrm{ANISO} \,.
\end{equation}

\begin{figure}[htbp]
    \centering
    \includegraphics{./data/out/pmld-ukazka.pdf}
    \caption{Určení MLD koeficientů $P$. Měřená data s CoFe, pokojová teplota, \SI{1050}{\nano\meter}. Hodnoty $P$ lze odečítat přímo z grafu.}
    \label{fig:urceni-MLD-ilustrace}
\end{figure}

Pro určení síly kvadratického MO jevu (popsanou fenomenologickými parametry $P$) je nutné znát měřený signál v závislosti na $\phim$.
V praxi vždy během analýzy určujeme i magnetickou anizotropii, která nám převod $\phih \mapsto \phim$ poskytne.
Pokud není magnetická anizotropie srovnatelná s vnějším polem, pak je možné poměrně přesně určit $P$ i z dat s předpokladem $\phim=\phih$.

