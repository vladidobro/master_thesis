\section{Zpracování dat}
\label{chap:zpracovani-dat}

Zde shrneme přesný postup zpracování dat v takové podobě, v jaké byl použit pro získání výsledků kap. \ref{chap:5} (kromě měření teplotního přechodu).
V této formě lze metodu aplikovat pro téměř kolmý průchod/odraz od kubického [001]-normálově orientovaného vzorku s téměř izotropní nemagnetickou permitivitou, se saturovanou in-plane magnetizací, jejíž chování je dobře modelováno Stonerovým-Wohlfarthovým modelem s uniaxiální a kubickou in-plane anizotropií.
Zrcadlo mezi vzorkem a můstkem je již kompenzováno druhým identickým zrcadlem.

Při zpracování dat provádíme určení síly MO jevů i magnetické anizotropie současně.
Pro vybrané polarizace změříme průběh $\Udif$ a $\Usum$ při otáčení vnějšího pole v plném rozsahu $\phih \in 0$--\SI{360}{\degree}.
Vždy měříme několik navazujících cyklů, které předzpracujeme způsobem popsaným v dodatku \ref{app:zpracovani}, abychom dále měli pouze jednu křivku $U_\textrm{A-B/A+B}(\phih)$ pro každé $\beta$.

Dalším krokem je převedení průběhu napětí na průběh stočení polarizace pomocí vzorce $\Delta\beta=\Udif/2\Usum$.
Poté provedeme kompenzaci půlvlnné destičky můstku podle oddílu \ref{chap:mustek-kap2}: každé $\beta$ je měřeno s oběma polohami vyvažovací destičky, výsledné $\Delta\beta$ je aritmetickým průměrem obou křivek.

Doposud tedy máme matici hodnot $\Delta\beta(\phih, \beta)$.
V tento moment je nutné odstranit všechny případné $\phih$-závislé signály, které nepochází přímo od vzorku.
Na tento problém jsme narazili při měření CoFe za nízké teploty, kdy světlo prochází okénkem komory kryostatu s polární magnetizací, takže v něm dochází k Faradayově jevu (s konkrétními okénky byl studován v \cite{baduraMagnetooptickaMereniPro2019}).
Neexistuje univerzální řešení, v uvedeném případě jsme signál od okénka a vzorku oddělili na základě předpokladu odlišné $\phih$ a $\beta$-závislosti (Faradayův jev je lichý v $\vec{M}$ a konstantní v $\beta$), viz příslušný oddíl \ref{chap:vysledky-cofe-lowt}.

Ze signálu jsou odděleny lineární jevy pomocí symetrizace v $\vHext$, viz \eqref{eqn:symetrizace-H}.

V následujícím kroku máme na výběr: můžeme a nemusíme uplatnit předpokládanou $\beta$-závislost (vzorec \eqref{eqn:PMLD-ansatz}) a sadu křivek redukovat na dvě -- $\phih$-závislé koeficienty $\cos2\beta$ a $\sin2\beta$.
Jinými slovy, provést na křivkách Fourierovu transformaci v $\beta$ a ponechat pouze frekvence $2\beta$.
Výsledek metody je na tomto kroku teoreticky nezávislý.
Odstraněná data jsou nepopsatelná fitovaným modelem a v jistém smyslu ``ortogonální'', takže pouze mění tvar cílové funkce $\mathcal{L}$, ale ne polohu jejích minim.
Na druhou stranu je ale tato úprava $\mathcal{L}$ nežádoucí: data si ``přizpůsobujeme'' modelu a uměle tím snižujeme nejistotu.
V některých případech (jako odstranění Faradayova jevu v okénkách) je ale krok nezbytný, a proto ho v zájmu konzistence provádíme vždy.
V budoucnu bude tento krok pravděpodobně ještě předmětem diskuze.

Následuje srdce celé metody -- současný nelineární fit parametrů $P$ charakterizujících sílu kvadratického MO jevu a magnetických anizotropních konstant.
Jak bylo vyloženo v předchozím oddílu, cílovou funkci považujeme za funkci pouze vhodně parametrizovaných anizotropních konstant $\mathcal{L}(\tilde{k}_4, \tilde{k}_u)$ s tichým předpokladem, že za parametry $b$ v \eqref{eqn:anizotropie-fit} jsou dosazeny ty optimální určené lineární regresí.
Oproti \eqref{eqn:anizotropie-fit} však nyní anonymní parametry $b$ mají význam MO koeficientů $P$ z \eqref{eqn:PMLD-matice}
Jednou z metod nelineární optimalizace je nalezeno globální minimum.
Celkové optimální parametry $P$ jsou pak rovny těm optimálním v minimu $\mathcal{L}$.

Na závěr jsou parametry $\tilde{k}_4$, $\tilde{k}_u$ převedeny do lidsky čitelné parametrizace pomocí $k_4$, $k_u$, $\phiu$ a natočení $\gamma_\textrm{SW}$; a parametry $P$ pomocí $P_\textrm{ISO}$, $P_\textrm{ANISO}$, $\alpha_\textrm{ISO}$, $\gamma_\textrm{MO}=\alpha_\textrm{ANISO}-\alpha_\textrm{ISO}$.

Úhel in-plane rotace $\gamma$ máme určen dvěma způsoby: jako osu magnetické anizotropie a jako osu anizotropie MLD.
Jejich porovnání nám poskytuje kontrolu, že metoda zcela neselhala.
Oba úhly ve fitu držíme oddělené především kvůli pohodlí.

Další kontrolu nám poskytuje $\alpha_\textrm{ISO}$, které má význam vzájemné rotace souřadné soustavy polarizace a magnetického pole, a proto zřejmě nemůže přesáhnout přesnost s jakou byly stanoveny, cca \SI{5}{\degree}.

Takto jsou určeny všechny parametry modelu, zbývá ještě určit jejich nejistoty.
Vzhledem ke komplexitě problému a množství dat volíme jako způsob určení nejistoty ten jediný příčetný -- založený na principech Bayesovské analýzy dat\cite{gelmanBayesianDataAnalysis2014}.
V minimu $\mathcal{L}$ je spočítána matice druhých derivací podle fitovaných parametrů, její inverzí vznikne kovarianční matice, a na její diagonále jsou variance jednotlivých parametrů.
Pro určení nejistot transformovaných parametrů ($k_4$, $k_u$, atd.) nejdříve transformujeme kovarianční matici a až potom odečítáme z diagonály.
