\section{Zpracování dat}
\label{chap:zpracovani-dat}

Zde shrneme přesný postup zpracování dat v takové podobě, v jaké byl použit pro získání výsledků uvedených v kap. \ref{chap:5}.
V této formě lze metodu aplikovat pro téměř kolmý průchod/odraz od kubického [001]-normálově orientovaného vzorku s téměř izotropní nemagnetickou permitivitou, se saturovanou in-plane magnetizací, jejíž chování je dobře modelováno Stonerovým-Wohlfarthovým modelem s uniaxiální a kubickou in-plane anizotropií.
Případné zrcadlo mezi vzorkem a můstkem je již kompenzováno druhým identickým zrcadlem.

Při zpracování dat provádíme určení síly MO jevů i magnetické anizotropie současně.
Pro vybrané polarizace $\beta$ změříme průběh $\Udif$ a $\Usum$ při otáčení vnějšího pole v plném rozsahu $\phih \in 0$--\SI{360}{\degree}.
Vždy měříme několik navazujících cyklů, které předzpracujeme způsobem popsaným v dodatku \ref{app:zpracovani}, abychom dále měli pouze jednu křivku $U_\textrm{A-B/A+B}(\phih)$ pro každé $\beta$.

Dalším krokem je převedení průběhu napětí na průběh stočení polarizace pomocí vzorce $\Delta\beta=\Udif/2\Usum$.
Poté provedeme kompenzaci půlvlnné destičky můstku podle oddílu \ref{chap:mustek-kap2}: každé $\beta$ je měřeno s oběma polohami vyvažovací destičky před optickým můstkem, výsledné $\Delta\beta$ je aritmetickým průměrem obou křivek.

Doposud tedy máme matici hodnot $\Delta\beta(\phih, \beta)$.
V tento moment je nutné odstranit všechny případné $\phih$-závislé signály, které nepochází přímo od vzorku.
Na tento problém jsme narazili při měření CoFe za nízké teploty, kdy světlo prochází okénkem komory kryostatu s polárním směrem magnetického pole, takže v něm dochází k Faradayově jevu (s konkrétními okénky byl tento jev studován v \cite{baduraMagnetooptickaMereniPro2019}).
Neexistuje sice univerzální řešení, ale v uvedeném případě jsme signál od okénka a vzorku oddělili na základě předpokladu odlišné $\phih$ a $\beta$-závislosti (Faradayův jev je lichý v $\vHext$ a konstantní v $\beta$), viz příslušný oddíl \ref{chap:vysledky-cofe-lowt}.

Ze signálu jsou následně odděleny lineární MO jevy pomocí symetrizace v $\vHext$, viz \eqref{eqn:symetrizace-H}.

V následujícím kroku máme na výběr: můžeme a nemusíme uplatnit předpokládanou $\beta$-závislost (vzorec \eqref{eqn:PMLD-ansatz}) a sadu křivek redukovat na dvě -- $\phih$-závislé koeficienty $\cos2\beta$ a $\sin2\beta$.
Jinými slovy, provést na křivkách Fourierovu transformaci v $\beta$ a ponechat pouze frekvence $2\beta$.
Výsledek metody je na tomto kroku teoreticky nezávislý.
Odstraněná data jsou nepopsatelná fitovaným modelem a v jistém smyslu jsou ``ortogonální'', takže pouze mění tvar cílové funkce $\mathcal{L}$, ale ne polohu jejích minim.
Na druhou stranu je ale tato úprava $\mathcal{L}$ nežádoucí: data si ``přizpůsobujeme'' modelu a uměle tím snižujeme nejistotu.
V některých případech (jako odstranění Faradayova jevu v okénkách) je ale krok nezbytný, a proto ho v zájmu konzistence provádíme vždy.
V budoucnu bude použití tohoto kroku v námi vyvíjené metodě pravděpodobně ještě předmětem diskuze.

Následuje srdce celé metody -- současný nelineární fit parametrů $P$ charakterizujících sílu kvadratického MO jevu a magnetických anizotropních konstant.
Jak bylo vyloženo v předchozím oddílu, cílovou funkci považujeme za funkci pouze vhodně parametrizovaných anizotropních konstant $\mathcal{L}(\tilde{k}_4, \tilde{k}_u)$ s tichým předpokladem, že za parametry $b$ v \eqref{eqn:anizotropie-fit} jsou dosazeny ty optimální určené lineární regresí.
Oproti \eqref{eqn:anizotropie-fit} však nyní anonymní parametry $b$ mají význam MO koeficientů $P$ z \eqref{eqn:PMLD-matice}
Jednou z metod nelineární optimalizace je nalezeno globální minimum.
Celkové optimální parametry $P$ jsou pak rovny těm optimálním v minimu $\mathcal{L}$.

Na závěr jsou parametry $\tilde{k}_4$, $\tilde{k}_u$ převedeny do parametrizace pomocí $k_4$, $k_u$, $\phiu$ a natočení vzorku $\gamma_\textrm{SW}$; a parametry $P$ jsou vyjádřeny pomocí $P_+$, $P_-$, $\pi_+$, $\gamma_\textrm{MO}=(\pi_--\pi+)/2$.

Úhel in-plane rotace $\gamma$ máme určen dvěma způsoby: jako osu magnetické anizotropie a jako osu anizotropie MLD.
Jejich porovnání nám poskytuje kontrolu, že metoda zcela neselhala.
Oba úhly ve fitu držíme oddělené především kvůli pohodlí.

Další kontrolu nám poskytuje $\pi_+$, které má význam vzájemné rotace souřadné soustavy polarizace a magnetického pole, a proto zřejmě nemůže přesáhnout přesnost, s jakou byly stanoveny, tj. cca \SI{5}{\degree}.

Takto jsou určeny všechny parametry modelu, zbývá ještě určit jejich nejistoty.
Volíme ten nejjednodušší přístup: bez ověření předpokládáme nezávislé normálně rozdělené experimentální chyby a cílovou funkci $\mathcal{L}$ v okolí minima přibližujeme kvadratickým rozvojem ve fitovaných parametrech.
Spočítáme matici druhých derivací (Hessova matice), její inverzí vznikne kovarianční matice a na její diagonále jsou variance jednotlivých parametrů.
Vzhledem k tomu, že fit magnetických anizotropních konstant a konstant $P$ provádíme odděleně, i kovarianční matice z pohodlnosti počítáme odděleně (zanedbáváme např. vzájemné korelace $P_+$ a $k_u$).
V obrázcích označujeme koeficient $\cos2\beta$ jako $\beta=\SI{0}{\degree}$ a $\sin2\beta$ jako $\beta=\SI{45}{\degree}$.

Výsledky fitu anizotropních konstant zobrazujeme pomocí oblastí spolehlivosti s hladinou spolehlivosti \SI{95}{\percent} ($2\sigma$), viz obr. \ref{fig:ukazka-zpracovani-anizotropie}.

\begin{figure}[htbp]
    \centering
    \includegraphics{./data/out/ukazka-k.pdf}
    \caption{Ilustrace výsledků fitu anizotropních konstant.
    (a) Cílová funkce $\mathcal{L}$ v závislosti na uniaxiální anizotropní konstantě $\tilde{k}_u$. (b) Detail minima. Jsou naznačeny $1\sigma$, $2\sigma$ a $3\sigma$ oblasti. Při vyhodnocení zobrazujeme $2\sigma$ (prostřední).}
    \label{fig:ukazka-zpracovani-anizotropie}
\end{figure}
