V předchozí kapitole jsme vyložili obecný fenomenologický popis magnetooptiky, nyní se zaměříme na magnetooptické (MO) \emph{jevy}, tedy jakým způsobem se tenzor permitivity závislý na magnetickém stavu projevuje v experimentu.
Změna permitivity se přirozeně projeví změnou transmisních a reflexních koeficientů, a tím pádem i v polarizačním stavu prošlého a odraženého světla.
Typický magnetooptický experiment zkoumá polarizační stav světla po průchodu či odrazu od magnetooptického materiálu, měřenou veličinou je obvykle buď úhel natočení hlavní roviny polarizace $\beta$, elipticita $\chi$ nebo méně často intenzita $I$, v závislosti na magnetickém stavu.
Magnetooptické experimenty mají společné to, že u měřené veličiny většinou nelze prakticky určit absolutní hodnotu a je nutné měřit pouze rozdíly.
Navíc se tato absolutní hodnota mění při změně většiny parametrů experimentu (např. posunutí/otočení čehokoliv), a jako jediná možnost zbývá měřit rozdíl signálu při různých magnetických stavech. To se realizuje např. přiložením vnějšího pole, současným pozorováním regionů v různých magnetických stavech (MO mikroskopie), nebo časově rozlišeným pozorováním po aplikaci silného krátkého laserového pulzu (pump-probe metody).

Jediná podmínka na zkoumaný materiál je, aby existoval v různých magnetických stavech.
Ty jsou nejčastěji charakterizované magnetizací - MO jevy byly pozorovány v diamagnetikách, paramagnetikách, feromagnetikách; není to ale podmínkou - existují i např. v kompenzovaných antiferomagnetech\cite{SaidlOpticalNeel}, ve kterých celková magnetizace vymizí. Tenzor permitivity je v nich závislý na jiném parametru uspořádání - Néelově vektoru, který charakterizuje magnetický stav.

MO jevy lze pozorovat v transmisi i v reflexi.
Reflexní jevy se souhrně nazývají jako MO Kerrovy jevy (MOKE) a většinou jsou slabší než transmisní.
Pro Kerrovy jevy existuje ustálená notace výsledků měření.
Kvůli praktickým účelům je nenulový úhel dopadu a jako vstupní polarizace se zpravidla volí buď s- nebo p-polarizace. Změna polarizačního stavu po odrazu je popsána tzv. Kerrovou rotací a Kerrovou elipticitou vyjádřených společně komplexní Kerrovou rotací $\Psi_{s/p}$ jako v \eqref{e:komplexni rotace}.


Přirozeným dělením MO jevů je na ty liché a sudé v magnetizaci, což je vzhledem k tomu, že dosud nebyl pozorován žádný MO jev vyššího než druhého řádu, ekvivalentní dělení na lineární a kvadratické jevy.
Dělení nerozlišuje, jestli je daný jev způsobený lineárním nebo kvadratickým MO tenzorem, ale jakým způsobem na magnetizaci závisí měřená veličina -- lineární MO tenzor $K$ má na svědomí i kvadratické jevy, jak se později přesvědčíme.
Lineární jevy mohou naopak být způsobeny jedině lineárním $K$.

Zmíníme ještě jedno dělení MO jevů: na základě toho, jaký druh optické anizotropie do vzorku zavádí magnetická permitivita.
Anizotropie může mít charakter dvojlomu (rozdílný reálný index lomu -- rozdílná fázová rychlost) nebo dichroismu (rozdílný imaginární index lomu -- rozdílný absorpční koeficient).
Druhá z charakteristik anizotropie je, pro které vlastní módy se daná vlastnost liší, rozlišujeme jestli jsou to lineární\footnote{Slovo lineární má zde význam lineární polarizace; nesouvisí s dělením na lineární a kvadratické jevy výše.} nebo kruhové polarizace.
Kombinací těchto charakteristik dělíme jevy na magnetický lineární/kruhový dvojlom/dichroismus: MLB, MLD, MCB, MCD\footnote{angl. Magnetic Linear/Circular Birefringence/Dichroism.}\cite{ZvezdinKotov}.
Toto dělení se z pedantského pohledu chybným způsobem používá i v reflexi, kde se např. jako MLD označuje anizotropie reálné části reflexního koeficientu, která je však spjatá s anizotropií reálné i komplexní části indexu lomu - MLB i MLD. \cite{systematicgamnas}

Poznamenáme, že pro intuitivní a dokonce i kvantitativní pochopení vlivu průchodu světla vzorkem vykazujícím jednu z uvedených anizotropií je dostačující grafické znázornění Muellerovych matic zavedené v oddílu \ref{k:stokes}.
Dvojlom je otáčení a dichroismus doutníkovatění ve směru osy procházející příslušnými lineárními/kruhovými módy, viz obr. \ref{f:MCBLDD}.

\begin{figure}\centering
\includegraphics[width=\linewidth]{./img/m1.png}
\caption{Znázornění magnetooptických jevů v průchodu pomocí Stokesových vektorů. (a) MCB - rotace (B) Poincarého sféry podél osy procházející kruhovými (C) polarizacemi, což má za následek stočení lineární polarizace nezávislé na jejím směru. (b) MLD - protáhnutí (D) Poincarého sféry v ose procházející lineárními (L) vlastními módy, což má za následek stočení polarizace uměrné $\sin(2\beta)$, s $\beta$ počítaném od lineárního módu s nižším koeficientem útlumu.}\label{f:MCBMLD}
\end{figure}

\subsection*{Lineární jevy}

Lineární jevy jsou obecně silnější a navíc mají tu výhodu, že díky opačným znaménkám signálu pro opačné směry magnetizace lze často jednoduše oddělit relevantní signál od nemagnetooptického pozadí.
V reflexi se lineární jevy označují jako LinMOKE a rozlišují se podle toho, na jaké složky magnetizace jsou citlivé, viz obr. \ref{f:KE}
\begin{itemize}
\item Polární (PMOKE) - $\M$ kolmé na rozhraní.
\item Longitudinální (LMOKE) - $\M$ v rovině rozhraní a rovině dopadu.
\item Transverzální (TMOKE) - $\M$ v rovině rozhraní a kolmo na rovinu dopadu.
\end{itemize}
LMOKE a TMOKE jsou nulové při kolmém dopadu a při téměř kolmém dopadu jejich amplituda roste lineárně s úhlem dopadu. PKE je nenulový i při kolmém dopadu.

V transmisi se vyskytují podobné jevy, z nichž samostatný název má jako jediný polární - Faradayův jev (MCB) objevený jako první už v roce 1845, ještě před formulací Maxwellových rovnic. \cite{ZvezdinKotov}


\begin{figure}\center
\includegraphics[scale=0.28]{./img/LKE.png}
\caption{Lineární Kerrovy jevy\cite{Silber}. Zatímco polární a longitudinální MOKE se projeví v mimodiagonálních složkách reflexní Jonesovy matice, a tedy stočením s- a p- polarizace, transverzální ponechává mimodiagonální složky nulové a projeví se pouze změnou intenzitní reflektivity. Při dopadu šikmé polarizace, která není ani s- ani p- se však už i TMOKE projeví stočením.}\label{f:KE}
\end{figure}

\subsection*{Kvadratické jevy}

Zatímco lineární magnetooptika se stala užitečným nástrojem v mnoha oborech, kvadratická dlouho unikala pozornosti.
Prakticky jediná situace, ve které je díky vymizení lineárních jevů možné pozorovat čistě kvadratické jevy, je kolmý dopad s transverzální magnetizací\footnote{Nebo nulový $K$ tenzor jako v případě kompenzovaných antiferomagnetů.}, jak znázorňuje obr. \ref{f:Voigt}.
Poprvé byl kvadratický jev pozorován v transmisi -- Voigt v roce 1902 pozoroval v plynech stočení kvadraticky závislé na transverzálním magnetickém poli a v roce 1907 to samé nezávisle pozorovali v kapalinách Cotton a Mouton. \cite{ZvezdinKotov}
Z pohledu třídění magnetooptických anizotropií se jedná o MLD.
Jev se dnes nazývá Voigtův jev, Cotton-Moutonův jev, terminologie není ustálená.
V reflexi se kvadratické jevy nazývají souhrnně QMOKE, případně reflexní Voigtův, Cotton-Moutonův jev nebo reflexní MLD.

Kvadratické jevy se postupem času ukázaly jako obecně téměř všudypřítomné a nezanedbatelné.
V roce 2005 byl v magnetickém polovodiči GaMnAs pozorován v reflexi při téměř kolmém dopadu \emph{obří} MLD - kvadratické stočení polarizace, které bylo svou velikostí porovnatelné s lineárními jevy  \cite{pozor2}.
K pozorování jevu bylo využito chování hysterezních smyček v materiálech se čtyřmi snadnými osami, viz. obr. \ref{f:giant mld}

\begin{figure}
    \centering
    \includegraphics[scale=0.45]{./img/mld1.png}
    \caption{Obří MLD v GaMnAs \cite{pozor2}. (a) Snadné osy magnetizace. (b) Stočení polarizace v při aplikaci vnějšího pole ve směru $\theta$. Spektrum MO jevů (c) a absorpce (d).}
    \label{f:giant mld}
\end{figure}

Silné MLB a MLD bylo také pozorováno při nízkých teplotách v paramagnetickém terbium-galiovém granátu (TGG)\cite{pozor3} a obecně jsou často silné v Heuslerových sloučeninách\cite{Heusler}.
V roce 2020 byl objeven obří QMOKE v tenkém filmu (Eu,Gd)O dosahující amplitudy až \SI{1}{\degree}.

Jedním z hlavních důvodů, proč v současné době kvadratické jevy nabývají na popularitě, je, že narozdíl od lineárních jevů existují i v kompenzovaných antiferomagnetech, které jsou relevantní pro spintroniku\cite{Jungwirthantifero}\cite{Nemecantifero}.
Byly úspěšně použity např. pro určení Néelova vektoru v antiferomagnetickém CuMnAs\cite{SaidlOpticalNeel}, pro mikroskopii antiferomagnetických domén\cite{antifdomeny} (viz obr. \ref{f:AFdomeny}) a pro pozorování reakcí antiferomagnetů na ultrarychlé změny teploty\cite{antifteploty}.

\begin{figure}
    \centering
    \includegraphics[scale=0.35]{./img/AFdomeny.png}
    \caption{Pozorování antiferomagnetických domén v tenkém filmu NiO \cite{antifdomeny}. Je zkoumán rozdíl stočení polarizace při odrazu od domén s vzájemně kolmou orientací Néelova vektoru (a). Podle vzorce \eqref{e:PMLD} dochází k největšímu kontrastu pokud vstupní polarizace svírá s Néelovým vektorem\footnote{Stejný vzoreček platí pro antiferomagnety s úhlem Néelova vektoru $\varphi_N$ místo magnetizace $\phim$.} \SI{45}{\degree}, protože pak mají obě domény efekt stočení s opačným znaménkem (b-e).}
    \label{f:AFdomeny}
\end{figure}

V nejjednodušším případě Voigtovy geometrie, kdy má navíc magnetická závislost permitivity plnou symetrii prázdného prostoru ($K$, $G$ i všechny vyšší řády jsou izotropní), má stočení polarizace vlivem Voigtova jevu jednoduchý tvar s fenomenologickým parametrem $\PMLD$ popisujícím amplitudu jevu
\begin{equation} \label{e:PMLD}
\Delta \beta=\PMLD \sin\left(2(\phim-\beta)\right) \,.
\end{equation}

Krystaly však obecně nemají izotropní $G$ tenzor, a navíc pro měření odrazu se často používá malý, ale nenulový úhel dopadu, který vnese do signálu lineární MOKE.
Vztahy mezi měřenými stočeními a materiálovými parametry, které jsme uvedli v předchozí kapitole, jsou sice řešitelné pro každou konkrétní situaci, ale pro interpretaci experimentu je nutné mít nějaký fenomenologický vzorec typu \eqref{e:PMLD}, který je platný pro širší třídu situací.
Šikmý odraz přímo na polonekonečném vzorku s permitivitou, $K$ a $G$ tenzory s přesnou kubickou symetrií, s omezením na in-plane magnetizaci byl spočítán pomocí Yehovy metody v \cite{osmismerna}.

Jiná situace byla spočítána v \cite{VISMOKE}: odraz na struktuře tvořenou izotropním polonekonečným substrátem, \emph{ultra-tenkým}\footnote{$n\w d/c\ll 1$, kde $n$ je index lomu a $d$ tloušťka.} filmem s obecně anizotropním tenzorem permitivity a izotropní nadvrstvou.
Výsledkem je vzorec, který v daném přiblížení platí přesně.
Dosazením $\e^0$, $K$ a $G$ pro [001] orientovaný vzorek je pak reprodukován vzorec pro polonekonečný vzorek z \cite{osmismerna}, pro  $\gamma$ natočení [100] směru v rovině $xy$\footnote{Pro detaily spojené s kenvencemi viz \cite{Silber}.}
\begin{align} \label{e:Silber QMOKE vzorec}
\Psi_{s/p}=&\pm A_{s/p} \left[ 2G_{44}-\frac{K^2}{\varepsilon^0}+\frac{\Delta G}{2}(1-\cos 4\gamma)  \right] M_x M_y\\
& \pm A_{s/p} \left[ \frac{\Delta G}{4}\sin 4\gamma  \right] \left(M_x^2-M_y^2\right) \pm B_{s/p} K M_y \,,
\end{align}
kde $A_{s/p}$ a $B_{s/p}$ jsou vážící konstanty závisející na úhlu dopadu a na parametrech substrátu a nadvrstvy. $A_{s/p}$ je sudou a $B_{s/p}$ lichou funkcí úhlu dopadu, jejich konkrétní tvar je uveden v příslušných původních článcích.

Výpočet v \cite{VISMOKE} je ale obecnější, dává totiž $\Psi_{s/p}$ jako funkci složek permitivity, a tak nás opravňuje použít vzorec \eqref{e:Silber QMOKE vzorec} i např. v případě strainovaných ultratenkých filmů, které mají i nemagnetooptickou anizotropii spojenou s mechanickým napětím.
Také dovoluje výpočet vzorce obdobného \eqref{e:Silber QMOKE vzorec} pro kubické filmy s jinou orientací krystalografických os (např. [111] kolmé na rozhraní), čemuž se věnuje \cite{Silber}.

Vliv izotropního substrátu a nadvrstvy diskutuje \cite{Vispolar} metodou \emph{efektivních rozhraní}. Rozšířením vzorce \eqref{e:Silber QMOKE vzorec} pro širší třídu situací (např. středně tenký film\footnote{$\Delta n\w d/c\ll 1$, kde $\Delta n$ je štepení indexu lomu vlivem anizotropie.}) se zabývá dodatek \ref{k:dodatek vypocet}.

V uvedeném vzorci jsou patrné dva druhy příspěvků: lineární v $\M$, který vymizí při kolmém dopadu a kvadratický v $\M$, který je nenulový i při kolmém dopadu.
Jak bylo avizováno, kvadratický člen v sobě obsahuje i člen $K^2$.
Lze také vidět, že při kolmém dopadu $B=0$ a izotropním $G$ tenzoru $\Delta G=0$ nezáleží signál na natočení vzorku a dostaneme
\begin{equation}
    \Psi_{s/p}=&\pm A_{s/p} \left[ 2G_{44}-\frac{K^2}{\varepsilon^0}  \right] M_x M_y \,,
\end{equation}
což je ekvivalentní \eqref{e:PMLD}.


\begin{figure}\center
\includegraphics[scale=0.35]{./img/voigtgeometry.png}
\caption{Voigtova geometrie\cite{systematicgamnas}. Kolmý dopad a magnetizace je transverzální - kolmo na směr šíření světla.}\label{f:Voigt}
\end{figure}


Při nekolmém dopadu se lineární i kvadratický člen sčítají a pro kvantitativní analýzu je třeba je nějakým způsobem oddělit.
Metodám, které se tím zabývají, věnujeme zbytek této kapitoly.
