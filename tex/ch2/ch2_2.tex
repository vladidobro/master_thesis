\section{Metoda rotujícího pole/ROTMOKE}
\label{chap:rotmoke}
\endinput
Metoda rotujícího pole\cite{rotpole1}\cite{rotpole2} rozšiřuje osmisměrnou metodu na Stoner-Wohlfarthův model, tedy situace, kdy pole je dostatečně silné, aby vzorek byl v jednodoménovém stavu a magnetizace byla saturovaná $|\M|=M_S$, ale ne dostatečně silné na to, aby se neprojevila magnetická anizotropie $\phim\neq\phih$.

Přiložené pole se při konstantí velikosti otáčí v rovině vzorku a měří se stočení polarizace, viz obr. \ref{f:rotujici pole}.
Předpokladem pro oddělení lineárního a kvadratického příspěvku je zde symetrie in-plane inverze magnetické anizotropie, tj. $\M(-\Hext)=-\M(\Hext)$, $\phim(\phih+\SI{180}{\degree})=\phim(\phih)+\SI{180}{\degree}$.
Pokud není narušená, je lineární $\Phi^L$, resp. kvadratický $\Phi^Q$ příspěvek lichý, resp. sudý nejen v $\M$, ale i v $\Hext$.
Proto je možné je jednoduše oddělit
\begin{equation}
\Phi^L(\phih)=\frac{1}{2} \left[ \Phi(\phih+\SI{180}{\degree}) - \Phi(\phih) \right] \,, \, \Phi^Q(\phih)=\frac{1}{2} \left[ \Phi(\phih+\SI{180}{\degree}) + \Phi(\phih) \right] \,.
\end{equation}
Lineární signál má jednoduchou závislost na směru magnetizace, lze díky němu získat $\phim(\phih)$
\begin{equation}
    \Phi^L(\phih)\propto \cos(\phim(\phih)) \,, \qquad \cos \phim=\Phi^L(\phih)/\Phi^L_\textrm{max} \,,
\end{equation}
kde $\Phi^L_\textrm{max}$ je maximální naměřený lineární signál pro $\phim=0$.

Z celkového signálu je tedy odseparovaný kvadratický signál, z lineárního je určen směr $\phim$ pro každé $\phih$, a dále už je postup stejný jako u osmisměrné metody, tedy pro každé natočení vzorku $\gamma$ určení koeficientů jednotlivých členů v \eqref{e:Silber QMOKE vzorec}, tentokrát už ale podle známého $\phim$ (viz obr. \ref{f:rotpolevysledky}.

\begin{figure}\center
\includegraphics[scale=0.5]{./img/rotpole.png}
\caption{Schématické znázornění metody rotujícího pole\cite{rotpole2}. Při šikmém dopadu je měřeno stočení polarizace během otáčení magnetického pole v rovině vzorku $xy$.}\label{f:rotujici pole}
\end{figure}

\begin{figure}
    \centering
    \includegraphics[scale=0.45]{./img/rotpolevysledky.png}
    \caption{Ilustrace metody rotujícího pole (Fe(12nm)/GaAs(001)) \cite{rotpole2}.
    Měřené (a) stočení polarizace v závislosti na experimentálně ovladatelném $\phiH$, rozdělené na lineární (c) a kvadratický (e) příspěvek. V (b),(d),(f) jsou signály vyjádřené v závislosti na $\phim$ určeném z (c). To, že se (a,c,e) a (b,d,f) navzájem liší dokazuje, že magnetická anizotropie není zanedbatelná, a materiál nebylo možné měřit osmisměrnou metodou.}
    \label{f:rotpolevysledky}
\end{figure}

Metoda rotujícího pole je velmi podobná metodě ROTMOKE z roku 1999\cite{ROTMOKE}, která také v rotujícím in-plane vnějším poli separuje lineární a kvadratické jevy stejným způsobem.
Narozdíl od metody rotujícího pole je však cílem pouze magnetometrie pomocí lineárního jevu, kvadratické jevy jsou považovány pouze za artefakty a odstraněny ze zpracování.
