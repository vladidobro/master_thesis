\section{Metoda rotujícího pole, ROTMOKE}
\label{chap:ROTMOKE}

Metoda rotujícího pole\cite{liangSeparationLinearQuadratic2015,liangQuantitativeStudyQuadratic2015} rozšiřuje osmisměrnou metodu na Stoner-Wohlfarthův model: $\vHext$ je dostatečně silné na to, aby vzorek byl v jednodoménovém stavu a magnetizace byla saturovaná $|\M|=M_S$, ale nedostatečně na to, aby se neprojevila magnetická anizotropie $\phim\neq\phih$.

Přiložené pole se při konstantí velikosti otáčí v rovině vzorku a je měřeno stočení polarizace (příp. elipticita), viz obr. \ref{fig:metoda-rotujiciho-pole}.
Předpokladem pro oddělení lineárního a kvadratického příspěvku, který supluje striktní magnetickou saturaci osmisměrné metody,
je zde symetrie in-plane inverze magnetické anizotropie, tj. $\vec{M}(-\vHext)=-\vec{M}(\vHext) \Leftrightarrow \phim(\phih+\SI{180}{\degree})=\phim(\phih)+\SI{180}{\degree}$.
Pokud není narušená, lineární $\Psi^L$, resp. kvadratický $\Psi^Q$ příspěvek signálu je lichý, resp. sudý nejen v $\vec{M}$, ale i v $\vHext$;
proto je možné je jednoduše oddělit\footnote{V původních článcích je Kerrova rotace $\Psi$ značena $\Phi$.}
\begin{align}
    \Psi^L(\phih)&=\frac{1}{2} \left[ \Psi(\phih+\SI{180}{\degree}) - \Psi(\phih) \right] 
    \,, \\ \Psi^Q(\phih)&=\frac{1}{2} \left[ \Psi(\phih+\SI{180}{\degree}) + \Psi(\phih) \right] \,.
\end{align}

Lineární signál má podle \eqref{eqn:QMOKE-vzorec} jednoduchou závislost na směru magnetizace, takže z něj lze odečíst magnetickou anizotropii, tj. $\phim(\phih)$
\begin{equation}
    \Psi^L(\phih)\propto \cos(\phim(\phih)) \,, \qquad \cos \phim=\Psi^L(\phih)/\Psi^L_\textrm{max} \,,
\end{equation}
kde $\Psi^L_\textrm{max}$ je maximální naměřený lineární signál pro $\phim=\SI{0}{\degree}$.

\begin{figure}[htbp]
    \centering
    \includegraphics{./img/static/rotpole-metoda-liang.pdf}
    \caption{Metoda rotujícího pole.
    Při šikmém dopadu je měřeno stočení polarizace během otáčení magnetického pole v rovině vzorku $xy$. \cite{liangQuantitativeStudyQuadratic2015}}
    \label{fig:metoda-rotujiciho-pole}
\end{figure}

Celá metoda tedy spočívá ve třech krocích, pro ilustraci výsledků viz obr. \ref{fig:metoda-rotujiciho-pole-vysledky}
\begin{enumerate}
    \item Změřený signál je rozdělen na $\Psi^L$ a $\Psi^Q$.
    \item Z $\Psi^L$ je určena magnetická anizotropie: $\phim$ pro každé $\phih$.
    \item Stejný jako u osmisměrné metody: opakování pro každé natočení vzorku $\gamma$ a určení koeficientů jednotlivých členů v \eqref{eqn:QMOKE-vzorec}, tentokrát však už podle známého $\phim$. \label{item:rotujici-pole-analyza}
\end{enumerate}

\begin{figure}[htbp]
    \centering
    \includegraphics{./img/static/rotpole-vysledky-liang.pdf}
    \caption{Ilustrace výsledků metody rotujícího pole na vzorku Fe(12nm)/GaAs(001).
    (a) Měřené stočení polarizace v závislosti na experimentálně ovladatelném $\phih$, rozdělené na (c) lineární a (e) kvadratický příspěvek. (b, d, f) Stejná stočení vynesená v závislosti na $\phim$ určeném z (c). 
    To, že se (a, c, e) a (b, d, f) navzájem liší dokazuje, že magnetická anizotropie není zanedbatelná, a materiál nebylo možné měřit osmisměrnou metodou. \cite{liangSeparationLinearQuadratic2015}}
    \label{fig:metoda-rotujiciho-pole-vysledky}
\end{figure}

Metoda rotujícího pole je velmi podobná metodě ROTMOKE\cite{mattheisDeterminationAnisotropyField1999} z roku 1999, která také v rotujícím in-plane vnějším poli separuje lineární a kvadratické jevy stejným způsobem.
Narozdíl od metody rotujícího pole je však cílem pouze magnetometrie pomocí lineárního jevu (je vynechán krok č. \ref{item:rotujici-pole-analyza}), kvadratické jevy jsou považovány za artefakty a odstraněny ze zpracování.
