\section{Optický můstek}
\label{chap:mustek-kap2}
V MO experimentech, ve kterých se měří stočení hlavní roviny polarizace či elipticita, se často využívá schéma optického můstku\footnote{angl. \emph{optical bridge}, také \emph{differential intensity measurement}.}\cite{silberQuadraticMagnetoopticKerr2019a}.
V jednoduché formě je znázorněn na obr. \ref{fig:mustek-schema}.

\begin{figure}[htbp]
    \centering
    \missingfigure{a}
    \caption{Mustekmustek-schema}
    \label{fig:mustek-schema}
\end{figure}

Ústředním prvkem je polarizační dělič (často Glan Laser či Wollastonův hranol), který rozdělí svazek do dvou lineárně polarizovaných ramen.
Při měření je nejprve nastaven úhel rotace půlvlnné destičky $\theta_{\lambda/2}$ (``vyvážení můstku'') tak, aby oba detektory detekovali stejnou intenzitu.
V Jonesově formalismu\todo{spocitat} pro ideální prvky
\begin{align}
    \J_\textrm{A} = \mathcal{T}_\textrm{A} \mathcal{T}_{\lambda/2} \J_\textrm{in} \,&, \quad 
    \mathcal{T}_\textrm{A}=\begin{pmatrix} 1&0\\0&0 \end{pmatrix} \,, \\
    \J_\textrm{B} = \mathcal{T}_\textrm{B} \mathcal{T}_{\lambda/2} \J_\textrm{in} \,&, \quad
    \mathcal{T}_\textrm{B}=\begin{pmatrix} 0&0\\0&1 \end{pmatrix} \,,\\
    \mathcal{T}_{\lambda/2}=\mathcal{R}(\theta_{\lambda/2})\begin{pmatrix} 1&0\\0&-1 \end{pmatrix}\mathcal{R}(-\theta_{\lambda/2}) \,&, \quad
    \mathcal{R}(\theta)=\begin{pmatrix} \cos\theta&\sin\theta\\-\sin\theta&\cos\theta \end{pmatrix} \,.
\end{align}

Dosazením Jonesova vektoru \eqref{eqn:Jones-elipsa} odpovídajícímu natočení hlavní roviny polarizace $\beta$ a elipticitě $\chi$ dostaneme rozdíl intenzit\todo{opsat ze sesitu}
\begin{equation}
    I_\textrm{A}-I_\textrm{B}\equiv\J_\textrm{A}^\dagger\J_\textrm{A}-\J_\textrm{B}^\dagger\J_\textrm{B}=I\sin(2\beta-4\theta_{\lambda/2})\cos2\chi \,, \quad I_\textrm{A}+I_\textrm{B}=I
\end{equation}

Vyvážením můstku nastavíme $\sin(2\beta-4\theta_{\lambda/2})=0$, pro malé výchylky $\textrm{d}\beta, \textrm{d}\chi, \textrm{d}I$ pak platí
\begin{equation}
\label{eqn:A-B-mustek}
    \frac{1}{2I}\textrm{d}(I_\textrm{A}-I_\textrm{B})=\cos(2\chi) \textrm{d}\beta \,.
\end{equation}
V měřeném signálu se tedy neprojeví změny elipticity ani intenzity.
Pro $\chi=0$ je měřený signál dokonce přímo $\textrm{d}\beta$.
Vhodným vložením čtvrtvlnné destičky lze měřit i elipticitu\cite{silberQuadraticMagnetoopticKerr2019a}.

Optický můstek zlepšuje úroveň šumu dvěma způsoby.
Zaprvé umožňuje, aby k odečítání blízkých čísel docházelo na úrovni předzesilovače, a zadruhé se díky \eqref{eqn:A-B-mustek} neprojevují fluktuace intenzity
Detailnějšímu popisu optického můstku se věnuje oddíl \ref{chap:detekce}.

