\section{Osmisměrná metoda}
\label{chap:osmismerna-metoda}
\endinput
Základem osmisměrné metody\cite{osmismerna} je vzorec \eqref{e:Silber QMOKE vzorec}.
Všechny tři členy mají různou závislost na směru magnetizace, takže měřením signálu pro různé směry in-plane magnetizace je možné je od sebe oddělit.
V praxi se měří rotace pro 8 směrů vnějšího pole $\Hext$ jako na obr. \ref{f:osmismerna} (b).
Koeficienty jednotlivých členů \eqref{e:Silber QMOKE vzorec} se pak získají vhodnými součty a odečty signálu v různých směrech.
Podmínkou je však striktní magnetická saturace, tj. pole musí být dostatečně silné, aby magnetizace byla pro všechny směry saturovaná ve směru pole (vždy platilo $\phim=\phih$ a $|\M|=M_S$).

Měření je opakováno pro různé natočení vzorku $\gamma$ a koeficienty se mění jako v rovnici \eqref{e:Silber QMOKE vzorec} (viz obr. \ref{f:osmismerna}).
Vzorec \eqref{e:Silber QMOKE vzorec} v tomto tvaru platí pouze pro [001] orientované vzorky, nicméně metoda byla rozšířena\cita[Silber] pro [011] a [111] orientované kubické vzorky, ve kterých má podobný tvar\cite{Hamrlova}.
\begin{figure}\center
\includegraphics[scale=0.4]{./img/8dir.png}
\caption{Osmisměrná metoda\cite{osmismerna}. (a) Vzorek je otáčen v rovině $xy$ o úhel $\gamma$. (b) Osm pevných směrů, ve kterých je přikládáno magnetické pole.}\label{f:osmismerna}
\end{figure}

\begin{figure}\center
\includegraphics[scale=0.45]{./img/rotmokevysledek.png}
\caption{Naměřená závislost koeficientů MO jevů na natočení vzorku (epitaxní vrstva Fe na MgO) \cite{osmismerna}. ($\circ$) koeficient lineárního jevu, ($+$) koeficient $M_xM_y$, ($\diamond$) koeficient $M_x^2-M_y^2$. Harmonická závislost potvrzuje platnost vzorce \eqref{e:Silber QMOKE vzorec} a striktní magnetickou saturaci v daném případě.}\label{f:osmismerna}
\end{figure}
