\thesistype{DIPLOMOVÁ PRÁCE}
\titleCS{Laserová spektroskopie spintronických materiálů}
\titleEN{Laser spectroscopy of spintronic materials}
\yearsubmitted{2022}

\author{Vladislav Wohlrath}
\departmentCS{Katedra chemické fyziky a optiky}
\departmentEN{Department of Chemical Physics and Optics}
\studyprogramme{Fyzika}
\studybranch{Optika a optoelektronika}

\supervisor{prof. RNDr. Petr Němec, Ph.D.}
\supervisordepartmentCS{Katedra chemické fyziky a optiky}
\supervisordepartmentEN{Department of Chemical Physics and Optics}

\dedication{Tímto děkuji vedoucímu prof. Petrovi Němcovi za cenné zkušenosti, poskytnuté příležitosti a obecně férový přístup.
Také za dohled nad prováděným experimentem a zkušené rady při tvorbě textu.

Děkuji doc. Tomášovi Ostatnickému za teoretickou podporu a mnohé plodné konzultace.

Děkuji Zeynab Sadeghiové, MSc. za spolupráci při měření závěrečné sady dat se vzorkem FeRh.

Děkuji Lukášovi Nádvorníkovi, Ph.D. za poskytnutí měřeného vzorku CoFe.

Děkuji Mgr. Jozefovi Kimákovi za uvedení do probíhajícího experimentu a také za pochopení a vstřícný přístup při mém opakovaném zapomínání klíčů od laboratoře.

Děkuji kolektivu pracovníků a studentů působících na pracovišti, kromě již zmíněných jmenovitě Evě Schmoranzerové, Ph.D., Mgr. Peterovi Kubaščíkovi, Mgr. Miloslavovi Surýnkovi, Bc. Jiřímu Jechumtálovi a Deepovi Joshimu, MSc.
Děkuji jim za inspiraci, občasnou pomoc při práci a hlavně za zpříjemnění stráveného času.

Děkuji rodině a přátelům za podporu.
Zvláštní dík si zaslouží Bc. Jaroslav Pešek, v jehož inkubátoru diplomových prací vznikla podstatná část tohoto textu.}

\abstractCS{Práce se věnuje magnetooptics.}
\abstractEN{Thesis deals with magnetooptics.}
\keywordsCS{spintronika, kvadratická magneto-optika, Voigtův jev, CoFe, FeRh}
\keywordsEN{spintronics, quadratic magneto-optics, Voigt effect, CoFe, FeRh}
