\thesistype{DIPLOMOVÁ PRÁCE}
\titleCS{Laserová spektroskopie magneticky uspořádaných materiálů}
\titleEN{Laser spectroscopy of magnetically-ordered materials}
\yearsubmitted{2022}

\author{Vladislav Wohlrath}
\departmentCS{Katedra chemické fyziky a optiky}
\departmentEN{Department of Chemical Physics and Optics}
\studyprogramme{Fyzika}
\studybranch{Optika a optoelektronika}

\supervisor{prof. RNDr. Petr Němec, Ph.D.}
\supervisordepartmentCS{Katedra chemické fyziky a optiky}
\supervisordepartmentEN{Department of Chemical Physics and Optics}

\dedication{Tímto děkuji vedoucímu prof. RNDr. Petrovi Němcovi, Ph.D. za cenné zkušenosti, poskytnuté příležitosti a obecně férový přístup.
Také za dohled nad prováděným experimentem a zkušené rady při tvorbě textu.

Děkuji doc. RNDr. Tomášovi Ostatnickému, Ph.D. za teoretickou podporu a mnohé plodné konzultace.

Děkuji Zeynab Sadeghi, MSc. za spolupráci při měření závěrečné sady dat se vzorkem FeRh.

Děkuji RNDr. Lukášovi Nádvorníkovi, Ph.D. za poskytnutí měřeného vzorku CoFe.

Děkuji Mgr. Jozefovi Kimákovi za uvedení do probíhajícího experimentu a také za pochopení a vstřícný přístup při mém opakovaném zapomínání klíčů od laboratoře, kterým jsem ho obtěžoval.

Děkuji kolektivu pracovníků a studentů působících na pracovišti, kromě již zmíněných jmenovitě RNDr. Evě Schmoranzerové, Ph.D., Mgr. Peterovi Kubaščíkovi, Mgr. Miloslavovi Surýnkovi, Bc. Jiřímu Jechumtálovi a Deepovi Joshimu, \linebreak MSc.
Děkuji jim za inspiraci, občasnou pomoc při práci a hlavně za zpříjemnění stráveného času.

Děkuji rodině a přátelům za podporu.
Zvláštní dík si zaslouží Bc. Jaroslav Pešek, v jehož inkubátoru diplomových prací vznikla podstatná část tohoto textu.}


\abstractCS{V Laboratoři OptoSpintroniky je dlouhodobě vyvíjena experimentální metoda pro studium magnetických vzorků pomocí magneto-optických jevů kvadratických v magnetizaci vzorku, jako je například Voigtův jev.
Vlivem použité experimentální geometrie je v našem případě, na rozdíl od obdobných metod, možné používat také kryostat.
Díky tomu je možné příslušné vzorky studovat jak za snížené, tak i za zvýšené teploty.
V rámci této diplomové práce bylo identifikováno a následně odstraněno několik problémů, které praktické využití této metody doposud znemožňovaly.
Použití metody bylo demonstrováno v transmisní i reflexní geometrii pro feromagnetické vzorky CoFe a FeRh.
Naše měření ukázala, že koeficient popisující kvadratickou magneto-optickou odezvu může být silně anizotropní, přičemž velikost této anizotropie i její znaménko silně závisí na použité vlnové délce světla.
To má poměrně zásadní důsledky pro plánování a/nebo vyhodnocování příslušných experimentů využívajících kvadratickou magneto-optiku, která v současné době přitahuje rostoucí pozornost díky nástupu zájmu o antiferomagnetickou spintroniku.}

\abstractEN{For a relatively long time, a new experimental technique for the study of magnetic materials by magneto-optical effects quadratic in sample magnetization, like Voigt effect, is being developed at Laboratory of OptoSpintronics. Thanks to the used experimental geometry, in our setup, unlike in different methods, it is possible to use a cryostat. Thanks to this the investigated samples can be studied both at low and high temperatures. In this master thesis we identified and eventually solved several problems that prevented the practical utilization of this technique in the past. We successfully demonstrated the application of this method both in transmission and reflection geometries using ferromagnetic samples of CoFe and FeRh. Our measurements revealed that the coefficient describing the quadratic magneto-optical response can be strongly anisotropic with a wavelength-dependent magnitude and sign. This observation has strong consequences for the design and/or interpretation of experiments based on quadratic magneto-optics, which are gaining an increasing attention nowadays thanks to the growing interest in antiferromagnetic spintronics.}


\keywordsCS{spintronika, kvadratická magneto-optika, Voigtův jev, CoFe, FeRh}
\keywordsEN{spintronics, quadratic magneto-optics, Voigt effect, CoFe, FeRh}
