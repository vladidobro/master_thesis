\section{CoFe}
\label{chap:vzorek-cofe}

Vzorek byl poskytnut prof. Y. Z. Wu z \emph{Department of Physics, State Key Laboratory of Surface Physics, Fudan Univerzity, Shangai} v Číně a jedná se o jeden ze skupiny vzorků studovaných v \cite{zengIntrinsicMechanismAnisotropic2020}.
Tento oddíl je souhrnem relevantních informací z původního článku \cite{zengIntrinsicMechanismAnisotropic2020}, poskytnutých doprovodných dokumentů k tomuto článku a naší korespondence s pracovníky tohoto pracoviště.

Ve slitině \ch{Co_xFe_{1-x}} byl z prvních principů předpovědězen intrinsický mechanismus anizotropní magnetoresistence (AMR): dochází ke křížení energetických pásů, které je závislé na orientaci magnetizace.
Změnou poměru \ch{Co} a \ch{Fe} je navíc možné posouvat tyto body křížení vzhledem k Fermiho hladině a tak ladit velikost AMR.
Po vypěstování monokrystalu metodou MBE bylo provedeno magneto-transportní měření a předpověď silného AMR byla potvrzena\cite{zengIntrinsicMechanismAnisotropic2020}.
AMR je projevem magnetické závislosti tenzoru vodivosti, a do jisté míry je možné ho považovat za $\omega\to 0$ limitu Voigtova jevu\cite{tesarovaSystematicStudyMagnetic2014}.

CoFe je z hlediska reálných spintronických aplikací perspektivní zejména z toho důvodu, že Co i Fe jsou velice snadno dostupné materiály a samotné CoFe je již jinými způsoby široce používané v současných technologiích.

Vzorek studovaný v této práci je v původním článku označen $x=0.5$, je to \SI{10}{\nano\meter} vrstva monokrystalu \ch{Co_{0,5}Fe_{0,5}} na substrátu \ch{MgO}(001), s \SI{3}{\nano\meter} nadvrstvou \ch{MgO}(001).
Vzorek má kubickou mřížku.
Fotografie vzorku s vyznačenými krystalografickými osami a definicí úhlů je na obr. \ref{fig:vzorek-cofe} (a).

Metodou \emph{torque-metry} byla změřena in-plane magnetická anizotropie vzorku, změřený torque je na obr. \ref{fig:vzorek-cofe} (b).
Torque byl fitovaný SW modelem \eqref{eqn:SW-funkcional} s výsledkem $H_4=\SI{605}{Oe}$, $H_u=\SI{126}{Oe}$, $\phiu=\SI{114}{\degree}$.
Natočení snadného směru uniaxiální anizotropie $\phiu$ je odečítáno od jednoho z hlavních krystalografických směrů [100] nebo [010], z naší korespondence s autory článku \cite{zengIntrinsicMechanismAnisotropic2020} však není zřejmé od kterého a jakým směrem.
V našem značení \eqref{eqn:SW-funkcional} a jednotkách SI to odpovídá (záporné znaménko $k_4$ značí snadnou osu pootočenou oproti [100] o \SI{45}{\degree})
\begin{equation}
\label{eqn:cofe-anizotropni-konstanty-cina}
    k_4 = -\SI{30.25}{\milli\tesla} \,,\quad k_u = \SI{6.3}{mT} \,.
\end{equation}

\todocite{supplemental cofe}
\begin{figure}[htbp]
    \centering
    \begin{tikzpicture}
\begin{scope}[rotate=20,yshift=1cm]
    \filldraw[fill=gray!20](-2.8,-1.25) -- (-2.8,1.25) -- (-1,1.25) -- (-1,-1) arc [start angle=90, end angle=180, radius=0.25cm] -- cycle;
    \draw[thick] (-3.5,-1.25) -- (-3.5,1.25) -- (-1,1.25) -- (-1,-1.25) -- cycle;
    \draw[color=green!50!black] (-2.25,-1.25cm) arc[start angle=270, end angle=250, radius=1.25cm];
    \fill[fill=green!20] (-2.25,0) -- (-2.25,-1.25cm) arc[start angle=270, end angle=250, radius=1.25cm] -- cycle;
    \draw (-2.25,0) -- (-2.25,-2);
    \draw[->] (-2.25,0) -- +(-110:2cm) node[anchor=east] {$x$};
    \draw[->] (-2.25,0) -- +(-20:2cm) node[anchor=south] {$y$};
    \path (-2.25,0) ++(-100:1cm) node {$\gamma$};

    \draw[<->] (-3.5,1.45) -- node[anchor=south,rotate=20] {CoFe[100]/MgO[110]} (-1,1.45);

\end{scope}
    \path (3.5,0) node {\includegraphics[width=7cm]{./img/static/a/cofe-torque-sampleinfo_a.pdf}};
    \path (-5,2) node {(a)};
    \path (0,2) node {(b)};
\end{tikzpicture}

    \caption{(a) Měřený vzorek CoFe, zavedení úhlu in-plane rotace $\gamma$. Neexponovaná hrana a roh dovolují snadnou orientaci. Při pohledu čtenáře je vzorek nahoře, substrát dole. (b) Výsledek torque-metrie, fitem byly určeny anizotropní konstanty \eqref{eqn:cofe-anizotropni-konstanty-cina}\cite{zengIntrinsicMechanismAnisotropic2020}.}
    \label{fig:vzorek-cofe}
\end{figure}
