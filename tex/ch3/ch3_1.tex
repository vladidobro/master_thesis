\section{CoFe \cite{AMRCoFe}}

Ve slitině CoFe byl z prvních principů předpovědězen intrinsický mechanismus anizotropní magnetoresistence (AMR) - dochází ke křížení energetických pásů, které je závislé na orientaci magnetizace \cite{AMRCoFe}.
Změnou poměru Co a Fe je navíc možné posouvat tyto body křížení vzhledem k Fermiho hladině a tak ladit velikost AMR.
Po vypěstování monokrystalu metodou MBE bylo provedeno magneto-transportní měření a předpověď silného AMR byla potvrzena \cite{AMRCoFe}.

AMR je důsledkem magnetické závislosti tenzoru vodivosti, a do jisté míry je možné ho považovat za $\w\to 0$ limitu Voigtova jevu \cite{Ostatnicky}.

Vzorek byl poskytnut prof. Y. Z. Wu z Department of Physics, State Key Laboratory of Surface Physics, Fudan University, Shanghai v Číně.
Jedná se o monokrystal Co$_{\num{0.5}}$Fe$_{\num{0.5}}$ (\SI{10}{\nano\meter}) na substrátu MgO(001) a s nadvrstvou MgO(001) (\SI{3}{\nano\meter}).

Obrázky...
