\section{\ch{Co_{0,5}Fe_{0,5}}}
\label{chap:vzorek-cofe}

Vzorek byl poskytnut prof. Y. Z. Wu z \emph{Department of Physics, State Key Laboratory of Surface Physics, Fudan Univerzity, Shangai} v Číněa jedná se o jeden ze skupiny vzorků studovaných v \todocite{amr}.
Pokud není uvedeno jinak, vychází všechny informace v tomto oddílu z uvedeného článku a poskytnutých doprovodných informací.

Ve slitině \ch{Co_xFe_{1-x}} byl z prvních principů předpovědězen intrinsický mechanismus anizotropní magnetoresistence (AMR) -\tododash dochází ke křížení energetických pásů, které je závislé na orientaci magnetizace.
Změnou poměru \ch{Co} a \ch{Fe} je navíc možné posouvat tyto body křížení vzhledem k Fermiho hladině a tak ladit velikost AMR.
Po vypěstování monokrystalu metodou MBE bylo provedeno magneto-transportní měření a předpověď silného AMR byla potvrzena.
AMR je projevem magnetické závislosti tenzoru vodivosti, a do jisté míry je možné ho považovat za $\omega\to 0$ limitu Voigtova jevu.

Vzorek studovaný v této práci je v původním článku označen $x=0.5$, je to \SI{10}{\nano\meter} vrstva monokrystalu \ch{Co_{0,5}Fe_{0,5}} na substrátu \ch{MgO}(001), s \SI{3}{\nano\meter} nadvrstvou \ch{MgO}(001).
Vzorek má kubickou mřížku.
Fotografie vzorku s vyznačenými krystalografickými osami a definicí úhlů je na obr. \ref{fig:vzorek-cofe} (a).

Pro vzorek byla také torque-metrií změřena jeho in-plane magnetická anizotropie, změřený torque je na obr. \ref{fig:vzorek-cofe} (b).
Torque byl fitovaný SW modelem \eqref{eqn:SW-funkcional} s výsledkem $H_4=\SI{605}{Oe}$ a $H_u=\SI{126}{Oe}$.
V našem značení \eqref{eqn:SW-funkcional} a jednotkách to odpovídá\todo{spocitat}
\begin{equation}
    k_4 = \SI{0}{\milli\tesla} \,,\quad k_u = \SI{0}{mT} \,,\quad \phiu = \SI{0}{\degree} \,.
\end{equation}

\begin{figure}[htbp]
    \centering
    \missingfigure{cofe supplemental}
    \caption{(a) vzorek cofe a (b) torque}
    \label{fig:vzorek-cofe}
\end{figure}
