Funkčnost metody jsme ověřili jak v transmisní, tak reflexní geometrii.
Za zásadní nesporný důkaz funkčnosti považujeme určení magnetické anizotropie.
Výsledky získané různými vlnovými délkami jsou ve velmi dobré shodě, což je vzhledem k spektrální rozmanitosti MO koeficientů velice přesvědčivé: pokud bychom nepozorovali přímo signál od vzorku způsobený kvadratickým MO jevem, musel byl dodatečný nežádoucí signál mít stejnou spektrální závislost.

Na druhou stranu se ale vyskytuje problém zmíněný v oddílu \ref{}, že anizotropie změřené různou velikostí pole $\Hext$ či různém natočení vzorku jsou odlišné.
To je pravděpodobně způsobeno nepřesnostmi v magnetickém poli, jejichž dodatečná charakterizace je v době odevzdání práce v plánu.
Proto je třeba brát uvedené výsledky magnetické anizotropie s rezervou, uvádíme je především pro ilustraci spolehlivosti metody.

Uvádíme pouze finální sady měření.
Pokud byla použita zrcadla mezi vzorkem a detekcí, byla vždy kompenzována druhým ``ortogonálním'' zrcadlem.

Na závěr jsme v širší míře zopakovali měření fázového přechodu feromagnet--antiferomagnet ve FeRH, provedené v \cite{kubascikMagnetooptickeStudiumAntiferomagnetickych2019}.
