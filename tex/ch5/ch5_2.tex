\section{Měření v reflexní geometrii: FeRh}
\label{chap:vysledky-ferh}

\subsection{Měření feromagnetické fáze}
\label{chap:ferh-fm}

Měření proběhlo při teplotě \SI{420}{\kelvin} v reflexní geometrii, schéma je na obr. \ref{fig:ferh-fm-schema-data} (a).
Úhel dopadu byl \SI{0.5}{\degree}.
Orientace vzorku byla vždy přibližně $\gamma\approx\SI{0}{\degree}$ (viz obr. \ref{}), kromě jednoho kontrolního měření při $\lambda=\SI{1600}{\nano\meter}$, ve kterém bylo $\gamma\approx\SI{-90}{\degree}$.
Tento bod je ve všech obrázcích náležitě označen.
Příklad změřených dat je na obr. \ref{fig:ferh-fm-schema-data} (b).

Všechny vlnové délky byly změřeny s velikostí pole $\Hext=\SI{50}{\milli\tesla}$, některé navíc s $\Hext=\SI{207}{\milli\tesla}$.
Výsledná magnetická anizotropie je znázorněná na obr. \ref{fig:ferh-fm-schema-data} (c), všechny vlnové délky jsou v dobré shodě (pro každé $\Hext$ zvlášť).
MO koeficienty jsou znázorněny na obr. \ref{fig:ferh-fm-pmld}.

Zde je patrný již dvakrát zmíněný problém: uniaxiální magnetická anizotropie není konzistentní napříč velikostmi $\Hext$ a natočeními vzorku $\gamma$.
Přibližně konstantní poměr $k_u/\Hext$ a její směr (v soustavě spojené s magnetem) nezávislý na natočení vzorku napovídá, že nejde o anizotropii vzorku, ale magnetu.
Pro více informací viz dodatek \ref{app:magneticka-anizotropie}.

\begin{figure}[htbp]
    \centering
    \missingfigure{}
    \caption{(a) schéma (b) data}
    \label{fig:ferh-fm-schema-data}
\end{figure}

\begin{figure}[htbp]
    \centering
    \missingfigure{}
    \caption{pmld}
    \label{fig:ferh-fm-pmld}
\end{figure}

\subsection{Měření přechodu feromagnet--antiferomagnet}
\label{chap:ferh-field-cooling}

Jak bylo zmíněno v oddíle \ref{chap:vzorek-ferh}, polohu Néelova vektoru ve FeRh lze ovlivnit pomocí chlazení ve vnějším poli.
Předpokládáme, že po aplikaci pole ve směru $\phih$ ve FM fázi (\SI{420}{\kelvin}) a následnému zchlazení do AFM fáze (\SI{320}{\kelvin}) bude vzorek v AFM stavu popsaném Néelovým vektorem v rovině vzorku, jeho směr označíme $\phin(\phih)$.
