\section{Měření v reflexní geometrii: FeRh}
\label{chap:vysledky-ferh}

\subsection{Měření feromagnetické fáze}
\label{chap:ferh-fm}

Měření proběhlo při teplotě \SI{420}{\kelvin} v reflexní geometrii podle schématu na obr.~\ref{fig:ferh-schema}.
Úhel dopadu byl \SI{0.5}{\degree}.
Orientace vzorku byla vždy přibližně $\gamma\approx\SI{0}{\degree}$ (viz obr. \ref{fig:vzorek-ferh} (a)), kromě jednoho kontrolního měření při $\lambda=\SI{1600}{\nano\meter}$, ve kterém bylo $\gamma\approx\SI{-90}{\degree}$.
Všechny vlnové délky byly změřeny s velikostí pole $\Hext=\SI{50}{\milli\tesla}$, některé navíc s $\Hext=\SI{207}{\milli\tesla}$.

Příklad změřených dat je na obr. \ref{fig:ferh-fm-schema-data}.
Opět jako v případě měření CoFe není patrný žádný příspěvek lineárních jevů (mezi daty před a po symetrizaci v $\vHext$ není přílišný rozdíl).
Pro úplnost poznamenáme, že celá sada dat byla změřena ještě jednou s úhlem dopadu \SI{1}{\degree} (výsledky zde neuvádíme).
Zjistili jsme, že už takto malý úhel dopadu byl dostačující na to, aby amplituda lineárních MO jevů byla porovnatelná s MLD a výsledky následné analýzy pak zdaleka nebyly tak přesvědčivé.

Standardní procedurou jsme určili MO koeficienty $P$ a magnetickou anizotropii vzorku.
Spektrální závislost $P$ je vynesena na obr.~\ref{fig:ferh-fm-pmld}.
Zde musíme však mít na paměti, že u tohoto vzorku máme úhel natočení $\gamma$ definován vzhledem ke směru FeRh[1-10] (viz obr. \ref{fig:vzorek-ferh} (a)), takže \eqref{eqn:PminusG} platí s opačným znaménkem\footnote{Koeficient $P_-$ definujeme vzhledem k úhlu natočení $\gamma$ tak, aby platil vzorec \eqref{eqn:PMLD-matice}. Jiná situace je u magnetických anizotropních konstant, které jsou definovány pevně vzhledem ke směru FeRh[100], nezávisle na tom, jakou krystalografickou osu označujeme úhlem $\gamma$. Proto do definice vlnkovaných $\tilde{k}_4$, $\tilde{k}_u$ \eqref{eqn:tilda-k} musíme za $\gamma$ dosazovat místo \SI{0}{\degree}, resp. \SI{-90}{\degree} směry FeRh[100], tzn. \SI{45}{\degree}, resp. \SI{-45}{\degree}.}.
Spektrum je velice pestré, oba koeficienty několikrát mění znaménko.
Tak silná anizotropie a spektrální závislost MO koeficientu, pokud je nám známo\cite{hamrleHugeQuadraticMagnetooptical2007,liangQuantitativeStudyQuadratic2015,silberQuadraticMagnetoopticKerr2019}, dosud nebyla pozorována v žádném materiálu.
Poznamenáváme, že takto silná anizotropie kvadratického MO koeficientu, která navíc velice silně závisí na vlnové délce, má významné důsledky pro metody studia materiálů pomocí MLD (např. měřením hysterézních smyček\cite{tesarovaSystematicStudyMagnetic2014} nebo pomocí metody excitace a sondování\cite{saidlOpticalDeterminationNeel2017}).
Běžně je totiž předpokládáno, že experimentálně detekovaná změna velikosti MO signálu je určena výhradně změnou polohy\cite{tesarovaSystematicStudyMagnetic2014} nebo velikosti\cite{saidlOpticalDeterminationNeel2017} magnetizace.
Naše měření ale ukazuje, že je potřeba uvážit také případnou změnu velikosti (resp. i znaménka) příslušného MO koeficientu, což celou interpretaci naměřených dat značně komplikuje.
Případně je velice vhodné provést příslušná MO měření na vlnové délce, kde je tento koeficient zcela izotropní, tj. když $P_-=0$.
Současný požadavek na co největší hodnotu $P_+$ následně určuje nejvhodnější vlnovou délku pro příslušné experimenty -- v případě FeRh ve feromagnetické fázi se jedná o vlnovou délku \SI{600}{\nano\meter} (viz obr.~\ref{fig:ferh-fm-pmld}).

\begin{figure}[htbp]
    \centering
    \includegraphics[scale=1.5,viewport=0 40 70 130,clip]{./img/svg/ferh-schema.drawio.pdf}
    \caption{Schéma aparatury pro měření v reflexní geometrii za použití kryostatu.}
    \label{fig:ferh-schema}
\end{figure}

\begin{figure}[htbp]
    \centering
    \includegraphics{./data/out/ferh-data.pdf}
    \caption{Měření FeRh v reflexní geometrii, $\lambda=\SI{1600}{\nano\meter}$, $\Hext=\SI{50}{\milli\tesla}$. (a) Hrubá změřená data. (b) Stejná data po symetrizaci v poli a filtraci frekvencí $2\beta$ (body) a proložený model (čára).}
    \label{fig:ferh-fm-schema-data}
\end{figure}

\begin{figure}[htbp]
    \centering
    \includegraphics{./data/out/ferh-pmld.pdf}
    \caption{Spektrální závislost izotropní ($P_+$) a anizotropní ($P_-$) části kvadratického MO koeficietu vzorku FeRh v reflexní geometrii při teplotě \SI{420}{\kelvin}, $\Hext=\SI{50}{\milli\tesla}$.}
    \label{fig:ferh-fm-pmld}
\end{figure}

\subsubsection*{Určení magnetické anizotropie v nepřesném poli}

V rámci našeho experimentu byl vzorek měřen s dvěma velikostmi magnetického pole $\Hext=50$, \SI{207}{\milli\tesla} a pro dva úhly in-plane rotace vzorku $\gamma\approx0$, \SI{-90}{\degree}.
Výsledná magnetická anizotropie je znázorněná na obr. \ref{fig:ferh-anizotropie}.
Souhrn pro všechny čtyři kombinace $\Hext$ a $\gamma$ je uveden v tabulce~\ref{tab:ferh-anizotropie}.
I zde musíme mít na paměti definici $\gamma$ pro vzorek FeRh, protože kubická anizotropní konstanta $k_4$ je definovaná vzhledem k FeRh[100].
Snadné osy jsou ve směru FeRh[100] (``experimentální'' směr \SI{45}{\degree}), takže $k_4>0$.
``Experimentální'' směr uniaxiální anizotropie je dán $\phiu+\gamma$.

Narážíme na zajímavý problém.
Magnetická anizotropie je dobře konzistentní pro danou měřící proceduru (posloupnost nastavovaných proudů v cívkách elektro-magnetů) a danou orientaci vzorku.
Potom jsou anizotropní konstanty určeny různými vlnovými délkami v dobré shodě (obr. \ref{fig:ferh-anizotropie} (a, b), modré body).
Problém však nastává při porovnání různých měřicích procedur či pozic vzorku.

Uvedeme několik poznatků.
Směr osy\footnote{Kvůli definici $\gamma$ se zde jedná o těžkou osu magnetizace.} kubické anizotropie vyjádřený úhlem $\gamma_\textrm{SW}$ je nezávislý na $\Hext$, ale při otočení vzorku o \SI{90}{\degree} se konzistentně změnil o $\approx\SI{4}{\degree}$\footnote{Z hlediska kubické anizotropie jsou $\gamma=\SI{0}{\degree}$ a $\gamma=\SI{90}{\degree}$ ekvivalentní.}, takže pravděpodobně reflektuje skutečné pootočení o \SI{94}{\degree}.
Velikost $k_4$ je poměrně nezávislá na $\Hext$ i $\gamma$.

Zcela jiná situace je ale u uniaxiální anizotropie.
Při rotaci vzorku $\gamma \mapsto\gamma+\SI{90}{\degree}$ by podle \eqref{eqn:tilda-k} mělo dojít ke změně znaménka $\tilde{k}_u$.
Jednoduššími slovy, osa uniaxiální anizotropie se musí otáčet zároveň se vzorkem, což se ale neděje, viz obr. \ref{fig:ferh-anizotropie} a tabulka \ref{tab:ferh-anizotropie}.
Naopak, osa uniaxiální anizotropie zůstává ve stejném ``experimentálním'' směru popsaném úhlem $\phiu+\gamma_\textrm{SW}$.
Navíc síla uniaxiální anizotropie $k_u$ je přímo úměrná použitému poli $\Hext$.

Uvedené poznatky nás vedou k závěru, že vlastní metoda je sice velice citlivá a spolehlivá, ale skutečná magnetická pole $\vHext$ realizovaná v experimentu jsou (vlivem neideálně kompenzované hystereze jednotlivých pólových nástavců cívek v elektromagnetu) mírně odlišná od nominálních hodnot použitých při analýze.
Pokud je magnetická anizotropie, jako v případě FeRh, slabá (tj. na úrovni několika \si{\milli\tesla}), tak měříme dominantně anizotropii magnetického pole (viz obr. 6.12 v práci \cite{kimakOptickaSpektroskopieAntiferomagnetu2019}) a ne vzorku.

V době odevzdání této práce jsou v přípravě dodatečná zpřesňující měření magnetického pole, které umožní z již naměřených dat určit správnou magnetickou anizotropii.
Důsledky odlišnosti reálného a nominálního magnetického pole pro analýzu magnetické anizotropie se mimo jiné věnuje dodatek \ref{app:magneticka-anizotropie}.

Nicnémě určení kubické anizotropní konstanty považujeme za spolehlivé i v případě FeRh
\begin{equation}
    k_4=\SI{5.3(4)}{\milli\tesla} \,.
\end{equation}

\begin{table}[tp]
    \centering
    \begin{tabular}{cc|cccc}
        \toprule
        $\Hext$ (\si{\milli\tesla}) & $\gamma$ (\si{\degree}) & $k_4$ (\si{\milli\tesla}) & $k_u$ (\si{\milli\tesla}) & $\phiu+\gamma_\textrm{SW}$ (\si{\degree}) & $\gamma_\textrm{SW}$ (\si{\degree}) \\ \midrule[\heavyrulewidth]
        50 & 0 & \num{5.3(4)} & \num{1.7(1)} & \num{126(4)} & \num{-1.9(5)} \\
        207 & 0 & \num{5(1)} & \num{5.7(3)} & \num{130(1)} & \num{-3.5(1)} \\ \midrule[\heavyrulewidth]
        50 & \num{-90} & \num{4.9} & \num{1.6} & \num{132} & \num{-97} \\
        207 & \num{-90} & \num{5.9} & \num{5.7} & \num{127} & \num{-97} \\
        \bottomrule 
    \end{tabular} 
    \caption{Změřená magnetická anizotropie FeRh. Úhel $\phiu+\gamma_\textrm{SW}$ značí směr osy uniaxiální anizotropie v souřadné soustavě magnetu.}
    \label{tab:ferh-anizotropie}
\end{table}


\begin{figure}[htbp]
    \centering
    \includegraphics{./data/out/ferh-spatna-anizotropie.pdf}
    \caption{Výsledky magnetických anizotropních konstant FeRh. (a, c) Uniaxiální. (b, d) Kubická. Na (a, b) jsou zobrazeny pouze střední hodnoty odhadů pro čtyři kombinace různých velikostí pole a rotace vzorku, čtverci jsou naznačeny oblasti zobrazené v (c, d). (c, d) Zaměření na $\Hext=\SI{50}{\milli\tesla}$, $\gamma=\SI{0}{\degree}$, zobrazeny $2\sigma$ oblasti měřené jednotlivými vlnovými délkami.}
    \label{fig:ferh-anizotropie}
\end{figure}
