\section{Měření v reflexní geometrii: FeRh}
\label{chap:vysledky-ferh}

\subsection{Měření feromagnetické fáze}
\label{chap:ferh-fm}

Měření proběhlo při teplotě \SI{420}{\kelvin} v reflexní geometrii, schéma je na obr. \ref{fig:ferh-fm-schema-data} (a).
Úhel dopadu byl \SI{0.5}{\degree}.
Orientace vzorku byla vždy přibližně $\gamma\approx\SI{0}{\degree}$ (viz obr. \ref{fig:vzorek-ferh} (a)), kromě jednoho kontrolního měření při $\lambda=\SI{1600}{\nano\meter}$, ve kterém bylo $\gamma\approx\SI{-90}{\degree}$.
Příklad změřených dat je na obr. \ref{fig:ferh-fm-schema-data} (b).
Všechny vlnové délky byly změřeny s velikostí pole $\Hext=\SI{50}{\milli\tesla}$, některé navíc s $\Hext=\SI{207}{\milli\tesla}$.

Výsledná magnetická anizotropie je znázorněná na obr. \ref{fig:ferh-anizotropie}.
Z výsledků je patrný již dvakrát zmíněný problém: uniaxiální magnetická anizotropie není konzistentní napříč velikostmi $\Hext$ a natočeními vzorku $\gamma$.
$\tilde{k}_u$ by mělo při rotaci vzorku o \SI{90}{\degree} podle \eqref{eqn:tilda-k} změnit znaménko, což se neděje.
Naopak je na rotaci vzorku nezávislé a je úměrné velikosti pole $\Hext$.
To naznačuje, že změřené $\tilde{k}_u$ pochází od magnetu, a uniaxiální anizotropie samotného vzorku je proti ní velice malé.

Kubická $\tilde{k}_4$ by měla podle \eqref{eqn:tilda-k} zůstat nezměněna při rotaci o \SI{90}{\degree}.
Při malé nepřesnosti natočení se ale otočí v komplexní rovině o čtyřnásobek dodatečného úhlu.
Podle obr. \ref{fig:ferh-anizotropie} (b) by to odpovídalo dodatečné rotaci o $\approx \SI{5}{\degree}$, což je přesvědčivé.
Kubickou anizotropní konstantu tedy považujeme za věrohodně určenou, s hodnotou\todo{overit chybu}
\begin{equation}
    k_4 (\textrm{FR06}) = \SI{5.2(2)}{\milli\tesla} \,.
\end{equation}

Určení MLD koeficientů $P$ je necitlivé na nepřesnosti v $\vHext$, spektrální závislost je vynesena na obr. \ref{fig:ferh-fm-pmld}.

\begin{figure}[htbp]
    \centering
    \begin{subfigure}{.3\textwidth}
        \centering
        \includegraphics{./img/svg/ferh-schema.drawio.pdf}
    \end{subfigure}
    \begin{subfigure}{.65\textwidth}
        \includegraphics{./data/out/ferh-data.pdf}
    \end{subfigure}
    \caption{Měření FeRh v reflexní geometrii. (a) Schéma. (b) Ilustrace změřených dat s $\lambda=\SI{1600}{\nano\meter}$ a $\Hext=\SI{50}{\milli\tesla}$, po symetrizaci v poli a polohách WP2.}
    \label{fig:ferh-fm-schema-data}
\end{figure}

\begin{figure}[htbp]
    \centering
    \includegraphics{./data/out/ferh-spatna-anizotropie.pdf}
    \caption{Výsledky magnetických anizotropních konstant FeRh. (a, c) Uniaxiální. (b, d) Kubická. Na (a, b) jsou zobrazeny pouze střední hodnoty odhadů pro čtyři kombinace různých velikostí pole a rotace vzorku, čtverci jsou naznačeny oblasti zobrazené v (c, d). (c, d) Zaměření na $\Hext=\SI{50}{\milli\tesla}$, $\gamma=\SI{0}{\degree}$, zobrazeny $2\sigma$ oblasti měřené jednotlivými vlnovými délkami.}
    \label{fig:ferh-anizotropie}
\end{figure}

\begin{figure}[htbp]
    \centering
    \missingfigure{}
    \caption{Spektrální závislost reflexních MO koeficientů $P$ vzorku FeRh při teplotě \SI{420}{\kelvin}.}
    \label{fig:ferh-fm-pmld}
\end{figure}

