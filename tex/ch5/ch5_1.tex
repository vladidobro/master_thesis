\section{Měření v transmisní geometrii: CoFe}
\label{chap:vysledky-cofe}

\subsection{Měření při pokojové teplotě}
\label{chap:vysledky-cofe-roomt}

Měření v transmisní geometrii při pokojové teplotě je nejjednodušší, protože nevyžaduje žádných zrcadel mezi vzorkem a můstkem.
Aparatura byla použita ve stejném stavu jako je znázorněna schématem na obr. \ref{fig:mustek-desticka-ilustrace} (b).
Ilustrace změřených dat je na obr. \ref{fig:vysledky-cofe-schema-data}.
Výsledky fitu MO koeficientů $P$ jsou na obr. \ref{fig:vysledky-cofe-PMLD}.

Znázornění výsledků fitu magnetické anizotropie je na obr. \ref{fig:vysledky-cofe-roomt-anizotropie}.
Výsledky určení magnetických anizotropních konstant pomocí různých vlnových délek se velice dobře shodují. 
Z obrázků je také zřejmé, že nejistota konstant je pouze mírně podhodnocena.

\begin{figure}[htbp]
    \centering
    \includegraphics{./data/out/cofert-data.pdf}
    \caption{CoFe v transmisní geometrii při pokojové teplotě. (a) Ilustrace měřených dat ($\lambda=\SI{620}{\nano\meter}$). (b) Stejná data po symetrizaci v poli a filtraci frekvencí $2\beta$.}
    \label{fig:vysledky-cofe-schema-data}
\end{figure}

\begin{figure}[htbp]
    \centering
    \includegraphics{./data/out/cofe-anizotropie.pdf}
    \caption{Výsledky magnetických anizotropních konstant CoFe při pokojové teplotě. (a) Uniaxiální. (b) Kubická. Zobrazeny jsou $2\sigma$ oblasti měřené jednotlivými vlnovými délkami.}
    \label{fig:vysledky-cofe-roomt-anizotropie}
\end{figure}

\subsubsection{Nepřesnosti magnetického pole}

Z experimentálních výsledků uvedených dále v kapitole \ref{chap:5} vychází, že určení magnetické anizotropie uvedenou metodou je velice spolehlivé.
Konkrétně, s danou měřicí procedurou (tj. nastavovanou posloupností proudů v cívkách elektro-magnetů) a neměnnou pozicí vzorku jsou téměř v dokonalé shodě anizotropní konstanty určené různými vlnovými délkami, jak by se dalo očekávat.
Vzhledem k rozmanité spektrální závislosti MO koeficientů (viz výsledky v kap. \ref{chap:5}) to dokazuje, že měřený signál skutečně pochází od vzorku a je dobře popsán modelem \eqref{eqn:anizotropie-fit}.
Problém však někdy nastává při porovnání různých měřicích procedur či pozic vzorku.
Tato nepříjemnost byla pozorována se vzorkem FeRh (viz oddíl \ref{chap:ferh-fm}), kde je magnetická anizotropie velice slabá.

Tento vzorek byl měřen se dvěma velikostmi $\Hext = 50$, \SI{207}{\milli\tesla}.
V obou případech byla naměřená uniaxiální anizotropie ve stejném směru, avšak s jinou velikostí.
Zachoval se však poměr $k_u/\Hext$, což napovědělo, že problém pochází od magnetu.
Měření bylo následně zopakováno se vzorkem in-plane otočeným o \SI{90}{\degree} -- výsledkem byla uniaxiální anizotropie ve stejném \emph{experimentálním} směru, tzn. o \SI{-90}{\degree} otočená v soustavě spojené se vzorkem, což samozřejmě nedává fyzikální smysl.

Pravděpodobnou příčinou tohoto problému je, že vlastní metoda je sice velice citlivá a spolehlivá, ale skutečná magnetická pole $\vHext$ realizovaná v experimentu jsou (vlivem neideálně kompenzované hystereze jednotlivých pólových nástavců cívek v elektromagnetu) mírně odlišná od nominálních hodnot použitých při analýze.
V době odevzdání této práce jsou v přípravě dodatečná zpřesňující měření magnetického pole, které umožní z již naměřených dat určit správnou magnetickou anizotropii.

Důsledky odlišnosti reálného a domnělého magnetického pole pro analýzu magnetické anizotropie se mimo jiné věnuje dodatek \ref{app:magneticka-anizotropie}.

\subsection{Měření při kryogenní teplotě}
\label{chap:vysledky-cofe-lowt}

Měření proběhlo při nejnižší teplotě dosažitelné kryostatem, tj. přibližně \SI{15}{\kelvin}.
Měření v transmisní geometrii je při zchlazeném kryostatu výrazně složitější, viz obr. \ref{fig:vysledky-cofe-lowt-schema-data} (a).
První ze zkřížených zrcadel je umístěno uvnitř komory kryostatu, druhé -- kompenzační -- je venku.
Není proto apriori zřejmé, zda se zrcadla skutečně kompenzují, když jsou udržovány na tak rozdílných teplotách.
Překvapivě tomu tak ale je: i při zběžném nastavení zrcadel byla po obou odrazech naměřena v celém spektru elipticita $<\SI{1}{\degree}$ pro vstupní polarizaci $\beta=45$, \SI{135}{\degree} (rovnoměrný mix s- a p-).

Druhý problém činí okénka komory kryostatu, na rozdíl od reflexní geometrie totiž světlo prochází postranním okénkem, ve kterém se indukuje Faradayův jev.
Důsledkem je pak měřené stočení téměř nezávislé na $\beta$, viz obr. \ref{fig:vysledky-cofe-lowt-schema-data} (b).
Provedli jsme pokus o oddělení Faradayova jevu od MLD na základě jejich symetrie.
Faradayův jev je lichý v magnetizaci (okénka) a je nezávislý na natočení vstupní lineární polarizace\footnote{V aparatuře se okénko vyskytuje až za prvním zrcadlem, takže není striktně pravda, že by stočení okénkem bylo nezávislé na $\beta$. Je nezávislé na natočení lineární polarizace vstupující do okénka.}.
Stočení po symetrizaci ve vnějším poli a vybrání $\beta$-závislosti s frekvencí $2\beta$ je na obr. \ref{fig:vysledky-cofe-lowt-schema-data} (c).
Signál je posléze fitovaný modelem popsaným v kap. \ref{chap:4} a poměrně překvapivě dává přibližně stejnou magnetickou anizotropii pro většinu vlnových délek, viz obr. \ref{fig:vysledky-cofe-lowt-schema-data} (d).
Právě kvůli Faradayově jevu v okénkách považujeme transmisní geometrii při nízké teplotě za nejméně spolehlivou.
Obecně je amplituda vysoká ve viditelné oblasti, zatímco v infračervené jsou data téměř neovlivněna.

Spektrální závislost výsledných MO koeficientů je vynesena na obr. \ref{fig:vysledky-cofe-PMLD}, společně s těmi měřenými při pokojové teplotě.
Výsledná magnetická anizotropie je na obr. \ref{fig:vysledky-cofe-lowt-anizotropie}.
Shrnutí magnetických anizotropních konstant pro obě měřené teploty je v tabulce \ref{tab:cofe-anizotropie}.

\begin{figure}[htbp]
    \centering
    \begin{subfigure}{.3\textwidth}
        \centering
        \includegraphics{./img/svg/cofe-lt-schema.drawio.pdf}
    \end{subfigure}
    \begin{subfigure}{.67\textwidth}
        \centering
        \includegraphics{./data/out/cofelt-data.pdf}
    \end{subfigure}
    \caption{Měření CoFe v transmisní geometrii při snížené teplotě. (a) Schéma. (b) Ilustrace změřených dat s $\lambda=\SI{620}{\nano\meter}$ a $\Hext=\SI{207}{\milli\tesla}$. (c) Stejná data po symetrizaci v poli a filtraci frekvencí $2\beta$.}
    \label{fig:vysledky-cofe-lowt-schema-data}
\end{figure}

\begin{figure}[htbp]
    \centering
    \includegraphics{./data/out/cofe-lowt-anizotropie.pdf}
    \caption{Výsledky magnetických anizotropních konstant CoFe při \SI{15}{\kelvin}. (a) Uniaxiální. (b) Kubická. Zobrazeny jsou $2\sigma$ oblasti měřené jednotlivými vlnovými délkami.}
    \label{fig:vysledky-cofe-lowt-anizotropie}
\end{figure}

\begin{figure}[htbp]
    \centering
    %\includegraphics{./data/out/cofe-pmld.pdf}
    \caption{PMLD cofe roomt, lowt}
    \label{fig:vysledky-cofe-PMLD}
\end{figure}

\begin{table}[tp]
    \caption{Magnetická anizotropie CoFe.}
    \label{tab:cofe-anizotropie}
    \centering
    \begin{tabular}{cccc}
        \toprule
        $T$ (\si{\kelvin}) & $k_4$ (\si{\milli\tesla}) & $k_u$ (\si{\milli\tesla}) & $\phiu$ (\si{\degree}) \\ \midrule[\heavyrulewidth]
        295 & 0 & 0 & 0 \\
        15 & 0 & 0 & 0 \\
        \bottomrule 
    \end{tabular} 
\end{table}
