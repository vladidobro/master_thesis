\section{Měření v transmisní geometrii: CoFe}
\label{chap:vysledky-cofe}

\subsection{Měření při pokojové teplotě}
\label{chap:vysledky-cofe-roomt}

Měření v transmisní geometrii při pokojové teplotě je ze všech testovaných verzí metody nejjednodušší, protože nevyžaduje žádných zrcadel mezi vzorkem a můstkem.
Aparatura byla použita ve stejném stavu jako je znázorněna schématem na obr. \ref{fig:mustek-desticka-ilustrace} (b).
Velikost pole byla pro vzorek CoFe vždy použita $\Hext=\SI{207}{\milli\tesla}$.

Ilustrace změřených dat je na obr. \ref{fig:vysledky-cofe-data} (a).
Z obrázku jsou patrné oba dva rysy MLD, tj. změna znaménka při otočení $\beta \mapsto \beta+\SI{90}{\degree}$ a sudost v magnetickém poli $\Delta\beta(\phih)=\Delta\beta(\phih+\SI{180}{\degree})$ (lineární jevy jsou zanedbatelné).
Po provedení standardní procedury předzpracování dat (obr. \ref{fig:vysledky-cofe-data} (b)) je signál téměř identický.

Takto předzpracovaná data následně fitujeme modelem popsaným v oddílech \ref{chap:anizotropie-MLD}, \ref{chap:urceni-magneticke-anizotropie} a \ref{chap:zpracovani-dat}, tj. souběžně určímě MO koeficienty $P_{ij}$ a magnetické anizotropní konstanty.
Na obr. \ref{fig:vysledky-cofe-data} (b) jsou vykreslené naměřené body společně s nafitovaným modelem.

Takto byly určeny magnetické anizotropní konstanty pro všechny měřené vlnové délky.
$2\sigma$ oblasti komplexních konstant $\tilde{k}_{u/4}$ jsou vykresleny na obr. \ref{fig:vysledky-cofe-roomt-anizotropie}.
Z obrázků je patrné, že se určená anizotropie velice dobře shoduje pro většinu vlnových délek (jen modrý konec viditelného spektra nemá dobré výsledky, z neznámých důvodů je zde přítomný výrazně vyšší šum).
Zároveň je vidět, že nejistota určená způsobem popsaným v dodatku \ref{app:zpracovani} je pouze velmi mírně podhodnocena (elipsy se téměř překrývají).

V podobě reálných konstant $k_4$, $k_u$ a $\phiu$ jsou výsledky zaneseny do tabulky \ref{tab:cofe-anizotropie}.
Velikosti konstant $k_4$ a $k_u$ se dobře shodují s těmi určenými autory článku \cite{zengIntrinsicMechanismAnisotropic2020} metodou torque-metry (viz oddíl \ref{chap:vzorek-cofe}).
Co se týče směru uniaxiální anizotropie $\phiu$, autoři v korespondenci připustili, že naše výsledky se shodují s jedním kandidátem na definici $\phiu$, která mohla být při měření torque-metry použita.
S touto pravděpodobnou definicí uvádíme výsledky torque-metrie také do tabulky \ref{tab:cofe-anizotropie} a příslušné funkcionály magnetické volné energie jsou vykresleny na obr.~\ref{fig:cofe-funkcional}.

Konzistenci určené magnetické anizotropie v celém studovaném spektrálním rozsahu a zároveň s výsledky torque-metry\footnote{Naše výsledky souhlasí s výsledky torque-metry jen přibližně. Důvod, proč nás to zatím nenutí k obavám, je vysvětlen v oddíle \ref{chap:ferh-fm}.} považujeme za nesporný důkaz funkčnosti metody.
Vzhledem k rozmanité spektrální závislosti MO koeficientů $P$ (viz obr.~\ref{fig:vysledky-cofe-PMLD}) je velice nepravděpodobná představa, že by měřený signál v sobě zahrnoval podstatnou část nějakého parazitního jevu, který by výsledky zkresloval.
Dodatečný signál by totiž musel mít podobnou spektrální závislost.

Spektrální závislost určených MO koeficientů $P$ je vynesena na obr.~\ref{fig:vysledky-cofe-PMLD}.
Nejistota $P$ určená standardním způsobem je však zřejmě nesmyslně nízká ($<\SI{2}{\micro\radian}$), což je pravděpodobně způsobeno přizpůsobením dat modelu (symetrizace v poli a filtace frekvencí $2\beta$).
Proto vizuální analýzou odhadujeme $2\sigma$ nejistotu jako hodnotou \SI{20}{\micro\radian}.

Ve spektrální oblasti \num{1000}--\SI{1450}{\nano\meter} dominuje izotropní člen $P_+$, pro \num{460}--\SI{1000}{\nano\meter} jsou naopak $P_+$ a $P_-$ srovnatelné a oba protnou nulu.
Pro kvantifikaci míry anizotropie MLD můžeme v zásadě postupovat dvěma způsoby.
Buďto jako poměr $P_-$, které je dovoleno symetrií pouze u anizotropních vzorků, vůči izotropní části $P_+$, nebo pragmaticky jako poměr amplitud při měření se vstupní polarizací ve význačných směrech $\beta=\gamma$ a $\beta=\gamma+\SI{45}{\degree}$, tj. pomocí koeficientů $P_\gamma$ a $P_{\gamma+\SI{45}{\degree}}$ z \eqref{eqn:Pgamma}.
Ve všech případech dosahuje vzorek CoFe v měřeném spektrálním rozsahu maximální možné anizotropie MLD, tzn. existují vlnové délky splňující $P_+=0$, $P_\gamma=0$ i $P_{\gamma+\SI{45}{\degree}}=0$.
U vzorku byla změřena vysoká anizotropie AMR\cite{zengIntrinsicMechanismAnisotropic2020}, takže není překvapivé, že vykazuje i anizotropii MLD.


\begin{figure}[htbp]
    \centering
    \includegraphics{./data/out/cofert-data.pdf}
    \caption{CoFe v transmisní geometrii při pokojové teplotě, $\lambda=\SI{620}{\nano\meter}$. (a) Hrubá měřená data. (b) Stejná data po symetrizaci v poli a filtraci frekvencí $2\beta$ (body) a proložený model (čára).}
    \label{fig:vysledky-cofe-data}
\end{figure}

\begin{figure}[htbp]
    \centering
    \includegraphics{./data/out/cofe-anizotropie.pdf}
    \caption{Výsledky magnetických anizotropních konstant CoFe při pokojové teplotě. (a) Uniaxiální. (b) Kubická. Zobrazeny jsou $2\sigma$ oblasti měřené jednotlivými vlnovými délkami.}
    \label{fig:vysledky-cofe-roomt-anizotropie}
\end{figure}

\begin{table}[tp]
    \centering
    \begin{tabular}{ccccc}
        \toprule
        $\Lambda$ (\si{\nano\meter}) & $k_4$ (\si{\milli\tesla}) & $k_u$ (\si{\milli\tesla}) & $\phiu$ (\si{\degree}) & $\gamma_\textrm{SW}$ (\si{\degree}) \\ \midrule[\heavyrulewidth]
        \num{460} & \num{-34.6} & \num{3.9} & \num{128.2} & \num{-3.7} \\
\num{530} & \num{-31.6} & \num{8.7} & \num{154.6} & \num{-3.3} \\
\num{620} & \num{-32.7} & \num{8.7} & \num{158.1} & \num{-3.1} \\
\num{710} & \num{-32.9} & \num{8.6} & \num{158.2} & \num{-3.1} \\
\num{810} & \num{-32.6} & \num{9.0} & \num{157.9} & \num{-3.1} \\
\num{920} & \num{-32.7} & \num{9.4} & \num{157.9} & \num{-3.0} \\
\num{1050} & \num{-32.1} & \num{8.3} & \num{158.1} & \num{-3.3} \\
\num{1200} & \num{-32.5} & \num{7.7} & \num{158.4} & \num{-3.2} \\
\num{1450} & \num{-32.5} & \num{9.1} & \num{158.4} & \num{-3.2} \\ \midrule[\heavyrulewidth]
        \SI{295}{\kelvin} & \num{-32.6(2)} & \num{8.7(5)} & \num{158.1(5)} & \num{-3.1(1)} \\
        \SI{15}{\kelvin} & \num{-37.0(5)} & \num{10.8(6)} & \num{162.0(13)} & \num{-1.7(3)} \\
        torque-metry & \num{-30.25} & \num{6.3} & \num{156} & --- \\
        \bottomrule 
    \end{tabular} 
    \caption{Magnetická anizotropie CoFe. Uvedená spektrální závislost je pro pokojovou teplotu. Směr uniaxiální anizotropie $\phiu$ je odečítán od směru CoFe[100] daném průměrným $\gamma_\textrm{SW}$ (tedy ne $\gamma_\textrm{SW}$ určeným konkrétní vlnovou délkou).}
    \label{tab:cofe-anizotropie}
\end{table}

\begin{figure}[htbp]
    \centering
    \includegraphics{./data/out/cofe-funkcional.pdf}
    \caption{Funkcionál magnetické volné energie vzorku CoFe při pokojové teplotě určen naší metodou a metodou torque-metry provedenou autory článku \cite{zengIntrinsicMechanismAnisotropic2020}.}
    \label{fig:cofe-funkcional}
\end{figure}

\begin{figure}[htbp]
    \centering
    \includegraphics{./data/out/cofe-pmld.pdf}
    \caption{Určené magneto-optické koeficienty $P$ vzorku CoFe v transmisní geometrii při pokojové a snížené teplotě.}
    \label{fig:vysledky-cofe-PMLD}
\end{figure}



\subsection{Měření při kryogenní teplotě}
\label{chap:vysledky-cofe-lowt}

Po úspěchu při pokojové teplotě jsme přistoupili k měření s kryostatem, který je jednou z hlavních předností naší metody oproti příbuzným dostupným metodám.
Měření proběhlo při nejnižší teplotě dosažitelné kryostatem, tj. přibližně \SI{15}{\kelvin}, v transmisní geometrii podle schématu na obr.~\ref{fig:vysledky-cofe-lowt-schema}.

Oproti měření při pokojové teplotě se při zchlazeném kryostatu vyskytuje podstatně více ``artefaktů''.
Pro vyvedení svazku z komory kryostatu je nutné použití zrcadla, které je nutné kompenzovat druhým zrcadlem (oddíl \ref{chap:kompenzace}).
První ze zkřížených zrcadel je umístěno uvnitř komory kryostatu, druhé -- kompenzační -- je venku.
Není proto apriori zřejmé, zda se zrcadla skutečně kompenzují, když jsou udržovány na tak rozdílných teplotách.
Překvapivě tomu tak ale je: i při zběžném nastavení zrcadel byla po obou odrazech naměřena v celém spektru elipticita $<\SI{1}{\degree}$ pro vstupní polarizaci $\beta=\SI{45}{\degree}$, \SI{135}{\degree} (rovnoměrný mix s- a p-).

Druhá nepřijemnost je způsobena okénkem komory kryostatu, světlo totiž prochází i postranním okénkem, ve kterém má magnetické pole pro jistá $\phih$ polární směr, viz obr. \ref{fig:vysledky-cofe-lowt-schema} (b).
V měřeném signálu pak pozorujeme příspěvek Faradayovy rotace v okénku téměř nezávislý na vstupní polarizaci $\beta$, viz obr. \ref{fig:vysledky-cofe-lowt-data} (a).

Provedli jsme pokus o oddělení Faradayova jevu od MLD na základě jejich symetrie.
Faradayův jev je lichý v magnetizaci (okénka) a je nezávislý na natočení vstupní lineární polarizace\footnote{V aparatuře se okénko vyskytuje až za prvním zrcadlem, takže není striktně pravda, že by stočení okénkem bylo nezávislé na $\beta$. Je nezávislé na natočení lineární polarizace vstupující do okénka.}.
Standardním postupem bychom tedy měli být schopni oba příspěvky oddělit, stočení po symetrizaci ve vnějším poli a filtraci frekvencí $2\beta$ je na obr. \ref{fig:vysledky-cofe-lowt-data} (b).
Výsledný signál je již vizuálně velmi podobný tomu změřenému při pokojové teplotě (obr. \ref{fig:vysledky-cofe-data} (b)).

Dále postupujeme stejně jako v případě měření při pokojové teplotě.
Výsledné $2\sigma$ oblasti spolehlivosti magnetických anizotropních konstant jsou vykresleny na obr. \ref{fig:vysledky-cofe-lowt-anizotropie}.
Jejich shrnutí je lze nalézt v tabulce \ref{tab:cofe-anizotropie}.
I zde se magnetické anizotropní konstanty určené pomocí různých vlnových délek dobře shodují, což nám dodává jistotu, že uvedený způsob odstranění Faradayovy rotace okének kryostatu je spolehlivý.
Spektrální závislost výsledných MO koeficientů $P$ je vynesena na obr.~\ref{fig:vysledky-cofe-PMLD} společně s těmi měřenými při pokojové teplotě.

Ani magnetická anizotropie ani spektrální závislost MO koeficientů téměř nezávisí na teplotě, což je pochopitelné pro materiál s Curieovou teplotou $T_C\approx \SI{950}{\kelvin}$\cite{sundarSoftMagneticFeCo2005}.


\begin{figure}[htbp]
    \centering
    \includegraphics{./img/svg/cofe-lt-schema.drawio.pdf}
    \caption{Schéma aparatury pro měření v transmisní geometrii za použití kryostatu. (a) Půdorys. (b) Bokorys, detail na komoru kryostatu.
    Světlo prochází horním okénkem, ve kterém se vlivem polárního magnetického pole indukuje Faradayův jev.}
    \label{fig:vysledky-cofe-lowt-schema}
\end{figure}

\begin{figure}[htbp]
    \centering
        \includegraphics{./data/out/cofelt-data.pdf}
        \caption{CoFe v transmisní geometrii při teplotě \SI{15}{\kelvin}, $\lambda=\SI{620}{\nano\meter}$. (a) Hrubá měřená data. (b) Stejná data po symetrizaci v poli a filtraci frekvencí $2\beta$ (body) a proložený model (čára).}
    \label{fig:vysledky-cofe-lowt-data}
\end{figure}

\begin{figure}[htbp]
    \centering
    \includegraphics{./data/out/cofe-lowt-anizotropie.pdf}
    \caption{Výsledky magnetických anizotropních konstant CoFe při \SI{15}{\kelvin}. (a) Uniaxiální. (b) Kubická. Zobrazeny jsou $2\sigma$ oblasti měřené jednotlivými vlnovými délkami.}
    \label{fig:vysledky-cofe-lowt-anizotropie}
\end{figure}


