\section{Optika v anizotropních multivrstvách \cite{Berreman}}

Cílem tohoto oddílu je představit teorii výpočtu transmisních a reflexních koeficientů (Jonesových matic průchodu a odrazu) obecných vrstevnatých struktur popsaných stupňovitým profilem tenzoru $\e$.
Vrstevnatou strukturou rozumíme takovou, která je homogenní v rovině kolmé na jednu osu, tu zvolíme jako $z$, a rovina homogenity a všech rozhraní bude $xy$, viz obr. \ref{f:vrstevnate prostredi}.
Stupňovitým profilem zase rozumíme, že struktura je složená z konečného počtu vrstev, ve kterých je $\e$ konstantní.

\begin{figure}\centering
\includegraphics[width=0.3\linewidth]{./img/t4.png}
\caption{Vrstevnaté prostředí}\label{f:vrstevnate prostredi}
\end{figure}

Budou nás zajímat řešení Maxwellových rovnic \eqref{e:rotE} a \eqref{e:rotB}.
Vzhledem k homogenitě v rovině $xy$ lze psát
\begin{equation} \label{e:kxy}
\E(x,y,z)=\E(z) e^{i(k_xx+k_yy)} \,, \qquad c\B(x,y,z)=c\B(z) e^{i(k_xx+k_yy)} \,,
\end{equation}
kde $k_x$ a $k_y$ jsou konstantní podél celé struktury i mimo ní.
Uvnitř každé vrstvy bude existovat i třetí složka $k_z$, bude se ale lišit v různých vrstvách.
Obecný postup je tedy řešit Maxwellovy rovnice zvlášť v každém prostředí, a řešení následně svázat pomocí okrajových podmínek, které říkají, že na rozhraních vrstev jsou tečné složky $E_x, E_y, cB_x, cB_y$ spojité.

Budeme následovat řešení Berremanovou maticovou metodou\cite{Berreman}.
Rovnice \eqref{e:NE} a \eqref{e:NB} mají v maticovém zápisu tvar
\begin{equation}
\begin{pmatrix}
0 & -N_z & N_y & -1 & 0 & 0 \\
N_z & 0 & -N_x & 0 & -1 & 0 \\
-N_y & N_x & 0 & 0 & 0 & -1 \\
\varepsilon_{11} & \varepsilon_{12} & \varepsilon_{13} & 0 & -N_z & N_y \\
\varepsilon_{21} & \varepsilon_{22} & \varepsilon_{23} & N_z & 0 & -N_x \\
\varepsilon_{31} & \varepsilon_{32} & \varepsilon_{33} & -N_y & N_x & 0
\end{pmatrix}
\begin{pmatrix}
E_x \\ E_y \\ E_z \\ cB_x \\ cB_y \\ cB_z
\end{pmatrix} = \begin{pmatrix}
0 \\ 0 \\ 0 \\ 0 \\ 0 \\ 0
\end{pmatrix} \,.
\end{equation}
Jedinou neznámou je zde $N_z$. Matice uměrná $N_z$ nemá plnou hodnost, dvě rovnice jsou lineárně závislé, a proto je můžeme vyřešit a dosadit do ostatních.
Zvolíme pro tento účel třetí a šestou rovnici (pro $E_z$ a $cB_z$), které neobsahují $N_z$ a zároveň se nezachovávají na rozhraní.
Nejdříve ale zavedeme úsporný blokově maticový zápis
\begin{align} \label{e:Berreman 6x6}
E^\perp = \begin{pmatrix}E_x \\ E_y\end{pmatrix}
\,, \, cB^\perp = \begin{pmatrix}cB_x \\ cB_y\end{pmatrix}
\,, \qquad N^\vert = \begin{pmatrix} N_y \\ -N_x \end{pmatrix}
\,, \, N^- = \begin{pmatrix} -N_y & N_x\end{pmatrix}
\,, \\ \varepsilon^\perp=\begin{pmatrix}\varepsilon_{11} & \varepsilon_{12} \\ \varepsilon_{21} & \varepsilon_{22}\end{pmatrix}
\,, \, \varepsilon^\vert=\begin{pmatrix} \varepsilon_{13} \\ \varepsilon_{23} \end{pmatrix}
\,, \, \varepsilon^-=\begin{pmatrix} \varepsilon_{31} & \varepsilon_{32} \end{pmatrix}
\,, \qquad \rho = \begin{pmatrix}0 & -1 \\ 1 & 0\end{pmatrix}
\,.
\end{align}
Třetí a šestá rovnice mají tvar
\begin{equation}
N^- E^\perp - cB_z=0 \,, \qquad \varepsilon^- E^\perp + \varepsilon_{33} E_z + N^- cB^\perp=0 \,,
\end{equation}
vyřešením a dosazením do \eqref{e:Berreman 6x6} dostáváme\footnote{Každý blok je $2\times 2$, 1 zde značí jednotkovou $2\times 2$ matici.}
\begin{equation}
\begin{pmatrix}
N_z \rho -\frac{N^\vert \varepsilon^-}{\varepsilon_{33}} & -1 - \frac{N^\vert N^- }{\varepsilon_{33}} \\
\varepsilon^\perp-\frac{\varepsilon^\vert \varepsilon^-}{\varepsilon_{33}}+N^\vert N^- & N_z \rho - \frac{\varepsilon^\vert N^-}{\varepsilon_{33}}
\end{pmatrix}
\begin{pmatrix} E^\perp \\ cB^\perp \end{pmatrix} = 0 \,,
\end{equation}
což lze přeformulovat jako vlastní úlohu
\begin{equation} \label{e:tvar N_z}
\begin{pmatrix}
\rho & 0 \\ 0 & \rho
\end{pmatrix}
\begin{pmatrix}
-\frac{N^\vert \varepsilon^-}{\varepsilon_{33}} & -1 - \frac{N^\vert N^- }{\varepsilon_{33}} \\
\varepsilon^\perp-\frac{\varepsilon^\vert \varepsilon^-}{\varepsilon_{33}}+N^\vert N^- & - \frac{\varepsilon^\vert N^-}{\varepsilon_{33}}
\end{pmatrix}
\begin{pmatrix} E^\perp \\ cB^\perp \end{pmatrix}
=N_z \begin{pmatrix} E^\perp \\ cB^\perp \end{pmatrix}
\end{equation}
Uvnitř dané vrstvy s konstantním $\e$ je závislost polí na souřadnici $z$ daná pomocí maticové exponenciály této $4\times 4$ matice
\begin{equation} \label{e:N_z exponenciala}
\begin{pmatrix} E^\perp(z_2) \\ cB^\perp(z_2) \end{pmatrix} = e^{ik_z(z_2-z_1)} \begin{pmatrix} E^\perp(z_1) \\ cB^\perp(z_1) \end{pmatrix} =
e^{iN_z\frac{\w(z_2-z_1)}{c}} \begin{pmatrix} E^\perp(z_1) \\ cB^\perp(z_1) \end{pmatrix}
\end{equation}
pro libovolné $z_1$, $z_2$ uvnitř vrstvy.
Složky polí $E^\perp(z_2)$ a $cB^\perp(z_2)$ jsou právě tečné složky, které jsou na rozhraní spojité, takže po vypočtení maticové exponenciály \eqref{e:N_z exponenciala} pro každou vrstvu dostaneme maticový vztah polí na začátku a na konci multivrstvy.

Matice \eqref{e:tvar N_z} má obecně 4 komplexní vlastní čísla a lze ji diagonalizovat pomocí vlastních vektorů\footnote{Matice není normální a proto vlastní vektory nejsou navzájem kolmé. Striktně vzato nemusí být matice diagonalizovatelná, praktické potíže to však nečiní.}, které lze rozdělit do dvou skupin s fyzikálním významem, že jde o dva módy šířící se ve směru $+z$ a dva v $-z$. Pro dva z nich platí\footnote{Třetí složka Poyntingova vektoru je dána právě tečnými složkami $S_z\propto E_xH_y-E_yH_x$} $\Re\lbrace N_z \rbrace\geq0$, $\Im\lbrace N_z \rbrace\geq0$, $S_z\geq0$ a šíří se ve směru $+z$, zbylé dva se šíří ve směru $-z$ a platí pro ně opačné nerovnice.

\subsection*{Odraz a průchod}

Předpokládejme, že se nám díky \eqref{e:N_z exponenciala} podařilo najít matici $M$, která svazuje příčná pole na začátku ($l$) a na konci $(r)$ multivrstvy, viz obr. \ref{f:odraz a pruchod} (a)
\begin{equation}
\begin{pmatrix} E^\perp_l \\ cB^\perp_l \end{pmatrix}=M \begin{pmatrix} E^\perp_r \\ cB^\perp_r \end{pmatrix}
\end{equation}

\begin{figure}
\centering
\begin{minipage}{.5\textwidth}
  \centering
  \includegraphics[width=\linewidth]{./img/t5a.png}
\end{minipage}%
\begin{minipage}{.5\textwidth}
  \centering
  \includegraphics[width=\linewidth]{./img/t5b.png}
\end{minipage}
\caption{Odraz a průchod}\label{f:odraz a pruchod}
\end{figure}

Úloha průchodu a odrazu je typicky zadána tak, že na strukturu posvítíme svazkem s definovaným $\N$, a nikdo jiný z žádné strany nesvítí.
Zvolíme si, že svítíme z levé strany (takže se záporným $N_z$).
Druhá podmínka znamená, že amplitudy všech zbylých módů, které přinášejí energii směrem ke struktuře, jsou nulové; to jsou dva módy na pravé straně s kladným $k_z$.
Na každé straně nám tedy zbývají dva módy, jejichž amplitudy zvýbá určit.

Vstupní a výstupní prostředí je nutné v tomto kontextu definovat jako to první, u kterého již nelze zpětné odrazy považovat za koherentní s dopadajícím svazkem (např. kvůli prostorovému oddělení).
Pokud zkoumáme odraz od vzorku s \SI{1}{\meter} tlustou nadvrstvou skla, musíme jako vstupní prostředí považovat právě tuto vrstvu skla, ikdyž je náš laser ve skutečnosti umístěn ve vzduchu ještě před sklem.
Podobně musíme činit i pokud je tlustá vrstva skla z druhé strany.

Na obou stranách si zvolíme pro $-z$ i $+z$ šířící se módy bázi lineárních polarizací, kterými definujeme Jonesovi vektory pro příslušné svazky
\begin{equation}
\begin{pmatrix} E^\perp_l \\ cB^\perp_l \end{pmatrix}=D_l \begin{pmatrix} J^-_l \\ J^+_l \end{pmatrix} \,, \,
\begin{pmatrix} E^\perp_r \\ cB^\perp_r \end{pmatrix}=D_r \begin{pmatrix} J^-_r \\ J^+_r \end{pmatrix}
\end{equation}
$4\times 4$ matice $D$, která svazuje amplitudy jednotlivých módů s jejich příčnými poli, se nazývá \emph{dynamická matice}.
$D$ není jednoznačné, což souvisí s nejednoznačností ve volbě souřadné soustavy v rovině kolmé na směr šíření --- volbou $D$ volíme bázi Jonesových vektorů.
Častá volba, kterou představíme, ale později opustíme, je báze lineárního příčného (TE, s-polarizace) a podélného (TM, p-polarizace) módu.
Zde ji uvedeme tak, abychom byli konzistentní s přístupem, který zvolíme později, nese to však s sebou přechod do báze odraženého světla s opačnou točivostí.
Pravotočivá kruhová polarizace (RCP) s $\chi=1$ se nám při kolmém dopadu odrazí zase jako $\chi=1$, kvůli opačné točivosti báze jde však o levotočivé LCP.
Opačný přístup volí opačné znaménko odražené p-polarizace, čímž se zachová točivost (viz např. Ref \cite{Silber}).

Souřadnou soustavu volíme tak, aby rovina dopadu byla rovnoběžná s jednou ze souřadných os, zde zvolíme $yz$, takže $N_x=0$, $N_y=\sin \alpha$, kde $\alpha$ je úhel dopadu.
Jako bázi módů volíme lineární polarizace příčné $J_s$ a podélné $J_p$ k rovině dopadu, viz. obr. \ref{f:odraz a pruchod}.
Dynamická matice má v této situaci explicitně tvar
\begin{equation}
\begin{pmatrix} E_x \\ E_y \\ cB_x \\ cB_y \end{pmatrix}
=\begin{pmatrix}
1 & 0 & 1 & 0 \\
0 & \cos\alpha & 0 & \cos\alpha \\
0 & 1 & 0 & -1 \\
-\cos\alpha & 0 & \cos\alpha & 0
\end{pmatrix}
\begin{pmatrix}
J^-_{s} \\ J^-_{p} \\ J^+_s \\ J^+_{p} \,,
\end{pmatrix}
\end{equation}
kde můžeme rozpoznat ve sloupcích příčná pole jednotlivých módů (např. první sloupec jsou příčná pole $-z$ šířícího se s-polarizovaného módu: $E_y=0$, $cB_x=0$).

Při řešení průchodu a odrazu tedy pokládáme $J^+_r=0$ a snažíme se vyjádřit zbylé prošlé $J^-_r \equiv J^\text{trans}$ a odražené $J^+_l \equiv J^\textrm{refl}$ vyjádřit pomocí známého dopadajícího $J^-_l\equiv J^\textrm{inc}$, což lze pomocí matic $M$ a $D$ jednoduše z
\begin{equation}
\begin{pmatrix} J^\textrm{inc} \\ J^\textrm{refl} \end{pmatrix}
=D_l^{-1} M D_r \begin{pmatrix}
J^\text{trans} \\ 0
\end{pmatrix} \,,
\end{equation}
což vede na Jonesovy transmisní a reflexní matice Fresnelových koeficientů v bázi s- a p-polarizací
\begin{equation}
\begin{pmatrix} J^\textrm{refl}_s \\ J^\textrm{refl}_p \end{pmatrix}
=\begin{pmatrix}
r_{ss} & r_{sr} \\ r_{ps} & r_{pp}
\end{pmatrix}
\begin{pmatrix} J^\textrm{inc}_s \\ J^\textrm{inc}_p \end{pmatrix} \,, \qquad
\begin{pmatrix} J^\textrm{trans}_s \\ J^\textrm{trans}_p \end{pmatrix}
=\begin{pmatrix}
t_{ss} & t_{sp} \\ t_{ps} & t_{pp}
\end{pmatrix}
\begin{pmatrix} J^\textrm{inc}_s \\ J^\textrm{inc}_p \end{pmatrix} \,.
\end{equation}

Multivrstva složená pouze z izotropních vrstev nemíchá s- a p-polarizaci, mimodiagonální členy jsou nulové.
Pokud je jedna nebo více vrstev pouze slabě anizotropní, projeví se to malými nenulovými mimodiagonálními členy.
Při dopadu s-polarizace pak prošlé a odražené světlo nabyde i malé amplitudy p-polarizace, podobně pro dopadající p-polarizaci.
V elipsometrických parametrech se to projeví stočením hlavní roviny polarizace $\Delta \beta$ a elipticitou $\chi$, které se pro odražené světlo společně popisují komplexním stočením \cite{Silber}
\begin{equation}\label{e:komplexni rotace}
\Psi_s \equiv \Delta \beta_s - i \chi_s \approx \frac{r_{ps}}{r_{ss}} \,, \qquad \Psi_p \equiv \Delta \beta_p - i \chi_p \approx -\frac{r_{sp}}{r_{pp}} \,,
\end{equation}
kde jsme kvůli přechodu k opačné točivosti použili $-i\chi$, abychom se vyhnuli komplexnímu sdružení reflexních koeficientů.
V transmisním komplexním stočení jsou reflexní koeficienty nahrazeny transmisními a druhý člen je $+i\chi$, protože nedochází k přechodu k bázi s opačnou točivostí.
Komplexní parametr stočení se používá při popisu magnetooptických Kerrových jevů.
