\section{Popis polarizace}

V izotropním prostředí ($\e$ je násobek jednotkové matice) díky \eqref{e:NB} platí $\N \cdot \E_0=0$, dosazením \eqref{e:NE} do \eqref{e:NB} dostaneme rovnici pro přípustná $\N$
\begin{equation}
\N \cdot \N =\e \,.
\end{equation}
Vektor $\N$ obecně může být ve speciálních případech komplexní i pro nedisipativní prostředí (reálné $\e$, např. evanescentní vlny) a proto $\N \cdot \N$ není obecně rovno normě $|\N|^2$.

Pro popis světla ve volném prostoru ($\e=1$) se dále omezíme na reálné $\N$.
Zvolíme osu $z$ proti směru šíření, tedy $\N=(0,0,-1)^\T$.
Řešením jsou pole charakterizovaná libovolnými komplexními amplitudami $E_x$ a $E_{y}$, které udávají polarizaci vlny.
$\B_0$ je pak dáno rovnicí \eqref{e:NE}.

Výkon přenášený rovinou vlnou je dán časově středovaným Poyntingovým vektorem $\vec{S}$
\begin{equation} \label{e:intenzita}
\langle \vec{S}(t)\rangle_t=\langle\E(t)\times\H(t)\rangle_t=\frac{\evac}{2\muvac}\N \left( E_{0x}^*E_{0x}+E_{0y}^*E_{0y} \right) \equiv \frac{\evac}{2\muvac}\N I\,,
\end{equation}
kde jsme definovali intenzitu $I=E_{0x}^*E_{0x}+E_{0y}^*E_{0y}$.

Existuje několik přístupů, jak popisovat stav světla. Jako první uvedeme elipsometrické parametry, které jsou názorné, ale nejsou vhodné pro výpočty.
Formalismus Jonesových vektorů pracuje přímo s komplexními amplitudami polí, a je proto užitečný při počítání transmisních a reflexních koeficientů.
Poslední, který popíšeme, je popis pomocí Stokesových parametrů a Muellerových matic, který pracuje s intenzitami v různých projekcích Jonesových vektorů, tj. intenzitami po myšleném průchodu vhodným ideálním polarizátorem.

Vektor $\E(t)$ v čase opisuje v rovině $xy$ elipsu, případně úsečku nebo kruh\cite{Bornwolf}.
Polarizační elipsa je pak popsána elipsometrickými parametry: $\beta$ - úhel natočení hlavní poloosy a $\chi$ - elipticita\footnote{Uvedené $\chi$ se někdy nazývá úhel elipticity, název elipticita pak nese jiná veličina.}.
K úplnému popisu polarizačního světla ekvivalentního popisu pomocí $E_{0x}$ a $E_{0y}$ je nutné přidat celkovou intenzitu $I$ a fázový posun $\delta$, určující v jakém bodě elipsy se $\E(t)$ nachází v čase $t=0$; $\E(t)$ protíná hlavní poloosu v čase $t=\w \delta$.
Jediný zbylý neurčený parametr je smysl obíhání elipsy, který popíšeme znaménkem $\chi$, viz diskuze na konci dalšího oddílu.
Konvenci točivosti světla používáme podle pohledu od zdroje: pravotočivé světlo obíhá na obr. \ref{f:pol_elipsa} po směru hodinových ručiček.

\begin{figure}\centering
\includegraphics[width=0.5\linewidth]{./img/t1.png}
\caption{Polarizační elipsa. Obrázek je kreslen z pohledu od zdroje, pokud vektor $\E$ obíhá vyznačeným směrem, pak ho označujeme za pravotočivé a v této souřadné soustavě $\chi>0$.}\label{f:pol_elipsa}
\end{figure}

\subsection{Jonesovy vektory a matice \cite{Jones}}
Jonesovy vektory jsou tvořeny komplexními amplitudami polí v rovině kolmé na vektor šíření
\begin{equation}
\J=\begin{pmatrix} E_{x} \\ E_{y} \end{pmatrix} \,.
\end{equation}
Prostor Jonesových vektorů je přirozeně normovaný intenzitou \eqref{e:intenzita}, která je dána skalárním součinem
\begin{equation}
\J_1^\dagger \J_2=
\begin{pmatrix}
J_{1x}^* & J_{1y}^*
\end{pmatrix}
\begin{pmatrix}
J_{2x} \\ J_{2y}
\end{pmatrix}
\,, \qquad I(\J)=\J^\dagger \J
\end{equation}
Dvě polarizace/Jonesovy vektory jsou "ortogonální" ($\J_1^\dagger \J_2=0$), pokud je celková intenzita prostý součet intenzit v obou polarizacích.

Akci každého lineární polarizačního prvku (nevyužívá nelineární optické jevy), lze popsat jako lineární transformaci Jonesova vektoru.
Pro každý takový prvek existuje $2\times 2$ komplexní matice $T$ (Jonesova matice), taková, že pokud je příchozí polarizace dána libovolným $\J_{\textrm{in}}$, tak odchozí po působení prvku je dána $\J_{\textrm{out}}$ splňujícím
\begin{equation}
\J_{\textrm{out}}=T \J_{\textrm{in}} \,.
\end{equation}
Postupnou akci více prvků lze popsat maticí, která je součinem matic jednotlivých prvků.
Jonesovy matice lze použít i např. v případě polarizačního děliče, každé rameno pak má vlastní matici.

Významné postavení mezi optickými prvky mají takové, které zachovávají intenzitu (např. fázové retardační destičky).
Zachování intenzity pro všechny příchozí polarizace
\begin{equation}
\J_{\textrm{in}}^*\J_{\textrm{in}}=I_{\textrm{in}}=I_{\textrm{out}}=\left(T \J_{\textrm{in}}\right)^* \left(T \J_{\textrm{in}}\right)=\J_{\textrm{in}}^* T^\dagger T \J_{\textrm{in}}
\end{equation}
dává pro matici $T$ podmínku $T^\dagger T=1$, a tedy $T$ musí být unitární.

Jonesův vektor, který odpovídá elipsometrickým parametrům $\beta$, $\chi$, $\delta$, $I$ je
\begin{equation} \label{e:jones uhly}
\J=\sqrt{I} e^{i\delta} \begin{pmatrix}
\cos \chi \cos \beta + i \sin \chi \sin \beta \\
\cos \chi \sin \beta - i \sin \chi \cos \beta
\end{pmatrix} \,.
\end{equation}
Tento vztah platí pro směr šíření $\vec{k} \parallel -x \times y$ jako na obr. \ref{f:pol_elipsa}.
Pozdeji však budeme stejné souřadnice $x,y$ používat i pro světlo v opačném směru, aby byla reflexní matice při kolmém dopadu jednotková matice.
Ve skutečnosti $\beta$, $\chi$ i $\delta$ definujeme vzhledem ke vztažné soustavě Jonesových vektorů.
$\beta$ definujeme vzhledem k $J_x$, s rostoucím úhlem ve směru $+J_y$.
$\delta$ je dáno počátkem času $t=0$ a vztah mezi znaménkem $\chi$ a točivostí je dán následující poučkou: pokud $J_x J_y \vec{k}$ tvoří levotočivý systém jako na obr. \ref{f:pol_elipsa}, pak $\chi>0$ odpovídá pravotočivé polarizaci.
Pro pravotočivý systém jsou znaménka prohozena.

\subsection{Stokesovy parametry a Muellerovy matice \cite{JonesMueller}} \label{k:stokes}
Stokesovy parametry obecně mohou narozdíl od Jonesových vektorů popsat i nepolarizované nebo částečně polarizované světlo.
V této práci si však vystačíme s úplně polarizovaným světlem a proto je nebudeme definovat obecně, ale pomocí Jonesových vektorů
\begin{align}
s_0 \equiv J_x^* J_x+J_y^* J_y\equiv \J^\dagger \sigma_0 \J &= I \,,\\
s_1 \equiv J_x^* J_x-J_y^* J_y\equiv \J^\dagger \sigma_1 \J &= I \cos 2\chi \cos 2\beta \,,  \\
s_2 \equiv J_x^* J_y+J_y^* J_x\equiv \J^\dagger \sigma_2 \J &= I \cos 2\chi \sin 2\beta \,, \\
s_3 \equiv i J_x^* J_y-i J_y^* J_x  \equiv \J^\dagger \sigma_3 \J &= I \sin 2\chi \,.
\end{align}
Stokesovy parametry je možné vyjádřit jako střední hodnoty vhodných $2\times 2$ hermitovských matic $\sigma_{i}$, jak je vyjádřeno druhými rovnostmi. Ty jsou
\begin{align}
\sigma_0=\begin{pmatrix} 1 & 0 \\ 0 & 1 \end{pmatrix} ,\,
\sigma_1=\begin{pmatrix} 1 & 0 \\ 0 & -1 \end{pmatrix} ,\,
\sigma_2=\begin{pmatrix} 0 & 1 \\ 1 & 0 \end{pmatrix} ,\,
\sigma_3=\begin{pmatrix} 0 & i \\ -i & 0 \end{pmatrix}
\end{align}
které můžeme rozeznat jako jednotkovou matici a přerovnané Pauliho matice.
Tvoří bázi všech $2\times 2$ hermitovských matic: každou $2\times 2$ hermitovskou matici lze napsat jako lineární kombinaci $\sigma_i$ s reálnými koeficienty.
Obecné (ne nutně hermitovské) komplexní $2\times 2$ matice mají také jednoznačný rozklad do báze $\sigma_i$, tentokrát však s komplexními parametry.

Tato vlastnost má možná neintuitivní důsledek, že každý optický prvek, který je lineární v Jonesových vektorech, je zároveň lineární ve Stokesových parametrech.
\begin{equation} \label{e:muellerova matice}
S^\textrm{out}_i=\J^\dagger T^\dagger \sigma_i T \J\equiv\J^\dagger \left(\sum_{j=0}^{3} M_{ij} \sigma_j \right) \J=\sum_{j=0}^{3} M_{ij} S^\textrm{in}_j \,,
\end{equation}
kde reálná $4\times 4$ matice $M$ je dána právě rozkladem hermitovských matic $T^\dagger \sigma_i T$ do báze $\sigma_j$.
Matice $M$ charakterizující optický prvek se nazývá Muellerova matice, a sloupcový vektor složený ze Stokesových parametrů se nazývá Stokesův vektor.
Složky Muellerovy matice příslušející Jonesově matici $T$ je možné počítat přímo z rozkladu \eqref{e:muellerova matice} díky vlastnosi, která se nazývá trace-ortogonalita $\operatorname{Tr}\lbrace\sigma_j\sigma_i\rbrace=2\delta_{ji}$
\begin{equation} \label{e:mueller rozklad}
M_{ij}=\frac{1}{2}\operatorname{Tr}\lbrace \sigma_j T^\dagger \sigma_i T \rbrace \,.
\end{equation}

Podobně jako elipsometrické parametry, Stokesovy vektory nejsou úplně ekvivalentní Jonesovým vektorům, protože ztrácí informaci o časovém zpoždění $\delta$.
Navíc aby daný Stokesův vektor popisoval fyzikální stav světla, je nutné aby splňoval určité podmínky.
Pro úplně polarizované světlo nejsou jeho složky nezávislé, platí totiž
\begin{equation} \label{e:norma S}
S_0=\sqrt{S_1^2+S_2^2+S_3^2}
\end{equation}
a tedy nám k vyjádření stačí tři parametry $S_1, S_2, S_3$.
Ty už mohou být libovolná reálná čísla a zobrazují se v třírozměrném prostoru.
Délka tohoto třírozměrného vektoru udává intenzitu a směr udává polarizaci.
Polarizace s jednotkovou intenzitou se zobrazují na tzv. Poincarého sféře jako na obr. \ref{f:poincareho sfera}.
Ortogonální polarizace jsou zobrazeny na body středově souměrné podle počátku.

\begin{figure}
\includegraphics[width=0.5\linewidth]{./img/t2.png}\centering
\caption{Poincarého sféra.}\label{f:poincareho sfera}
\end{figure}

Zachování \eqref{e:norma S} pro všechny Stokesovy vektory klade podmínku na Muellerovy matice.
Přestože libovolná myslitelná 4x4 reálná matice je dána 16 reálnými parametry, nedepolarizační (také nazývaná čistá\footnote{Ve smyslu čistého (nesmíšeného, angl. pure) stavu v kvantové mechanice.}) Muellerova matice je dána pouze 7 reálnými čisly\footnote{Jonesova matice je dána 4 komplexními čísly --- 8 reálných parametrů, ale při přechodu k Muellerově matici ztratíme informaci o celkové fázi Jonesovy matice.}. \cite{Muellerdiff}

\subsection{Charakteristické elipsoidy \cite{Muellergeom}}

Nyní se zaměříme na to, jakým způsobem působí obecné Muellerovy matice.
Muellerovy matice mají jednoduchý geometrický význam, který se graficky vyjadřuje pomocí tzv. charakteristických elipsoidů.
Charakteristický elipsoid Muellerovy matice $M$ je množina bodů v třírozměrném prostoru $(s_1, s_2, s_3)^T$, které vzniknou akcí $M$ na body ležící na Poincarého sféře.
Jinými slovy, každá Muellerova matice způsobí deformaci Poincarého sféry, výsledkem je vždy elipsoid.
V charakteristickém elipsoidu držíme i informaci o tom, na které body se transformují které body Poincarého sféry - např. fázové destičky mají za následek pouze rotaci Poincarého sféry.

Pro popis plně polarizovaného světla se omezíme na případ čistých Muellerových matic, které vzniknou rozkladem \eqref{e:mueller rozklad} z nějaké Jonesovy matice.
Zaměříme se na dva případy: prvky reprezentované unitární Jonesovou maticí $U$ a prvky reprezentované hermitovskou pozitivně semidefinitní Jonesovou maticí $H$.

Záminku, proč zkoumat tyto dva případy $U$ a $H$, nám poskytuje věta z lineární algebry o polárním rozkladu matice\cite{pestujemealgebru}, která tvrdí, že pro každou komplexní matici $T$ existují jednoznačné rozklady $T=U H_1$ a $T=H_2 U$.\footnote{$U$ je v obou rozkladech stejné, $H_1$ a $H_2$ nemusí.}

\subsubsection*{Obecná retardační destička}

Prvek je reprezentovaný unitární Jonesovou maticí $U$.
Zachování intenzity má za důsledek $M_{00}=1$, $M_{0i}=M_{j0}=0$ pro $i,j=1,2,3$ a navíc podmatice $M{ij}$ musí zachovávat normu 3-vektoru $(S_1, S_2, S_3)$, tedy být ortonormální.
Jediné takové matice jsou 3D rotační matice, případně složené se zrcadlením.
Vzhledem k tomu, že $U$ je unitární, má dvě ortogonální vlastní polarizace $\J_1$, $\J_2$ s vlastními čísly, které jsou pouze fázové faktory. Je možné ji diagonalizovat\footnote{Ve vzorci vystupuje dyadický součin Jonesových vektorů $\J_1\J_1^\dagger$, což je ortogonální projektor na $\J_1$, ne skalární součin, který by byl psaný $\J_1^\dagger\ J_1$.}
\begin{equation}
U=e^{i\Delta_1} \J_1 \J_1^\dagger + e^{i\Delta_2} \J_2 \J_2^\dagger\,.
\end{equation}
Tyto dva vlastní módy mají po průchodu prvkem stejný polarizační stav, takže musí být i vlastními vektory Muellerovy matice, prochází jimi osa zmíněné rotace.
Úhel rotace je daný fázovým zpožděním mezi vlastními módy $\Delta_1-\Delta_2$, viz obr. \ref{f:akce muelleru} (a).

\subsubsection*{Obecný polarizátor}

Prvek je reprezentovaný pozitivně semidefinitní hermitovskou Jonesovou maticí $H$.
To znamená, že pro ně existují dvě ortogonální vlastní polarizace $J_1$, $J_2$ s reálnými nezápornými vlastními čísly.
Normalizací matice tak, že větší z vlastních čísel se rovná 1, lze psát s reálným nezáporným $\eta$
\begin{equation}
H=J_1 J_1^* + \eta J_2 J_2^* \,.
\end{equation}
Znamená to, že prvek je obecný polarizátor, který $J_1$ propustí zcela a $J_2$ propustí s amplitudovou propustností $\eta$.
Ve speciálním případě, kdy polarizátor propouští lineární polarizaci v ose $x$: $J_1=(1,0)^\T$ a $J_1=(0,1)^\T$, je Jonesova a Muellerova matice
\begin{equation}
H=\begin{pmatrix}
1 & 0 \\ 0 & \eta
\end{pmatrix} \,, \qquad
M_H=\begin{pmatrix}
\frac{1+\eta^2}{2} & \frac{1-\eta^2}{2} & 0 & 0 \\ \frac{1-\eta^2}{2} & \frac{1+\eta^2}{2} & 0 & 0 \\
0 & 0 & \eta & 0 \\ 0 & 0 & 0 & \eta
\end{pmatrix} \,.
\end{equation}

Pro výpočet charakteristického elipsoidu dosadíme $S_0^{\textrm{in}}=1$ a dostaneme
\begin{align}
    S_1^{\textrm{out}}&=\frac{1+\eta^2}{2} S_1^{\textrm{in}}+\frac{1-\eta^2}{2} \\
    S_2^{\textrm{out}}&=\eta S_2^{\textrm{in}} \\
    S_3^{\textrm{out}}&=\eta S_3^{\textrm{in}}
\end{align}
Jedná se tedy o kontrakci v rovině $S_2S_3$ faktorem $\eta$, ve směru $S_1$ faktorem $(1+\eta^2)/2$ a zároveň posunutím o $(1-\eta^2)/2$.
Nebo ekvivalentně kontrakcí stejným faktorem se středem v $S_3=1$. Viz obr. \ref{f:akce muelleru} (b).

\begin{figure}\centering
\includegraphics[width=\linewidth]{./img/t3.png}
\caption{Grafické zobrazení akce (a) obecné retardační destičky a (b) obecného polarizátoru.}\label{f:akce muelleru}
\end{figure}

Shrneme-li uvedené poznatky, akce libovolného nedepolarizačního optického prvku je ekvivalentní postupnému působení obecného polarizátoru (zploštění a posunutí ve směru vlastního vektoru $H$ jako na obr. \ref{f:akce muelleru} (b)) a obecné fázové retardační destičky (rotace podle směru vlastního vektoru $U$ jako na obr. \ref{f:akce muelleru} (a)), případně v opačném pořadí.
