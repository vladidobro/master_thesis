\section{Magnetická anizotropie}

Magnetizace $\M$ materiálu není veličina, která by se dala v experimentu přímo ovládat.
V experimentu můžeme aplikovat vnější pole $\Hext$ a materiál si sám najde rovnovážnou polohu $\M$.
Na druhou stranu magnetooptické vlastnosti, jak je patrné z předchozího oddílu, záleží na $\M$.
Tento oddíl se věnuje vztahu mezi $\M$ a $\Hext$.

Zanedbáme problémy spojené s demagnetizačními poli a budeme pro ilustraci uvažovat systém s homogenní magnetizací.
Pokud je udržovaný na teplotě $T$, udává jeho termodynamické vlastnosti hustota volné energie\footnote{Materiál ve skutečnosti udržujeme zároveň i na konstantním tlaku, takže bychom správně měli používat Gibbsův potenciál.} $F(T,\M)$ \cite{Callen}.
Aby mohl být systém v rovnováze při konkrétním $\M$, musí externí pole jakožto přidružený intenzivní parametr splňovat
\begin{equation} \label{e:Hext=gradF}
\muvac\Hext(\M) = \nabla_{\vec{M}} F(\vec{M})
\end{equation}
Radši bychom ale znali závislost rovnovážného $\M$ v situaci, kdy je systém obklopen magnetickým polem $\Hext$ tvořeným např. cívkami elektromagnetu.
Mezi magnetem a studovaným systémem dochází k výměně energie prostřednictvím magnetického pole, systém je v kontaktu s "magnetickým rezervoárem" a v rovnováze proto dochází k minimalizaci \emph{celkové} volné energie.
V souladu s teorií termodynamických potenciálů tedy přejdeme k Legendrově transformaci v $\Hext$ -- "magnetické entalpii"\footnote{Někdy označované jako magnetický Gibbsův potenciál.} systému\cite{magentalpie}
\begin{equation} \label{e:mag entalpie}
\Omega(T,\Hext)= -\muvac\Hext\cdot\M(\Hext)+F\left(\M(\Hext)\right) \,.
\end{equation}
Princip minima termodynamických potenciálů nám říká, že v takové situaci $\M(\Hext)$ nabývá takové hodnoty, která minimalizuje magnetickou entalpii pro pevnou hodnotu $\Hext$.
Tímto způsobem tvar $F(\M)$ určuje, jakých $\M$ bude systém nabývat při všech možných $\Hext$.

V obecné situaci, kdy magnetizace není homogenní a jednotlivá místa systému spolu interagují, může volná energie být obecným nelokálním funkcionálem prostorového rozložení magnetizace.
Hustota $F(\M)$ z předchozího odstavce se proto nazývá \emph{funkcionál volné energie}.

Široce používaný model feromagnetů v jedno-doménovém stavu je tzv. Stonerův-Wohlfarthův model\cite{StonerWohlfarth}.
Předpokládá, že funkcionál $F(\M)$ má význam pouze lokální hustoty a $\M$ je tedy dána minimalizací \eqref{e:mag entalpie}.
Ve formě, v jaké SW model budeme používat, zahrnujeme do volné energie 4 příspěvky\cite{Reichlova}\cite{Janda}\cite{Kucharik}
\begin{equation}
F=F^\textrm{exchange} + F^\textrm{magnetocrystalline} + F^\textrm{shape} + F^\textrm{strain}
\end{equation}
První člen, způsobený výměnnou interakcí, má na svědomí feromagnetismus; závisí na celkové velikosti magnetizace $|\M|$ a má ostré minimum, když jsou všechny mikroskopické magnetické momenty orientované stejným směrem a magnetizace je saturovaná $|\M|=M_S$.
Magnetokrystalická anizotropie $F^\textrm{magnetocrystalline}$ popisuje interakci s krystalickou mřížkou, tvarová anizotropie $F^\textrm{shape}$ popisuje vliv tvaru vzorku a strainová anizotropie $F^\textrm{strain}$ popisuje anizotropie způsobené mechanickým napětím (např. když je vzorek nanesen na substrátu s jinou mřížkovou konstantou).

Dále se omezíme na situaci relevantní pro tuto práci.
Vzorek je feromagnetický, výměnná interakce způsobuje, že vždy $|\M|=M_S$.
Vzorek je kubický tenký film s krystalografickými směry [100], [010] a [001] shodnými s kladnými poloosami $x$, $y$ a $z$.
Tvarová anizotropie způsobí vymizení out-of-plane magnetizace $\M_z=0$
\begin{equation} \label{e:magnetizace v rovine}
\M=\begin{pmatrix}
M_x \\ M_y \\ M_z
\end{pmatrix} = M_S \begin{pmatrix}
\cos \phim \\ \sin \phim \\ 0
\end{pmatrix} \,.
\end{equation}
Magnetokrystalickou anizotropii rozvineme do mocninné řady v $\M$ a ponecháme pouze nejnižší člen respektující kubickou symetrii, s uvážením \eqref{e:magnetizace v rovine}
\begin{equation}
\frac{F^\textrm{magnetocrystalline}}{M_S}=k_4 \sin^2 \phim \cos^2 \phim \,,
\end{equation}
čímž jsme definovali kubickou anizotropní konstantu $k_4$.
Pro $k_4>0$ má minima (snadné osy), ve směrech [100] a [010] (tj. $\phim=\SI{0}{\degree}$, \SI{90}{\degree}), pro $k_4<0$ jsou to [110] a [1-10] (tj. $\phim=\SI{45}{\degree}$, \SI{135}{\degree}).

Navíc povolíme uniaxiální strainovou anizotropii.
Také jí rozvineme do řady a se zkušeností, že mívají často uniaxiální charakter, ponecháme pouze první člen a opět vydělíme $M_S$ pro definici uniaxiální anizotropní konstanty $k_u$ a směru $\phiu$.
\begin{equation}
\frac{F^\textrm{strain}}{M_S}=k_u \sin^2\left( \phim-\phiu  \right) \,.
\end{equation}
$\phiu$ je takto definováno vzhledem ke krystalografickému směru [100].
Je dostačující omezit se na $k_u\geq 0$, snadné směry jsou pak ve $\phim=\phiu, \phiu+\SI{180}{\degree}$.
Obě hodnoty $\phiu$ a $\phiu+\SI{180}{\degree}$ popisují stejné $F^\textrm{strain}$, takže pokud v konkrétním případě nemáme důvod konat jinak, omezujeme se na $\phiu \in [\SI{0}{\degree}, \SI{180}{\degree}]$.

Kanonický tvar funkcionálu volné energie tenkého kubického filmu v rovině $xy$ orientovaného $[100]=x$ tedy píšeme
\begin{equation} \label{e:F}
\frac{F(\phim)}{M_S}=k_4 \sin^2 \phim \cos^2 \phim + k_u \sin^2\left( \phim-\phiu  \right) \,.
\end{equation}

Když se omezíme na $\Hext$ v rovině $xy$
\begin{equation}
\Hext =\Hx \begin{pmatrix} \cos \phih \\ \sin \phih \\ 0 \end{pmatrix}
\end{equation}
pak závislost $\phim(\phih)$ je dána minimalizací hustoty magnetické entalpie (vydělené konstantním $M_S$)
\begin{equation}
\frac{\Omega}{M_S}=-\muvac \Hx \cos \left(\phim-\phih \right) + k_4 \sin^2 \phim \cos^2 \phim + k_u \sin^2\left( \phim-\phiu  \right)
\end{equation}
Dělení $M_S$ zavádíme, aby anizotropní konstanty $k_4$ a $k_u$ měly dimenzi magnetického pole a byly přímo porovnatelné s experimentálně ovladatelným $\muvac\Hx$, bez nutnosti znalosti $M_S$.

Pro praktické účely je výhodné vyjádřit \eqref{e:F} ekvivalentním způsobem pro vzorek obecně natočený v rovině $xy$ o úhel $\gamma$, tzn. [100] je ve směru vektoru $(\cos\gamma, \sin\gamma, 0)$.
Pak až na bezvýznamnou aditivní konstantu
\begin{align}
\frac{F(\phim)}{M_S}&=-\frac{k_4}{8} \cos 4(\phim-\gamma)-\frac{k_u}{2} \cos 2(\phim-\phiu-\gamma) \\
&=-\frac{k_{4x}}{8} \cos 4\phim - \frac{k_{4y}}{8} \sin 4\phim - \frac{k_{ux}}{2} \cos 2 \phim - \frac{k_{uy}}{2} \sin 2 \phim \,,
\end{align}
kde
\begin{eqnarray}
k_{4x}=k_4 \cos 4\gamma \,, & k_{ux}=k_u \cos 2 \left(\phiu+\gamma\right)\\
k_{4y}=k_4 \sin 4\gamma \,, & k_{uy}=k_u \sin 2 \left(\phiu+\gamma\right)
\end{eqnarray}


Existence volné energie má netriviální důsledek na tvar závislosti $\M(\Hext)$.
Pro relevantní situaci saturované in-plane magnetizace a rotujícího vnějšího pole konstantní velikosti má tvar
\begin{equation} \label{e:M integracni konstanta}
\muvac\Hx M_S \int_{0}^{2\pi}  \frac{\text{d}\phim}{\text{d}\phih} \sin\left(\phim-\phih\right) \text{d}\phih=0 \,,
\end{equation}
za podmínky, že $\phim$ je spojitou funkcí $\phih$ - nedochází k přeskokům magnetizace.

Důsledkem je např. intuitivní fakt, že z myslitelných průběhů $\phim(\phih)=\phih+c$ je jediný možný ten, pro který $c=0$.
Není tedy možné, aby magnetizace konzistentně "předbíhala" nebo se "opožďovala" za přiloženým polem.
Pokud z experimentu dokážeme určit pouze $\text{d}\phim/\text{d}\phih$, podmínka \eqref{e:M integracni konstanta} nám dovoluje určit integrační konstantu.

Obecný tvar a podrobnosti jsou uvedeny v dodatku \ref{k:dodatek volna energie}.
