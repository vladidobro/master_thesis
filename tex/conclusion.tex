\chapter*{Závěr}
\addcontentsline{toc}{chapter}{Závěr}

V naší laboratoři dlouhodobě vyvíjíme metodu studia kvadratických mag\-ne\-to-optických jevů, která dokáže současně určit jejich sílu a anizotropii a zároveň pomocí nich určit in-plane magnetickou anizotropii vzorku.

Díky velice malému úhlu dopadu ($\approx\SI{0.5}{\degree}$) je v naší geometrii možné použít vstupní světlo s libovolně orientovanou polarizací a tím suplovat otáčení vzorku, které je nezbytnou součástí postupu měření obdobných metod (viz kap. \ref{chap:2}).
To v důsledku umožňuje měření s kryostatem, což doposud nebylo u těchto obdobných metod možné.
V našem experimentálním uspořádání je možné vlastnosti studovaných vzorků měřit v širokém teplotním intervalu (od 15 do \SI{800}{\kelvin}) a širokém oboru vlnových délek světla (od 460 do \SI{1600}{\nano\meter}).

V rámci této diplomové práce byly identifikovány a následně odstraněny dva experimentální problémy, které praktické použití naší metody zatím znemožňovaly.
Oba problémy způsobují, že v měřeném signálu se nedefinovaným poměrem míchá stočení i elipticita způsobená vzorkem.
Konkrétně se jedná o:
\begin{enumerate}
    \item Nedokonalost $\lambda/2$ fázové destičky před optickým můstkem. 
        Řešením je měření pro dvě neekvivalentní polohy popsané v oddíle \ref{chap:kompenzace}.
    \item Změna polarizace při odrazu od zrcadla umístěným mezi vzorkem a optickým můstkem.
        Řešením je kompenzace pomocí druhého zrcadla popsaná taktéž v oddíle \ref{chap:kompenzace}, či kompenzace jinou metodou tam popsanou.
\end{enumerate}

Třetí problém byl v použitém zpracování dat a pro jeho odstranění bylo dosaženo podstatně hlubšího porozumění různých aspektů experimentu.
Je kladen důraz (oddíly \ref{chap:anizotropie-MLD}, \ref{chap:urceni-magneticke-anizotropie} a dodatek \ref{app:berreman}) na explicitní vymezení nutných předpokladů metody, aby bylo případně usnadněno její rozšíření na širší třídu vzorků než jsou [001]-normálově orientované tenké filmy s kubickou mříží.

Použití této metody bylo demonstrováno jak v transmisní geometrii (vzorek CoFe, oddíl~\ref{chap:vysledky-cofe}), tak v reflexní geometrii (vzorek FeRh, oddíl~\ref{chap:vysledky-ferh}).
Naše měření ukázala, že koeficient popisující kvadratickou magneto-optickou odezvu může být silně anizotropní, přičemž velikost této anizotropie i její znaménko silně závisí na použité vlnové délce světla.
To má poměrně zásadní důsledky pro plánování a/nebo vyhodnocování příslušných experimentů využívajících kvadratickou magneto-optiku, která v současné době přitahuje rostoucí pozornost díky nástupu zájmu o antiferomagnetickou spintroniku.
