\chapter*{Předmluva}
\addcontentsline{toc}{chapter}{Předmluva}

Výhoda automatizovaných výpočtů je lidstvu zřejmá již po staletí.
První výpočetní stroje byly založené na všemožných konstrukcích; nejdříve byly realizované člověkem s tužkou a papírem, od 19. století to byly ručně nebo parně poháněné stroje plné ozubených koleček zpracovávajících děrné štítky, a později s vývojem vakuových elektronek už šlo o počítače podobné těm dnešním, pouze mnohonásobně větší a poruchovější.
Obzvláště během druhé světové války potřeba počítačů extrémně vynikla, protože se náhle staly nezbytnými výpočetně náročné činnosti jako luštění šifer a modelování atomových bomb. Mnoho bylo v sázce a všechny tři zmíněné přístupy byly využity naplno.

Jejich společná nevýhoda byla škálovatelnost, nebylo dost dobře možné počítače založené na těchto principech zmenšovat a zrychlovat.
Když v roce 1947 William Shockley, John Bardeen a Walter Brattain vynalezli pevnolátkový elektrický spínač bez potřeby vakua - transistor, rychle bylo rozpoznáno, že to je právě ta součástka, na kterou všichni čekali.
Transistory byly oproti elektronkám mnohem účinnějsí a bylo je možné miniaturizovat, a tak začal trend rapidního zrychlování a zmenšování počítačů, který trvá až dodnes.
Teoretické limity na minimální velikost transistoru byly v nedohlednu a rychlost vývoje byla dobře vystižena tzv. Moorovým zákonem - maximální technicky realizovatelný počet transistorů v integrovaném obvodu se zdvojnásobuje každé dva roky.

Dnes se píše rok 2022, Moorův zákon začíná pozbývat platnosti, technika transistorových počítačů se dotýká teoretických limitů a lidstvo se ocitá v podobné situaci jako před rokem 1947 - stávající počítače jsou na konci své vývojové linie.

Jedním z možných směrů, kudy se vydat dále, je \emph{spintronika}.
Běžná elektronika využívá pouze jednu vlastnost elektronů - jsou to elektrické monopóly, a zcela ignoruje jinou - mají spin a jsou to magnetické dipóly.
Spintronika je vědní obor, který se zabývá vlivem spinu na transportní vlastnosti elektronů, s vysněným cílem obejít některé technologické potíže týkající se elektroniky.
Jako velký úspěch spintroniky se považuje objevení \emph{obří magnetorezistence} (GMR) v roce 1988; magnetorezistence obecně znamená změnu odporu při aplikaci magnetického pole, přívlastek obří značí jeho sílu (\SI{80}{\percent} oproti malým jednotkám procent pozorovaných před rokem 1988).
Mikroskopický základ GMR je právě interakce spinu elektronu s magnetickým polem, jedná se proto o spintronický jev.
Objev GMR umožnil výrobu dnešních hard-disků s nevídanou kapacitou a vývoj magnetoresistivních RAM (MRAM), které sice dosud nebyly komerčně využity, ale oproti ostatním pamětím některé významné výhody.

Hlavní způsoby, jakými se v pevných látkách projevuje spin elektronu, jsou interakce magnetického momentu s mikroskopickým magnetickým polem uvnitř materiálu a spin-orbitální interakce\cita.
První z nich se projevuje především v materiálech s magnetickým uspořádáním --- např. feromagnetech, antiferomagnetech --- spintronika proto přirozeně zahrnuje vývoj a studium těchto materiálů.
Takový vývoj ovšem vyžaduje zároveň diagnostické metody, kterými se mohou nové materiály zkoumat, vybírat z nich ty nejslibnější a optimalizovat jejich parametry.

Jedné z takových diagnostických metod -- magnetooptické spektroskopii -- se věnuje tato práce.
Magnetooptika (MO) zkoumá magnetismus skrze jeho projevy v optických vlastnostech.
MO jevy lze obecně rozdělit na liché a sudé.
Zatímco liché jevy jsou široce používané už desítky let, ty sudé nabývají na popularitě až v současné době.
Jedním z důvodů, proč jsou sudé jevy důležité pro spintroniku, je, že narozdíl od lichých existují i v kompenzovaných antiferomagnetech, které jsou slibné pro spintronické využití.

V roce 2017 byl v Laboratoři OptoSpintroniky (LOS) na Katedře CHemické Fyziky a Optiky (KCHFO) na Univerzitě Karlově (UK) uveden do provozu vektorový magnet s názvem \emph{Molzilla}, jehož účelem bylo získání metody pro magnetooptickou charakterizaci vzorků v širokém rozsahu teplot a vlnových délek, včetně měření sudých MO jevů.
Sudé MO jevy zatím nebyly kvantitativně studovány při jiných než pokojových teplotách.
Metoda využívá rotující pole a je proto koncepčně příbuzná tzv. 8-směrné metodě, metodě rotujícího pole a ROTMOKE.

V lednu 2021 na začátku tvorby této práce již bylo metodou studováno mnoho vzorků rozličných materiálů, avšak jakékoliv pokusy o kvantitativní interpretaci sudých jevů buď úplně selhávaly, nebo dávaly odporující si výsledky.
Nejrozumnějším řešením se v té době jevilo projekt opustit a minimalizovat tím budoucí ztráty.
Hlavním a jediným cílem této práce bylo metodu vyšetřit, pokusit se odhalit příčiny problémů a v ideálním případě ji opravit a uvést do provozu.

To vše se nakonec více než povedlo, výsledkem je funkční metoda, která rozšiřuje možnosti uvedených příbuzných metod, a její funkčnost byla demonstrována na dvou materiálech.
