\chapter*{Předmluva}
\addcontentsline{toc}{chapter}{Předmluva}
Výhoda automatizovaných výpočtů je lidstvu zřejmá již po staletí.
Nej\-dří\-ve realizované otrokem s tužkou a papírem, od 19. století to byly ručně nebo parně poháněné stroje plné ozubených koleček zpracovávajících děrné štítky, a později s vývojem vakuových elektronek už šlo o počítače podobné těm dnešním, pouze mnohonásobně větší a poruchovější.
Obzvláště během druhé světové války potřeba počítačů extrémně vynikla, protože se náhle staly nezbytnými výpočetně náročné činnosti jako luštění šifer a modelování atomových bomb.
Mnoho bylo v sázce a všechny zmíněné přístupy byly využity naplno.

Jejich společná nevýhoda byla škálovatelnost, nebylo dost dobře možné počítače založené na těchto principech zmenšovat a zrychlovat.
Když byl v roce 1947 vynalezen pevnolátkový elektrický spínač bez potřeby vakua -- tranzistor -- rychle bylo rozpoznáno, že to je právě ta součástka, na kterou všichni čekali.
Teoretické limity na miniaturizaci a zrychlování transistorových počítačů byly v nedohlednu a rychlost vývoje byla od šedesátých let dobře vystižena Moorovým zákonem: maximální technicky realizovatelný počet transistorů v integrovaném obvodu se zdvojnásobuje každé dva roky.

Dnes se píše rok 2022, Moorův zákon začíná pozbývat platnosti, technika transistorových počítačů se dotýká teoretických limitů a lidstvo se ocitá v podobné situaci jako před rokem 1947 - stávající počítače jsou na konci své vývojové linie.

Jedním z možných směrů, kudy se vydat dále, je \emph{spintronika}.
Běžná elektronika využívá pouze jednu vlastnost elektronů -- jsou to elektrické monopóly -- a zcela ignoruje jinou -- mají spin a jsou to magnetické dipóly.
Spintronika se zabývá vlivem spinu na transportní vlastnosti elektronů, s vysněným cílem obejít některé technologické potíže týkající se elektroniky.
Jako velký úspěch spintroniky se považuje objevení \emph{obří magnetorezistence} (GMR)\footnote{Magnetorezistence obecně znamená změnu odporu při změně magnetizace, přívlastek obří značí jeho sílu (\SI{80}{\percent} oproti malým jednotkám procent pozorovaných před rokem 1988).} v roce 1988.
Mikroskopický základ GMR je právě interakce spinu elektronu s magnetickým polem, jedná se proto o spintronický jev.
Objev GMR umožnil výrobu dnešních hard-disků s nevídanou kapacitou a vývoj magnetoresistivních RAM (MRAM).

Protože spin elektronu souvisí s magnetickým momentem, projevuje se především v materiálech s magnetickým uspořádáním (např. feromagnetech, antiferomagnetech), spintronika proto přirozeně zahrnuje jejich vývoj a studium.
Takový vývoj ovšem vyžaduje zároveň diagnostické metody, kterými se mohou nové materiály zkoumat, vybírat z nich ty nejslibnější a optimalizovat jejich parametry.

Tato práce se věnuje právě jedné takové diagnostické metodě -- magneto-optické spektroskopii.
Magneto-optika dovoluje zkoumat magnetismus skrze jeho projevy v optických vlastnostech.
Projevy magneto-optické aktivity lze obecně rozdělit na liché a sudé, podle toho, jestli při obrácení orientace magnetického pole nebo magnetizace mění či nemění znaménko.
Zatímco liché jevy jsou široce používané už desítky let, ty sudé začaly nabývat na popularitě až v nedávné době.
Oproti lichým jevům existují ty sudé v širší třídě materiálů (např. v kolineárních antiferomagnetech), což z nich dělá jednu z mála dostupných metod jejich studia.
V současné době jsou antiferomagnety předmětem intenzivního spintronického výzkumu.

V roce 2017 byl v Laboratoři OptoSpintroniky (LOS) na Katedře CHemické Fyziky a Optiky (KCHFO) na Univerzitě Karlově (UK) uveden do provozu prototyp vektorového magnetu, jehož účelem bylo získání metody pro magnetooptickou charakterizaci vzorků v širokém rozsahu teplot a vlnových délek, včetně zaměření na studium sudých MO jevů.
Metoda využívá rotující pole v rovině vzorku a je proto koncepčně příbuzná tzv. 8-směrné metodě, metodě rotujícího pole a ROTMOKE.

Metodou bylo v minulosti studováno mnoho vzorků rozličných materiálů, avšak jakékoliv pokusy o kvantitativní interpretaci sudých jevů buď úplně selhávaly, nebo dávaly protiřečící si výsledky.
Hlavním cílem této práce tedy bylo pochopit příčinu pozorovaných problémů a v ideálním případě metodu opravit, uvést do provozu a demonstrovat její použitelnost na vzorcích materiálů slibných pro spintroniku.

A jak ukážeme v tomto textu, tohoto cíle bylo zcela dosaženo.
Výsledkem je funkční metoda, která rozšiřuje možnosti uvedených příbuzných metod, a její funkčnost byla demonstrována studiem dvou materiálů: CoFe a FeRh.
