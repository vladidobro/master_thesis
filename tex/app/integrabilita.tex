\section{Podmínka integrability magnetické energie}
\label{app:podminka-integrability}

\endinput

Zvolíme-li uzavřenou křivku v $\M$ prostoru, a budeme podél ní integrovat \eqref{e:Hext=gradF}, dostaneme
\begin{equation} \label{e:mag integral 1}
\oint \muvac\Hext(\M) \cdot \text{d}\M = \oint \nabla_{\M} F \cdot \text{d}\M = F(\M_\textrm{start})-F(\M_\textrm{end})=0 \,.
\end{equation}
Přechodem k experimentálně ovladatelným souřadnicím $\Hext$
\begin{equation} \label{e:substituce M}
\text{d}\M=\begin{pmatrix}\text{d}M_x \\ \text{d}M_y \\ \text{d}M_z \end{pmatrix}
=\begin{pmatrix}
\frac{\partial M_x}{\partial \Hx_x} & \frac{\partial M_x}{\partial \Hx_y} & \frac{\partial M_x}{\partial \Hx_z} \\
\frac{\partial M_y}{\partial \Hx_x} & \frac{\partial M_y}{\partial \Hx_y} & \frac{\partial M_y}{\partial \Hx_z} \\
\frac{\partial M_z}{\partial \Hx_x} & \frac{\partial M_z}{\partial \Hx_y} & \frac{\partial M_z}{\partial \Hx_z} \\
\end{pmatrix}
\begin{pmatrix}\text{d}\Hx_x \\ \text{d}\Hx_y \\ \text{d}\Hx_z \end{pmatrix}
\equiv\left( \frac{\text{d}\M}{\text{d}\Hext} \right) \text{d}\Hext
\end{equation}
dostáváme z \eqref{e:mag integral 1} podmínku na průběh $\M(\Hext)$ po uzavřené křivce $\Hext$ 
\begin{equation}
\oint \muvac\Hext \cdot \left( \frac{\text{d}\M}{\text{d}\Hext} \right) \text{d}\Hext = 0 \,,
\end{equation}
která platí, pokud jsme mohli provést substituci \eqref{e:substituce M}, tj. pokud je na ní $\M(\Hext)$ spojité, nedochází k přeskokům.
Striktně vzato bychom ji mohli použít i v případě, kdy dochází k vratným\footnote{Ve smyslu vratného termodynamického procesu.} přeskokům, tj. výchozí i koncové $\M$ mají stejnou volnou energii; nedochází k hysterezi.

Přímým dosazením pro případ, kdy se $\Hext$ otočí v rovině $xy$ o \SI{360}{\degree} s konstantní amplitudou a $\M$ je dáno \eqref{e:magnetizace v rovine}
\begin{equation}
    \left( \frac{\text{d}\M}{\text{d}\Hext} \right) \text{d}\Hext=\begin{pmatrix}
    -\sin \phim \\ \cos \phim \\ 0
    \end{pmatrix} \frac{\text{d}\phim}{\text{d}\phih}  \text{d}\phih \,,
\end{equation}

dostáváme

\begin{equation} \label{e:M integracni konstanta dodatek}
\muvac\Hx M_S \int_{0}^{2\pi}  \frac{\text{d}\phim}{\text{d}\phih} \sin\left(\phih-\phim\right) \text{d}\phih=0
\end{equation}
