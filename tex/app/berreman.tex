\section{Průchod a odraz při poruchách permitivity}
\label{app:berreman}

Navazujeme zde na oddíl \ref{chap:optika-v-multivrstvach} a ukážeme, jak spočítat Jonesovy transmisní a reflexní matice při malých změnách tenzoru permitivity, tj. budeme se snažit vyjádřit derivace $\delta\Tjs$ a $\delta\Rjs$ podle prvků Berremanovy matice $\delta\Delta$.
Podobný výpočet byl do jisté míry naznačen v článku \cite{bertrandGeneralAnalyticalTreatment2001}.

Omezíme se na reálný normovaný příčný vlnový vektor $N$ pro popis laserových svazků.

\subsection*{Porucha reflexní a transmisní matice}

Označíme celkovou přenosovou matici \eqref{eqn:prenosova-matice-M} v bázi módových amplitud vstupního a výstupního prostředí
\begin{equation}
    \Nfr = \mathfrak{D}_{(L)}^{-1} \Mfr  \mathfrak{D}_{(R)} \,.
\end{equation}
Také zavedeme blokové značení $2\times2$ bloků $4\times4$ matic
\begin{equation}
    \Nfr = \begin{pmatrix} \Nfr^\nwarrow & \Nfr^\nearrow \\
    \Nfr^\swarrow & \Nfr^\searrow \end{pmatrix} .
\end{equation}
S tímto značením jsou Jonesovy matice průchodu a odrazu po vyřešení rovnice \eqref{eqn:rovnice-odraz-pruchod} dány
\begin{equation}
    \Tjs = \left(\Nfr^\nwarrow\right)^{-1} \,, \quad \Rjs=\Nfr^\swarrow \mathcal{T} \,.
\end{equation}
Poruchy matice $\delta\Nfr$ se v Jonesových maticích projeví\footnote{Derivace inverzní matice $(A^{-1})'=-AA'A$.}
\begin{equation}
    \label{eqn:app-derivace-RT}
    \delta\Tjs = -\Tjs \left(\delta\Nfr^\nwarrow \right) \Tjs \,, \quad \delta\Rjs=\left(\delta\Nfr^\swarrow\right) \Tjs + \Nfr^\swarrow (\delta\Tjs)
\end{equation}

\subsection*{Porucha přenosové matice}
\subsubsection*{Sendvičová struktura}

Zde se zaměříme na výpočet $\delta\Nfr$ v případě, kdy vrstva s porušenou permitivitou není přímo výstupní prostředí.
Potom $\DfrL$ i $\DfrR$ zůstávají neměnné a $\delta\Nfr = \DfrL^{-1}\delta\Mfr\DfrR$.
Celková přenosová matice $\Mfr$ je dána součinem tří dílčích přenosových matic: nadvrstev $\Mfr_i$, zkoumané (magnetické) vrstvy s porušenou permitivitou $\Mfr_m$ a podvrstev $\Mfr_o$, takže $\Mfr = \Mfr_i \Mfr_m \Mfr_o$.
Na poruchu pak reaguje pouze $\Mfr_m$, tzn. $\delta\Mfr = \Mfr_i (\delta\Mfr_m) \Mfr_o$.

Přenosová matice je dána řešením \eqref{eqn:Berreman-master}, maticovou exponenciálou
\begin{equation}
    \Mfr_m + \delta \Mfr_m = e^{ik_0d(\Delta+\delta\Delta)} = e^{ik_0d\Delta} + \delta \Mfr_m
\end{equation}
kde $d$ je tloušťka vrstvy.

První derivaci maticové exponenciály lze spočítat podle vzorce\cite{najfeldDerivativesMatrixExponential1995a}
\begin{equation}
    \label{eqn:app-derivace}
    \delta\Mfr_m = \delta(e^{ik_0d\Delta}) = ik_0d\int_0^1 e^{ik_0d\Delta(1-\tau)} (\delta\Delta) e^{ik_0d\Delta \tau} \textrm{d}\tau \,.
\end{equation}
Není tedy třeba počítat exponenciálu porušených $\Delta+\delta\Delta$, ale pouze neporušeného $\Delta$.
Pokud dokážeme neporušenou matici $\Delta$ diagonalizovat dynamickou maticí $\Dfr$, má vzorec \eqref{eqn:app-derivace} jednodušší tvar
\begin{equation}
    \label{eqn:app-lepsi-derivace}
    \delta\Mfr_m = ik_0d \Dfr \int_0^1 e^{ik_0d (\Dfr^{-1}\Delta\Dfr)(1-\tau)} (\Dfr^{-1}\delta\Delta\Dfr) e^{ik_0d(\Dfr^{-1}\Delta\Dfr)\tau} \textrm{d}\tau \Dfr^{-1}\,,
\end{equation}
kde se už vyskytuje exponenciála pouze diagonálních matic $\Dfr^{-1}\Delta\Dfr$, která se spočítá triviálně.
Pro přehlednost explicitně spočítáme derivaci v případě, kdy neporušená permitivita je izotropní, tj. v $2\times2$ blokové notaci
\begin{equation}
    \Dfr^{-1}\Delta\Dfr = \begin{pmatrix} -n\cos\alpha_t & 0 \\ 0 & n\cos\alpha_t \end{pmatrix} ,
\end{equation}
kde $n$ je index lomu a $\alpha_t$ úhel lomu\footnote{Pro absorbující prostředí je komplexní, nicméně stále platí $\cos\alpha_t\equiv\sqrt{1-N^TN/n^2}$.}.
Pak
\begin{equation}
    \delta \exp \left[ik_0d\Delta+ \begin{pmatrix} 
\mathfrak{A}^\nwarrow & \mathfrak{A}^\nearrow \\
\mathfrak{A}^\swarrow & \mathfrak{A}^\searrow
\end{pmatrix} \right] = ik_0d \, \Dfr
\begin{pmatrix} 
    e^{-ix}\mathfrak{B}^\nwarrow & \frac{\sin x}{x}\mathfrak{B}^\nearrow \\
    \frac{\sin x}{x}\mathfrak{B}^\swarrow & e^{ix}\mathfrak{B}^\searrow
\end{pmatrix} \Dfr^{-1} \,,
\end{equation}
kde jsme označili $x = k_0 d n \cos\alpha_t$ a matice $\mathfrak{B} = (b_{ij})$ vznikne z poruchy $\mathfrak{A} = (a_{ij})$ transformací $\mathfrak{B} = \Dfr^{-1} \mathfrak{A} \Dfr$ jako \eqref{eqn:app-lepsi-derivace}.
Každý z $2\times2$ bloků je násoben svým vlastním činitelem, takže výsledek není možné zapisovat pomocí notace klasického maticového násobení.

\subsubsection*{Odraz od bulku}

Druhý případ, kdy magnetická vrstva je přímo výstupní prostředí, je složitější.
Pokud se zajímáme i o prošlé světlo, narážíme na problém, který je jinak všudypřítomný v Yehově formalismu, totiž dynamická matice je v okolí degenerací (izotropní vrstva či šíření podél optické osy) singulární.
Ta v anizotropním prostředí nutně definuje bázi Jonesových vektorů\footnote{Protože oba svazky jsou oddělené.}, což má za následek neexistenci derivace $\delta\mathcal{T}$.
Pokud bychom přece jen chtěli popsat změny prošlého světla, museli bychom ho místo módovými amplitudami (Jonesovými vektory) popisovat pomocí složek polí $\vec{E}$, $\vec{H}$, které singularitami netrpí.

Častější situace, na kterou se zde zaměříme, je taková, že se zajímáme pouze o odražené světlo (odraz na anizotropním bulku).
Zmíněná singularita $\Tjs$ se pak objevuje pouze v mezivýpočtech a je možné jí obejít.

Z přenosové matice jsou nyní neměné činitele $\DfrL$ a $\Mfr$, tzn. $\delta\Nfr=\DfrL^{-1}\Mfr(\delta\DfrR)$.
Rovnici \eqref{eqn:rovnice-odraz-pruchod} přepíšeme do vhodnějšího tvaru
\begin{equation}
    \Nfr^{-1} \begin{pmatrix} \J^i \\ \J^r \end{pmatrix}
    = \begin{pmatrix} \J^t \\ 0 \end{pmatrix} ,
\end{equation}
takže řešení pro $\J^r$ je pak
\begin{equation}
    \label{eqn:app-projekce}
    \Kfr^\swarrow \J^i + \Kfr^\searrow \J^r = 0
\end{equation}
s označením $\Kfr=\Nfr^{-1} = \DfrR^{-1}\Mfr^{-1}\DfrL^{-1}$.
Výsledná reflexní matice $\Rjs=(\Kfr^\searrow)^{-1}\Kfr^\swarrow$ má derivaci, ale nelze jí počítat tímto způsobem, protože derivace $\DfrR^{-1}$ neexistuje v okolí degenerací.
Rovnice \eqref{eqn:app-projekce} má takový fyzikální význam, že amplitudy polí odpovídající módům přicházejícím zevnitř výstupního prostředí musí být nulové, a singularity jsou do ní zaneseny kvůli sejmutí degenerace těchto dvou módů.
Řešením je přejít od dynamických matic ke spektrálním projektorům -- projektorům na podprostor tvořený zmíněnými dvěma módy.
Zjednodušeně řečeno, nahradíme $\Kfr$ v rovnici \eqref{eqn:app-projekce} takovým $\tilde{\Kfr}$, se kterým je rovnice ekvivalentní a zároveň $\tilde{\Kfr}$ netrpí singularitami.

Pokud se zajímáme o poruchy v okolí již anizotropní permitivity, je možné derivace $\DfrR$ a tedy i $\Nfr$, $\Tjs$ a $\Rjs$ počítat standardními metodami poruchového počtu pro výpočet derivací vlastních vektorů\footnote{S tím rozdílem, že $\Delta$ není hermitovská.}.

\subsection*{Dynamická matice izotropní vrstvy}

Pro izotropní prostředí s indexem lomu $n$ je Berremanova matice $\Delta$ diagonalizována dynamickou maticí
\begin{equation}
    \mathfrak{D} = \begin{pmatrix} 0&0\\0&0 \end{pmatrix} \,,
\end{equation}
takže
\begin{equation}
    \Delta = n\cos\alpha_t \cdot\mathfrak{D} \begin{pmatrix} -1&0\\0&1 \end{pmatrix} \mathfrak{D}^{-1} \,.
\end{equation}

Tato volba $\mathfrak{D}$ má oproti \eqref{eqn:D-sp} výhodu, že v okolí kolmého dopadu je nezávislá na rovině dopadu, která je v experimentu volena v podstatě náhodně.
Báze Jonesových vektorů je tak volena přirozeným způsobem: pokud je rovina dopadu ve směru $\alpha_r$, pak p-polarizace je dána $\beta=\alpha_r$.



\subsection*{Diskuze}

Po dosazení tvaru $\varepsilon(\vec{M})$ z \eqref{eqn:permitivita-kub-K} a \eqref{eqn:permitivita-kub-G-xy} pro kubický vzorek orientovaný hlavními krystalografickými směry v osách $x$, $y$, $z$ se saturovanou magnetizací v rovině $xy$ má levý spodní blok Berremanovy matice $\Delta$ \eqref{eqn:Berreman-master} zodpovědný za MLD tvar
\begin{align}
    \varepsilon^\perp - \frac{\varepsilon^\vert\varepsilon^-}{\varepsilon_{33}} =& \frac{1}{2}\left( \frac{G_s}{2}-\frac{K^2}{n^2} \right) (\sigma_1\cos2\phim + \sigma2\sin2\phim) \\
                                                                                &+ \frac{1}{2}\frac{\Delta G}{2}(\sigma_1\cos2\phim - \sigma_2 \sin2\phim) \,.
\end{align}


