\section{Detaily zpracování dat}
\label{app:zpracovani}

\subsection*{Předzpracování dat}

Měřící proceduru, v rámci které detekujeme závislost magneto-optického signálu na směru externího magnetického pole konstantní velikosti, vždy provádíme v několika navazujících cyklech.
Koncový stav cyklu je vždy shodný s počátečním ($\phih=\SI{0}{\degree}$ a \SI{360}{\degree}).
Optický můstek však trpí dlouhodobou mechanickou nestabilitou (tzv. drift), takže rozdílové napětí na konci cyklu většinou nereplikuje hodnotu na začátku.
Pro odstranění alespoň části tohoto jevu od každého cyklu odečítáme lineární funkci takovou, aby se počáteční i koncový bod rovnal.
Koncový bod je pak duplicitní, takže ho dále vyřadíme ze zpracování.
Ze všech cyklů pak vezmeme aritmetický průměr.
Pro snadné vykreslování od výsledné křivky odečteme její střední hodnotu.

Symetrizaci v magnetickém poli $\vHext$ podle \eqref{eqn:symetrizace-H} nemůžeme provést přímo, protože pro každou změřenou hodnotu $\phih$ se směr opačného změřeného pole mírně liší od $\phih+\SI{180}{\degree}$.
V takovém případě bereme jako výsledné $\phih$ aritmetický průměr obou hodnot (po odečtení \SI{180}{\degree}).

\subsection*{Vztah MO koeficientů $P$}

Převod mezi koeficienty $P_{ij}$ a $P_{+/-}$, $\pi_{+/-}$ poskytuje matice
\begin{equation}
    \begin{pmatrix} P_+ e^{i2\pi_+} \\ P_- e^{i2\pi_-} \end{pmatrix}
    = \frac{1}{2}
    \begin{pmatrix} -i & 1 & -1 & -i \\ i & -1 & -1 & -i \end{pmatrix}
    \begin{pmatrix} P_{11}\\P_{12}\\P_{21}\\P_{22} \end{pmatrix}  .
\end{equation}


\subsection*{Určení nejistot}

Nejistoty veličin určených metodou nejmenších čtverců určujeme standardním způsobem\cite{gelmanBayesianDataAnalysis2014}.
V okolí minima cílové funkce numericky spočítáme matici druhých derivací (zde určujeme koeficienty $c_i$)
\begin{equation}
    \frac{\partial^2 \mathcal{L}}{\partial c_i \partial c_j} = H_{ij} \,,
\end{equation}
kterou následně normujeme odhadem variance chyb $\sigma^2 = \mathcal{L}_\textrm{min}/(n-m)$, kde $n$ je počet měřených bodů a $m$ počet fitovaných parametrů modelu.
Matice inverzní k $H/\sigma^2$ je pak kovarianční matice, na jejíž diagonále jsou variance fitovaných parametrů.

Oblasti spolehlivosti $1\sigma$, $2\sigma$, resp. $3\sigma$ jsou přibližně\footnote{Platí přesně pro lineární regresi.} ohraničené křivkou $\mathcal{L}=\mathcal{L}_\textrm{min}(1+j/(n-m))$, kde $j=1$, $4$, resp. $9$.
