\section{Formalismus Stokesových kovektorů}
\label{app:kovektory}


Popis pomocí Stokesových vektorů je zvláště užitečný pro popis detekčních systémů, protože Stokesovy parametry jsou \emph{zobecněné intenzity}.
Použití ilustruujeme na optickém můstku z oddílu \ref{chap:mustek-kap2}, viz obr. \ref{fig:mustek-dodatek}.

\begin{figure}[htbp]
    \centering
    \missingfigure{mustek2}
    \caption{Schéma optického můstku s vyznačenými Stokesovými vektory a kovektory.}
    \label{fig:mustek-dodatek}
\end{figure}

Intenzita měřená detektorem A je daná nultou složkou dopadajícího Stokesova vektoru
\begin{equation}
    I_\textrm{A}=s_0^\textrm{A} \equiv \Dtks^\textrm{A} \cdot \Stks^\textrm{A} \,.
\end{equation}
kde jsme označili $\Dtks^\textrm{A}=(1, 0, 0, 0)$ ``Stokesův kovektor`` detektoru A. 
Tečka ($\cdot$) značí kontrakci\footnote{``Skalární součin''.}.

Stejně jako můžeme světelný svazek popisovat Stokesovými vektory v různých místech jeho dráhy, můžeme i signál měřený detektory popisovat vzhledem ke Stokesovým vektorům v různých místech.
Vyjádřením $\Stks^\textrm{A}$ pomocí Stokesova vektoru vstupujícího do můstku $\Stks^\textrm{in}$ a Muellerových matic půlvlnné destičky a polarizačního děliče
\begin{equation}
    I_\textrm{A}= \Dtks^\textrm{A} \cdot \left( \M_\textrm{A} \M_{\lambda/2} \Stks^\textrm{in} \right) = \left( \Dtks^\textrm{A} \M_\textrm{A} \M_{\lambda/2}\right) \cdot \Stks^\textrm{in} \equiv {\Dtks''}^\textrm{A} \cdot \Stks^\textrm{in}  \,,
\end{equation}
kde jsme druhou rovností naznačili asociativitu maticového násobení, která nás motivovala k definici Stokesova kovektoru detektoru A vzhledem ke světlu vstupujícímu do můstku ${\Dtks''}^\textrm{A}$.

Stejně lze zřejmě činit i pro detektor B.
Rozdílové a součtové napětí je pak
\begin{align}
\label{eqn:af}
\Udif = \left( {\Dtks'}^\textrm{A}-{\Dtks'}^\textrm{B}\right) \cdot \Stks^\textrm{in} \equiv {\Dtks''}^\textrm{A-B}\cdot \Stks^\textrm{in} \,, \\
\Usum = \left({\Dtks'}^\textrm{A}+{\Dtks'}^\textrm{B}\right) \cdot \Stks^\textrm{in} \equiv {\Dtks''}^\textrm{A+B}\cdot \Stks^\textrm{in} \,.
\end{align}

Měřené signály jsou lineární ve vstupních Stokesových vektorech a jsou tedy reprezentovány lineárními formamy, zde ${\Dtks''}^\textrm{A-B}$ a ${\Dtks''}^\textrm{A+B}$.
Pro ideální půlvlnnou destičku a polarizační dělič jako v oddílu \ref{chap:mustek-kap2} jsou Muellerovy matice
\begin{align}
    \M_\textrm{A} = \begin{pmatrix} 1&1&0&0 \\ 1&1&0&0 \\ 0&0&0&0 \\ 0&0&0&0 \end{pmatrix} \,,\quad 
    \M_\textrm{B} = \begin{pmatrix} 1&-1&0&0 \\ -1&1&0&0 \\ 0&0&0&0 \\ 0&0&0&0 \end{pmatrix} \,, \\
    \M_{\lambda/2} = \begin{pmatrix} 0&0&0&0 \\ 0&\cos\theta_{\lambda/2}&\cos\theta_{\lambda/2}&0 \\ 0&\cos\theta_{\lambda/2}&\cos\theta_{\lambda/2}&0 \\ 0&0&0&1 \end{pmatrix}
\end{align}
a Stokesovy kovektory jsou po dosazení
\begin{align}
    \Dtks^\textrm{A-B}=(0, \cos\theta_{\lambda/2}, \cos\theta_{\lambda/2}, 0) \,, \label{eqn:Uab-Mueller} \\
    \Dtks^\textrm{A+B}=(1, 0, 0, 0) \,.
\end{align}

Místo sledování, co se při průchodu optickými prvky děje se všemi myslitelnými vstupními Stokesovými vektory, je výhodnější sledovat, jak se mění konstantní detektorové kovektory.
Navíc Stokesovy kovektory lze pro plně polarizováné světlo přímočaře graficky znázorňovat v 3-D prostoru redukovaných Stokesových vektorů.
Pokud je jediná nenulová složka kovektoru $d_0$ ta nultá (jako v $\Dtks^\textrm{A+B}$), pak je měřený signál jednoduše úměrný vstupní intenzitě (vzdálenosti redukovaného Stokesova vektoru od počátku) a netřeba ho zvlášť znázorňovat.
Pokud je naopak $d_0=0$ (jako v $\Dtks^\textrm{A+B}$), pak je signál lineární v redukovaném Stokesově vektoru a tedy úměrný vzdálenosti od určité roviny.

V obecném případě, kdy $d_0$ není ani nulové, ani jediné nenulové, je znázornění mírně složitější.
Nejedná se o pouhé posunutí roviny, od které je vzdálenost odečítána.
Na obr. \ref{fig:mustek-znazorneni-kovektoru} (a) jsou vyznačené nulové plochy (množina Stokesových vektorů, pro které $\Udif=0$) $\Dtks^\textrm{A-B}$ tak, jak je v \eqref{eqn:Uab-Mueller} a v situaci, kdy $d_0$ je nenulové.
Pro nenulové $d_0$ je nulová plocha kužel.

Pro body neležící na nulovém kuželu však $\Udif$ není určeno vzdáleností od kužele.
Pro body na Poincarého sféře je signál úměrný vzdálenosti od roviny, která vznikne průnikem nulového kuželu s Poincarého sférou (vždy kružnice); stejnou konstrukci lze provést i pro body mimo Poincarého sféru ($I\neq1$), posunutí roviny však závisí na $I$.

Uvedený postup selhává, pokud $d_0>|\vec{d}|\equiv|(d_1, d_2, d_3)|$, protože pak neexistují nenulové vektory, pro které je signál nulový; neexistuje nulová plocha.
Nicméně pro každé konstantní $I$ je stále signál úměrný vzdálenosti od určité roviny, tentokrát však ležící mimo sféru vektorů s intenzitou $I$.

Z provedené diskuze vyplývá jako nejvýhodnější znázorňovat Stokesovy kovektory jednotně pomocí \emph{nulové kružnice}, která je průsečíkem nulového kužele s Poincarého sférou, a jejím normálovým 3-vektorem $\vec{d}$, kterým je znázorněna konstanta úměrnosti mezi signálem a grafickou vzdáleností.
Případ $d_0>|\vec{d}$ v této práci nepoužíváme (formalismus používáme pro popis optického můstku, ve kterém se takové kovektory nevyskytují), ale znázorňovali bychom ho také 3-vektorem $\vec{d}$ a ``nulovou''\footnote{Která vzhledem k tomu, že leží mimo Poincarého sféru, odpovídá nedosažitelným Stokesovým vektorům $|\vec{s}|>s_0$.} rovinou příslušnou Poincarého sféře.
Viz obr. \ref{fig:mustek-znazorneni-kovektoru} (b).

\begin{figure}[htbp]
    \centering
    \missingfigure{kovektory}
    \caption{(a) Nulové plochy Stokesových kovektorů s nulovou/nenulovou nultou složkou. (b) Kanonické znázornění stejných kovektorů pomocí nulových kružnic.}
    \label{fig:mustek-znazorneni-kovektoru}
\end{figure}

Při znázornění kovektorů tímto způsobem lze beze změny převzít znázornění akce Muellerových matic pomocí mapování Poincarého sféry.
Protože nulové kružnice jsou množiny Stokesových vektorů, je možné je přímo zobrazit metodou vyloženou v oddílu \ref{chap:Stokes-Mueller} a výsledkem bude opět podmnožina nulové plochy.
V případě čistých retardérů se nulová kružnice zobrazí jako jiná kružnice na Poincarého sféře, a je to tedy přímo nulová kružnice transformovaného kovektoru; vektor $\vec{d}$ se otáčí společně s kružnicí.
Pokud však Muellerovou maticí dochází také k deformaci a posunutí Poincarého sféry (jako např. v případě čistého diatenuátoru), je obrazem zobrazení obecně elipsa ležící uvnitř Poincarého sféry, a je proto nutné provést následující konstrukci pro získání nulové kružnice.
Transformovaná elipsa vždy leží na nulovém kuželu, takže ho můžeme určit proložením elipsy kuželem s vrcholem v počátku.
Nulová kružnice je pak určena průnikem kuželu a Poincarého sféry.
Konstrukce je znázorněna na obr. \ref{fig:kovektor-akce-H}.\todo{mozna jeste co se deje s vec d}

\begin{figure}[htbp]
    \centering
    \missingfigure{konstrukce H kruznice}
    \caption{Transformace Stokesova kovektoru neunitárním optickým prvkem. Muellerova matice deformuje Poincarého sféru na elipsoid, takže nulová kružnice se zobrazí na elipsu uvnitř sféry. Elipsou je proložen kužel a průnikem se sférou je určena nová transformovaná nulová kružnice.}
    \label{fig:kovektor-akce-H}
\end{figure}
