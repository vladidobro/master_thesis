\section{Magneto-optické tenzory}
\label{chap:magnetoopticke-tenzory}

Všechny magneto-optické (MO) jevy lze v principu vysvětlit závislostí optických parametrů na magnetickém stavu\cite{silberQuadraticMagnetoopticKerr2019a}.
V našem modelu je jediným materiálovým parametrem tenzor relativní permitivity $\varepsilon$.
Zaměříme se na feromagnety, závislost na magnetickém stavu značíme $\varepsilon(\vec{M})$.
Obecně je možné rozdělit závislost do tří příspěvků
\begin{equation}
    \varepsilon(\vec{M})=\varepsilon^{(0)} + \varepsilon^{(-)}(\vec{M}) + \varepsilon^{(+)}(\vec{M}) \,,
\end{equation}
kde $\varepsilon^{(0)}$ je nemagnetická/strukturální permitivita\footnote{``Jako kdyby $\vec{M}=0$.''},
$\varepsilon^{(-)}(\vec{M})=-\varepsilon^{(-)}(-\vec{M})$ je permitivita lichá v magnetizaci
a $\varepsilon^{(+)}(\vec{M})=\varepsilon^{(+)}(-\vec{M})$ je permitivita sudá v magnetizaci.

Z termodynamických úvah plynou Onsagerovy relace reciprocity\cite{onsagerReciprocalRelationsIrreversible1931a,onsagerReciprocalRelationsIrreversible1931} pro $\varepsilon(\vec{M})$
\begin{equation}
    \varepsilon_{ij}(\vec{M})=\varepsilon_{ji}(-\vec{M}) \,,
\end{equation}
takže $\varepsilon^{(0)}$ a $\varepsilon^{(+)}$ jsou symetrické, zatímco $\varepsilon^{(-)}$ je antisymetrický.

Magnetická závislost permitivity se obvykle rozvíjí do mocninné řady v $\vec{M}$ \cite{visnovskyOpticsMagneticMultilayers2018}
\begin{align} 
\label{eqn:MO-tenzory}
    \varepsilon_{ij}(\vec{M})&=\varepsilon^{(0)}_{ij}
        + \sum_{k=1}^{3}\left[ \frac{\partial \varepsilon_{ij}}{\partial M_k}\right]_{\vec{M}=0} M_k 
        + \sum_{k,l=1}^{3} \frac{1}{2}\left[ \frac{\partial^2 \varepsilon_{ij}}{\partial M_k \partial M_l}\right]_{\vec{M}=0} M_k M_l + \dots 
    \\ &=\varepsilon^{(0)}_{ij} 
        + \sum_{k=1}^{3}K_{ijk} M_k 
        + \sum_{k,l=1}^{3} G_{ijkl} M_k M_l + \dots
        =\varepsilon^{(0)}_{ij} +\varepsilon^{(1)}_{ij} +\varepsilon^{(2)}_{ij} + \dots
\end{align}
kde jsme explicitně uvedli první dva řády, které definují \emph{lineární MO tenzor} $\mathbb{K}$ a \emph{kvadratický MO tenzor} $\mathbb{G}$.
Vyšší řády se většinou zanedbávají.
Je dobré ale mít na paměti, že zakončením rozvoje na určitém řádu nejen snižujeme přesnost, ale také uměle zvyšujeme symetrii $\varepsilon(\vec{M})$ \cite{silberQuadraticMagnetoopticKerr2019a}.
To je nejlépe nahlédnout např. u materiálu se šesterečnou symetrií v rovině $xy$: $\mathbb{G}$ tenzor je v rovině $xy$ izotropní, ale permitivita 6. řádu už má ``správnou'' šesterečnou symetrii.
MO tenzory $\mathbb{K}$ a $\mathbb{G}$ nedokáží popsat Voigtův jev (viz kap. \ref{chap:2}) se šesterečnou symetrií, proto je třeba mít se na pozoru a v případě takového kvalitativního důkazu do rozvoje přidat další členy.

MO tenzory se musí podřizovat stejným symetriím jako materiál, který popisují, což je spolu s Onsagerovými relacemi poměrně silně omezuje.
Tvar $\mathbb{K}$ a $\mathbb{G}$ pro všechny krystalografické třídy je uveden v \cite{visnovskyOpticsMagneticMultilayers2018}.
Dále uvedeme MO tenzory pro izotropní a kubický (krystalové třídy $\bar{4}3m, 432, m3m$) materiál s krystalografickými osami ve směrech souřadných os \cite{hamrlovaQuadraticinmagnetizationPermittivityConductivity2013,visnovskyOpticsMagneticMultilayers2018}.
Izotropní i kubický materiál mají izotropní nemagnetickou permitivitu (index lomu $n$)
\begin{equation}
    \varepsilon^{(0)}=\begin{pmatrix}
        n^2 & 0 & 0 \\ 0 & n^2 & 0 \\ 0 & 0 & n^2
    \end{pmatrix} \,,
\end{equation}
oba také mají izotropní $\mathbb{K}$
\begin{equation}
    \label{eqn:permitivita-kub-K}
    \begin{pmatrix} \varepsilon^{(1)}_{yz}=-\varepsilon^{(1)}_{zy} \\ \varepsilon^{(1)}_{zx}=-\varepsilon^{(1)}_{xz} 
        \\ \varepsilon^{(1)}_{xy}=-\varepsilon^{(1)}_{yx}\end{pmatrix}
    =\begin{pmatrix} K & 0 & 0 \\ 0 & K & 0 \\ 0 & 0 & K \end{pmatrix}
    \begin{pmatrix}M_x \\ M_y \\ M_z\end{pmatrix} 
    \,, \,\, \varepsilon^{(1)}= K 
    \begin{pmatrix} 0 & M_z & -M_y \\ -M_z & 0 & M_x \\ M_y & -M_x & 0 \end{pmatrix} \,,
\end{equation}
ale v druhém řádu už se liší.
Pro kubický materiál platí (používáme 2-indexovou notaci jako \cite{hamrlovaQuadraticinmagnetizationPermittivityConductivity2013})
\begin{equation}
    \begin{pmatrix}
        \varepsilon^{(2)}_{xx} \\ \varepsilon^{(2)}_{yy} \\ \varepsilon^{(2)}_{zz} 
        \\ \varepsilon^{(2)}_{yz}=\varepsilon^{(2)}_{zy} 
        \\ \varepsilon^{(2)}_{zx}=\varepsilon^{(2)}_{xz} 
        \\ \varepsilon^{(2)}_{xy}=\varepsilon^{(2)}_{yx}
    \end{pmatrix}
    =\begin{pmatrix}
        G_{11} & G_{12} & G_{12} & 0 & 0 & 0 \\
        G_{12} & G_{11} & G_{12} & 0 & 0 & 0 \\
        G_{12} & G_{12} & G_{11} & 0 & 0 & 0 \\
        0 & 0 & 0 & 2G_{44} & 0 & 0 \\
        0 & 0 & 0 & 0 & 2G_{44} & 0 \\
        0 & 0 & 0 & 0 & 0 & 2G_{44}
    \end{pmatrix}
    \begin{pmatrix} M_x^2 \\ M_y^2 \\ M_z^2 \\ M_y M_z \\ M_z M_x \\ M_x M_y \end{pmatrix} \,,
\end{equation}
pro izotropní navíc $\Delta G \equiv G_{11}-G_{12}-2G_{44}=0$.
Pro pozdější použití uvedeme pro speciální případ $M_z=0$
\begin{align}
    \label{eqn:permitivita-kub-G-xy}
    \varepsilon^{(2)} =& G_{12} |\vec{M}|^2 \\ 
    &+ \frac{G_s}{2} \begin{pmatrix}
        M_x^2 & M_x M_y & 0 \\ M_x M_y & M_y^2 & 0 \\ 0 & 0 & 0
    \end{pmatrix}
    + \frac{\Delta G}{2} \begin{pmatrix}
        M_x^2 & -M_xM_y & 0 \\ -M_xM_y & M_y^2 & 0 \\ 0 & 0 & 0
    \end{pmatrix}
\end{align}
kde $G_s=G_{11}-G_{12}+2G_{44}$.
Pro úplnost připomeneme, že složky MO tenzorů jsou stejně jako relativní permitivita $\varepsilon$ komplexní, bezrozměrné a frekvenčně závislé.
Pokud je materiál dobře popsaný MO tenzory, lze pro libovolné $\vec{M}$ dosazením do \eqref{eqn:MO-tenzory} získat $\varepsilon$, aplikovat metodu z oddílu \ref{chap:optika-v-multivrstvach} a tak spočítat všechny myslitelné transmisní a reflexní koeficienty.

Uvedený přístup není možné použít v případě, že osvětlované místo vzorku není tvořené homogenním $\vec{M}$, ale je tvořené více doménami, ve kterých se $\vec{M}$ liší.
$\vec{M}$ je ve více-doménovém stavu dané průměrem přes domény.
Pro $\mathbb{K}$ to nečiní problém, protože pro průměrné $\vec{M}$ dostaneme průměrné $\varepsilon$, ale pro $\mathbb{G}$ to již neplatí.
