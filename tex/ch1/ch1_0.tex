Celá práce se zabývá jedním typem experimentu, a to tím nejzákladnějším magneto-optickým experimentem, který je schematicky znázorněn na obr. \ref{fig:zakladni-schema}.

\begin{figure}[htbp]
    \centering
    \missingfigure{schema}
    \caption{Jednoduché schéma experimentu. 
    Laserový svazek je vhodně upraven vstupní optikou, dále dopadá na vzorek, kterým je propuštěn/odražen, nakonec je upraven výstupní optikou (a případně rozdělen do více ramen) a je detekována jeho intenzita. 
    Vzorek je umístěn v kryostatu a mezi rameny elektromagnetu, takže během experimentu mu vnucujeme libovolnou teplotu a magnetické pole.}
    \label{fig:zakladni-schema}
\end{figure}

Tato kapitola shrnuje relevantní fyzikální modely a jevy.
Ve zkratce:
\begin{itemize}
    \item Světlo je popsáno Maxwellovými rovnicemi.
        \begin{itemize}
            \item 
                Je monochromatické na kruhové frekvenci $\omega$. 
                K popisu polarizačního stavu používáme Jonesovy a Stokesovy vektory.
                Dráha svazku je rozdělena na posloupnost koherentních a nekoherentních bloků\footnote{Podle toho, jestli jsou vícenásobné odrazy koherentní a interferují s dopadajícím světlem. Např. vícenásobné odrazy mezi různými optickými prvky jsou nekoherentní, protože jsou prostorově odděleny. Naopak uvnitř vzorku složeného z tenkých vrstev je nutné vícenásobné odrazy považovat za koherentní.}.
                Všechny optické prvky jsou lineární a nedepolarizační, takže každý koherentní blok (např. fázová destička, vzorek) je popsán Jonesovou a Muellerovou maticí.
                Při popisu optických prvků neúnavně využíváme grafické znázornění Muellerových matic pomocí tzv. \emph{charakteristických elipsoidů}.
            \item 
                Interakce světla s látkou je pouze skrze vektory elektrické polarizace $\vec{P}$, magnetizace $\vec{M}$ a proudové hustoty $\vec{j}$. 
                Odezva materiálů je lineární a lokální, a je obecně anizotropní. 
                Navíc magnetická odezva je na vysokých optických frekvencích nulová, takže píšeme $\vec{M}(\vec{r},\omega)=0$, $\vec{D}(\vec{r},\omega)=\varepsilon{\textrm{vac}}\varepsilon_{\textrm{el}}(\vec{r},\omega)\vec{E}(\vec{r},\omega)$, $\vec{j}(\vec{r},\omega)=\sigma(\vec{r},\omega)\vec{E}(\vec{r},\omega)$, s komplexními tenzory permitivity $\varepsilon_{el}$ a vodivosti $\sigma$, které se v maxwellových rovnicích projeví souhrně jako tenzor efektivní permitivity $\varepsilon$.
            \item
                Uvnitř koherentního bloku (vzorek složený z planárních tenkých vrstev) je nutné řešit Maxwellovy rovnice, výsledkem je opět Jonesova a Muellerova matice.
                Výpočet je proveden v rámci Berremanova formalismu\todocite{} $4\times 4$ přenosových matic pro anizotropní vrstevnaté prostředí, který je ekvivalentní Yehově $4\times 4$ maticové algebře\todocite{}, ale netrpí některými technickými nedostatky\todocite{}.
        \end{itemize}
        
    \item 
        Magnetické látky mají dodatečný magnetický termodynamický stupeň volnosti (parametr uspořádání).
        Magneto-optická aktivita je modelována jako závislost efektivní permitivity $\varepsilon$ na tomto magnetickém stupni volnosti.
        \begin{itemize}
            \item 
                U feromagnetů je to celková magnetizace $\vec{M}$. 
                Feromagnety reagují na vnější magnetické pole změnou rovnovážné magnetizace, která je daná magnetickou volnou energií (magnetickou anizotropií).
                Pro relevantní případ tenkých vzorků používáme Stonerův-Wohlfarthův model\todocite{} pro vzorek v jedno-doménovém stavu, předpokládáme konstantní velikost magnetizace a nulový průmět do osy vzorku (magnetizace je tyv. \emph{in-plane}).
                Efektivní permitivita je funkcí $\vec{M}$ a je aproximována prvními členy Taylorovy řady, koeficienty tvoří složky lineárního $\mathbb{K}$ a kvadratického $\mathbb{G}$ magneto-optického tenzoru.
            \item
                U kolineárního antiferomagnetu FeRh (viz kap. \ref{chap:vzorek-ferh}) je to magnetizace podmřížky (Néelův vektor\todo{pravda?} $\vec{L}$).
                Na vnější magnetické pole přímo nereaguje, ale polohu Néelova vektoru lze ovlivnit pomocí efektu známéno jako \emph{field cooling}, tedy chlazení při vnějším poli.
                I antiferomagnet může být magneticky anizotropní, opět předpokládáme in-plane $\vec{L}$ o konstantní velikosti.
                Podobně jako u feromagnetů je $\varepsilon$ funkcí $\vec{L}$, až na to, že lineární člen je zakázán symetrií\footnote{Podmřížky s opačnou magnetizací jsou ekvivalentní, což není pravda u všech antiferomagnetů.}.
        \end{itemize}
\end{itemize}
