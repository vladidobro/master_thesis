V této kapitole shrneme fyziku

\oddelovac 

Cílem této kapitoly je podat čtenáři ucelený přehled relevantní fyziky, abychom mohli v pozdějších kapitolách odvodit a obhájit matematický popis experimentu --- k čemu \emph{přesně} se vztahují čísla na displejích lockinů?
I v tomto ohledu již vzdělaný čtenář si možná přijde na své, protože v některých směrech volíme mírně nestandardní přístup:

\begin{itemize}
\item
Stokesovy parametry a Muellerovy matice --- elegantní a názorný popis polarizace a jejich změn po působení polarizačních prvků.
Oproti standardně používanému formalismu Jonesových vektorů a matic mají kromě názornosti ještě jednu velkou výhodu: Stokesovy parametry jsou zobecněné intenzity.
V experimentu \footnote{kterému se věnuje tato práce} reálně vždy měříme pouze intenzitu, takže měřený signál je lineární v Stokesových parametrech (oproti kvadratickým formám v Jonesových vektorech).
S Muellerovými maticemi je popis měřící aparatury velice přirozený a názorný, a práce v laboratoři snadnější a přijemnější.

\item
K výpočtu transmisních a reflexních matic anizotropních planárních struktur se v magnetooptice tradičně citeBerremanperturb využívá Yehův formalismus citeYeh, který vychází z vlnové rovnice pro $E$ -- vektorové diferenciální rovnice \emph{druhého} řádu.
Magnetooptika se často zajímá o materiály, které jsou téměř izotropní, a předmětem zájmu jsou malé odchylky od izotropie.
Yehova metoda v sobě jako klíčový krok zahrnuje řešení kvartických rovnic a výpočet vlastních vektorů, tyto dvě operace se však nejhůře chovají právě v takových situacích, v okolí degenerací jsou dokonce singulární cite{Yehsing1}.
Do problému jsou zavedené uměle, měřitelné transmisní a reflexní matice žádnými takovými problémy netrpí.
Pro počítačové výpočty konkrétních situací význam nemá, ale výpočet analytických vzorců nutných pro pochopení a interpretaci experimentu je tím výrazně ztížencite{Vispolar}.
Berremanův formalismus založený přímo na Maxwellových rovnicích \emph{prvního} řádu se obloukem vyhýbá všem problémům spojených s Yehovým formalismem.
\end{itemize}

V rovnicích velice často využíváme úspornosti zápisu pomocí implicitního maticové násobení.
Sloupcové vektory značíme tlustým písmem a řádkové vektory pomocí transpozice (či hermitovského sdružení) sloupcového vektoru.
Maticovou povahu každého objektu vždy definujeme v textu, ale dále explicitně nezdůrazňujeme speciální notací.
Užitečnou zásadou při čtení této práce je o každém objektu (i o číslech) předpokládat, že se jedná o matici (násobek jednotkové matice).
Většina čísel jsou navíc komplexní, což také nezdůrazňujeme notací.
Při pochybách odkazujeme na dodatek refk:dodatek konvence

Pokud v názvu oddílu uvádíme odkaz na literaturu, vycházíme v něm z uvedeného zdroje, avšak notaci a konvence přizpůsobujeme konvencím této práce.
