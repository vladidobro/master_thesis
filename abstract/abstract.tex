\documentclass[12pt]{report}

\usepackage[a4paper, hmargin=1in, vmargin=1in]{geometry}
\usepackage[a-2u]{pdfx}
\usepackage[czech]{babel}
\usepackage{luavlna}

\begin{document}

V Laboratoři OptoSpintroniky je dlouhodobě vyvíjena experimentální metoda pro studium magnetických vzorků pomocí magneto-optických jevů kvadratických v magnetizaci vzorku, jako je například Voigtův jev.
Vlivem použité experimentální geometrie je v našem případě, na rozdíl od obdobných metod, možné používat také kryostat.
Díky tomu je možné příslušné vzorky studovat jak za snížené, tak i za zvýšené teploty.
V rámci této diplomové práce bylo identifikováno a následně odstraněno několik problémů, které praktické využití této metody doposud znemožňovaly.
Použití metody bylo demonstrováno v transmisní i reflexní geometrii pro feromagnetické vzorky CoFe a FeRh.
Naše měření ukázala, že koeficient popisující kvadratickou magneto-optickou odezvu může být silně anizotropní, přičemž velikost této anizotropie i její znaménko silně závisí na použité vlnové délce světla.
To má poměrně zásadní důsledky pro plánování a/nebo vyhodnocování příslušných experimentů využívajících kvadratickou magneto-optiku, která v současné době přitahuje rostoucí pozornost díky nástupu zájmu o antiferomagnetickou spintroniku.

\end{document}
